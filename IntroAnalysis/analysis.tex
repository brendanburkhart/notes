\documentclass[12pt]{article}

\usepackage{amsmath}
\usepackage{amssymb}
\usepackage{amsthm}
\usepackage{fullpage}
\usepackage{makecell}
\usepackage{tabularx}
\usepackage[hypcap=false]{caption}
\usepackage{tikz}
\usepackage{titling}
\usepackage{pdfpages}
\usepackage{enumitem}

\usepackage{common}

\begin{document}

\title{Introduction to Analysis Notes}
\date{}
\author{Brendan Burkhart}
\maketitle

\tableofcontents
\newpage

\section{Sets and Lists}

\begin{defn}\label{set}
    A set is an unordered group of distinct elements.
\end{defn}

\begin{defn}\label{empty-set}
    The empty set (denoted $\emptyset$) is the unique set having no elements.
\end{defn}

While sets are unordered groups of distinct elements, lists (also called $n$-tuples) are ordered groups of elements which are not necessarily distinct. An ordered pair $(a, b)$ is a list of a length two (a tuple), where $a$, and $b$ are elements of some set. 

\begin{defn}\label{tuple}
    An ordered pair $(a, b)$ is a tuple of elements of some set.
\end{defn}

Ordered pairs (and $n$-tuples more generally) can be represented as sets themselves - the pair $(a, b)$ can be represented as the set $\left\{a, \{a, b\}\right\}$.

\begin{defn}\label{cartesian-product}
    The Cartesian product of two sets $A$ and $B$ is denoted $A \times B$. It is equal to $\left\{\left(a, b\right) \compbar a \in A, b \in B \right\}$.
\end{defn}

\section{Boolean Algebra}

Boolean algebra is the algebra dealing exclusively with the values \emph{true} and \emph{false}.

The primary operations of Boolean algebra are \emph{negation} (also called \emph{not}) denoted by $\neg$, \emph{conjunction} (also called \emph{and}) denoted by $\land$, and \emph{disjunction} (also called \emph{or}) denoted by $\lor$.

Since Boolean algebra has only two elements, it is possible to enumerate all variable combinations for a function. This is often done in the form of a truth table --- a table listing the values of variables and the corresponding function value as rows. For example, Table \ref{primary-operations} gives a combined truth table for negation, conjunction, and disjunction. It also serves as the definition of these operations.

\begin{center}
    \captionof{table}{Truth table of primary operations}
    \label{primary-operations}
    \begin{tabularx}{\linewidth}{|X|X|X|X|X|}
        \hline
        \thead{$X$} & \thead{$Y$} & \thead{$\neg X$} & \thead{$X \land Y$} & \thead{$X \lor Y$} \\
        \hline
        True  & True & False & True & True \\
        \hline
        True  & False & False & False & True \\
        \hline
        False  & True & True & False & True \\
        \hline
        False  & False & True & False & False \\
        \hline
    \end{tabularx}
\end{center}

\end{document}
