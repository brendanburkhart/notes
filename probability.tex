\chapter{Probability}
\label{ch:probability}

\section{Axioms of Probability}

\begin{defn}\proofbreak
    \begin{itemize}
        \item Sample point: a possible outcome of a probabilistic experiment, often denoted by $\omega$.
        \item Sample space: the set of all sample points, often denoted by $\Omega$.
        \item Event: any subset of the sample space.
    \end{itemize}
\end{defn}

\begin{defn}
    If events $A_i \subseteq \Omega$ are disjoint, we say that these events are \emph{mutually exclusive}.
\end{defn}

\begin{defn}\label{kolmogorov-probability-axioms}
    A \emph{probability space} is a sample space $\Omega$ together with a function $P: \mathcal{P}(\Omega) \to \R$ that satisfies the following axioms:
    \begin{itemize}
        \item Non-negativity: for any event $A \subseteq \Omega$, $P(A) \geq 0$.
        \item Normalization: $P(\Omega) = 1$.
        \item Countable-additivity: if $A_i$ are a countable sequence of mutually exclusive events, then $P(\disjointunionbig_{i}A_i) = \sum_{i}P(A_i)$.
    \end{itemize}
\end{defn}

\begin{rmk}
    These axioms are due to Andrey Kolmogorov.
\end{rmk}

\begin{prop}
    Let $(\Omega, P)$ be a probability space, and $A \subseteq \Omega$ an event. Then since $A$ and $A^{c}$ are mutually exclusive such that $A \union A^{c} = \Omega$,
    \begin{itemize}
        \item $P(A) + P(A^{c}) = 1$,
        \item
    \end{itemize}
\end{prop}

\begin{proof}
    $P(A) + P(A^{c}) = P(A \union A^{c}) = P(\Omega) = 1$.
\end{proof}

\begin{cor}
    $P(\emptyset) = 0$.
\end{cor}

\begin{prop}Monotonicity\label{probability-monotonicity}\proofbreak
    Let $A, B \subseteq \Omega$ be events such that $A \subseteq B$. Then $P(A) \leq P(B)$.
\end{prop}

\begin{proof}
    Let $C = B - A$. Then $A \union C = B$ and $A \intersection C = \emptyset$. Since $A$ and $C$ are therefore mutually exclusive, $P(A) + P(C) = P(A \union C) = P(B)$. Since $P(C) \geq 0$ by the axiom of non-negativity, it follows that $P(A) \leq P(B)$.
\end{proof}

\begin{thm}{Inclusion-exclusion Principle}\label{inclusion-exclusion}
    Let $A, B, C \subseteq \Omega$ be events. Then
    \[P(A \union B) = P(A) + P(B) - P(A \intersection B),\]
    and
    \begin{align*}
        P(A \union B \union C) &= P(A) + P(B) + P(C) \\
                               &- P(A \intersection B) - P((A \intersection C) - P(B \intersection C) \\
                               &+ P(A \intersection B \intersection C).
    \end{align*}
\end{thm}

\begin{proof}
    Since $A \union B = (A - B) \union (B - A) \union (A \intersection B)$, which are necessarily mutually exclusive events, we have
    \[P(A \union B) = P(A - B) + P(B - A) + P(A \intersection B).\]
    Now note that $A = (A - B) \union (A \intersection B)$ and $B = (B - A) \union (A \intersection B)$, and so $P(A) = P(A - B) + P(A \intersection B)$ and $P(B) = P(B - A) + P(A \intersection B)$. Therefore, \begin{align*}
        P(A \union B) &= P(A - B) + P(B - A) + P(A \intersection B) \\
                      &= \big[P(A - B) + P(A \intersection B)\big] + \big[P(B - A) + P(A \intersection B)\big] - P(A \intersection B) \\
                      &= P(A) + P(B) - P(A \intersection B).
    \end{align*}

    To prove the three-way version of the principle, we can apply the two-way version to $A \union B$ and $C$. This gives us
    \begin{align*}\label{inclusion-exclusion-three-intermediate}\tag{$1$}
        P(A \union B \union C) &= P(A \union B) + P(C) - P((A \union B) \intersection C) \\
        &= \big[P(A) + P(B) - P(A \intersection B)\big] + P(C) - P((A \union B) \intersection C).
    \end{align*}
    Since $(A \union B) \intersection C = (A \intersection C) \union (B \intersection C)$, we can apply the principle again to find that
    \[P((A \union B) \intersection C) = P(A \intersection C) + P(B \intersection C) - P((A \intersection C) \intersection (B \intersection C)).\] Noting that $(A \intersection C) \intersection (B \intersection C) = A \intersection B \intersection C$ and substituting this back into \ref{inclusion-exclusion-three-intermediate}, we obtain
    \begin{align*}
        P(A \union B \union C) &= P(A) + P(B) + P(C) - P(A \intersection B) - \big[P(A \intersection C) + P(B \intersection C) - P(A \intersection B \intersection )\big] \\
        &= P(A) + P(B) + P(C) \\
        &- P(A \intersection B) - P((A \intersection C) - P(B \intersection C) \\
        &+ P(A \intersection B \intersection C).
    \end{align*}
\end{proof}
