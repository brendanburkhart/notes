\chapter{Discrete Mathematics}
\label{ch:discrete}

\section{Binomial Coefficients}

\begin{defn}\label{binomial-coefficient}
Let $n, k \in \N$. The \emph{binomial coefficient} $\binom{n}{k}$ is the number of subsets with cardinality $k$ of a set with cardinality $n$.
\end{defn}

\begin{exmp}\proofbreak
    \begin{multicols}{3}
        \begin{itemize}
            \item $\binom{0}{0} = 1$
            \item $\binom{1}{0} = 1$
            \item $\binom{1}{1} = 1$
            \item $\binom{2}{1} = 2$
        \end{itemize}

        \columnbreak

        \begin{itemize}
            \item $\binom{3}{2} = 3$
            \item $\binom{4}{2} = 6$
            \item $\binom{5}{0} = 1$
            \item $\binom{5}{1} = 5$
        \end{itemize}

        \columnbreak

        \begin{itemize}
            \item $\binom{5}{2} = 10$
            \item $\binom{5}{3} = 10$
            \item $\binom{5}{4} = 5$
            \item $\binom{5}{5} = 1$
        \end{itemize}
    \end{multicols}
\end{exmp}

\begin{prop}\label{binomial-complement}
    Let $n, k \in \N$ with $0 \leq k \leq n$. Then \[\binom{n}{k} = \binom{n}{n-k}.\]
\end{prop}

\begin{proof}
    Any selection of $k$ elements from a set with $n$ elements is equivalent to the selection of the other $n-k$ elements.
\end{proof}

\begin{exmp}
    $\binom{5}{2} = \binom{5}{3}$, and $\binom{3}{1} = \binom{3}{2}$.
\end{exmp}

\begin{defn}\label{pascals-triangle}
    \emph{Pascal's Triangle} is a particular way of writing out what is essentially a table of binomial coefficients that highlights several properties. Rows represent successive values for $n$, and columns represent $k$ --- the first row contains only a single column where $k=0$, in the second row the columns are $k=0$ and $k=1$, and so on.
    \begin{center}
        \begin{tabular}{rccccccccc}
            $n=0$:&    &    &    &    &  1\\\noalign{\smallskip\smallskip}
            $n=1$:&    &    &    &  1 &    &  1\\\noalign{\smallskip\smallskip}
            $n=2$:&    &    &  1 &    &  2 &    &  1\\\noalign{\smallskip\smallskip}
            $n=3$:&    &  1 &    &  3 &    &  3 &    &  1\\\noalign{\smallskip\smallskip}
            $n=4$:&  1 &    &  4 &    &  6 &    &  4 &    &  1\\\noalign{\smallskip\smallskip}
        \end{tabular}
    \end{center}
\end{defn}

\begin{rmk}
    Notice that $\binom{n}{0} = \binom{n}{n} = 1$, since there is only one set with cardinality $0$ (the empty set), and the only subset with cardinality $n$ of a set $X$ itself with cardinality $n$ is $X$. Alternatively, since $\binom{n}{0} = 1$, by Proposition \ref{binomial-complement} it follows that $\binom{n}{0} = \binom{n}{n}$ and so $\binom{n}{n} = 1$.
\end{rmk}

\begin{rmk}
    Note that each entry in Pascal's Triangle is the sum of the two elements immediately above it.
\end{rmk}

\begin{prop}{Pascal's Identity}\label{pascals-identity}
    Let $n \in \N$ with $0 < k < n$. Then $\binom{n}{k} = \binom{n-1}{k-1} + \binom{n-1}{k}$.
\end{prop}

\begin{proof}
    Let $X$ be a set with $|X| = n$. By Definition \ref{binomial-coefficient}, $\binom{n}{k}$ is the number of subsets with $k$ elements of $X$. Fix some $x \in X$. Then there are $\binom{n-1}{k}$ subsets of $X$ that do not include $x$, and $\binom{n-1}{k-1}$ which do, so $\binom{n}{k} = \binom{n-1}{k-1} + \binom{n-1}{k}$.
\end{proof}

\begin{thm}{Binomial theorem}\label{binomial-theorem}\proofbreak
    Let $F$ be a field. Then for any $x, y \in F$ and $n \in \Z_{\geq 1}$, \[\left(x + y\right)^n = \sum_{k=0}^n x^ky^{n-k}\binom{n}{k}.\]
\end{thm}

\begin{proof}
    We will prove this by induction. Our base case is when $n = 1$, so $(x + y)^n = x + y$. Then \[\sum_k=0^1x^ky^{1-k}\binom{1}{k} = x^0y^1\binom{1}{0} + x^1y^0\binom{1}{1} = x + y,\] so the base case is true.
    Now assume that for some $x, y \in F$ and $n \geq 1$ we have \[\left(x + y\right)^n = \sum_{k=0}^n x^ky^{n-k}\binom{n}{k}.\] Then since $(x + y)^{n+1} = (x+y)(x+y)^n$, we have $(x = y)^{n+1} = x(x+y)^n + y(x+y)^n$.
    Therefore, \[\left(x + y\right)^{n+1} = x\left(\sum_{k=0}^n x^ky^{n-k}\binom{n}{k}\right) + y\left(\sum_{k=0}^n x^ky^{n-k}\binom{n}{k}\right),\] so it follows that \[\left(x + y\right)^{n+1} = \left(\sum_{k=0}^n x^{k+1}y^{n-k}\binom{n}{k}\right) + \left(\sum_{k=0}^n x^ky^{n-k+1}\binom{n}{k}\right).\]
    We can now re-index the left sum to make the form of the terms match up better: \[\left(x + y\right)^{n+1} = \left(\sum_{k=1}^{n+1} x^{k}y^{n-k+1}\binom{n}{k-1}\right) + \left(\sum_{k=0}^n x^ky^{n-k+1}\binom{n}{k}\right).\]
    Now we can factor and match up all the terms except for the last term of the left sum, and the first term of the right sum.
    \[\left(x + y\right)^{n+1} = y^{n+1}\binom{n}{0} + \left(\binom{n}{k-1} + \binom{n}{k}\right)\left(\sum_{k=0}^{n} x^{k}y^{n-k+1}\right) + x^{n+1}\binom{n}{n}.\]
    By Pascal's Identity \ref{pascals-identity}, we then have
    \[\left(x + y\right)^{n+1} = y^{n+1}\binom{n}{0} + \binom{n+1}{k}\left(\sum_{k=0}^{n} x^{k}y^{n-k+1}\right) + x^{n+1}\binom{n}{n}.\]
    We can then move the two extra terms into the summation:
    \[\left(x + y\right)^{n+1} = \sum_{k=0}^{n+1}x^{k}y^{n+1-k}\binom{n+1}{k}.\]
    Therefore, the induction step is complete, and so by the induction principle we are done.
\end{proof}

\begin{prop}
    Let $n, k \in \N$ with $0 \leq k \leq n$. Then \[\binom{n}{k} = \frac{n!}{k!(n-k)!}.\]
\end{prop}

\begin{proof}
    When selecting a specific subset with $k$ elements, there are $n$ choices for the first element, $n-1$ for the second, and so on until $n-k+1$ for the last. Therefore, there are $\frac{n!}{(n-k)!}$ different ways to select the subsets. Each subset can have its elements selected in $k!$ different orders, so there are $\frac{n!}{k!(n-k!)}$ such subsets.
\end{proof}

\begin{defn}
    Let $n, x_1, x_2, \ldots, x_k \in \N$ such that $x_1 + x_2 + \cdots + x_k = n$. The \emph{multinomial coefficient} \[\binom{n}{x_1, x_2, \ldots, x_k}\] is the number of distinct ways subsets $X_1, X_2, \ldots, X_k$ with cardinalities $x_1, x_2, \ldots, x_k$ respectively can be selected from a set with $n$ elements such that $X_i \intersection X_j = \emptyset$ for all $i \neq j$. Note that this necessarily implies $\bigdisjointunion_{i}X_i$ is the entire set elements are selected from.
\end{defn}

\begin{prop}
    \begin{align*}
        \binom{n}{x_1, x_2, \ldots, x_k} &= \frac{n!}{x_1!x_2!\cdots x_k!} \\ &= \binom{x_1 + x_2 + \cdots + x_k}{x_1}\binom{x_2 + \cdots + x_k}{x_2}\cdots\binom{x_k}{x_k}.
    \end{align*}
\end{prop}

\begin{prop}
    Let $n, k \in \N$. The number of $k$-tuples of positive integers that sum to $n$ is
    \[\binom{n + k - 1}{k - 1}.\]
\end{prop}

\begin{rmk}
    This also gives the numbers of ways to partition $n$ indistinguishable balls into $k$ distinct boxes.
\end{rmk}

\section{Recurrence Relations}

\begin{defn}
    A sequence is an ordered collection of elements.
\end{defn}

\begin{exmp}
    The well-known Fibonacci sequence is the infinite sequence \[0, 1, 1, 2, 3, 5, 8, 13, 21, 34, 55, \ldots\]
    which is characterized by $F_{n} = F_{n-2} + F_{n-1}$, $F_0 = 0$, and $F_1 = 1$.
\end{exmp}

\begin{defn}
    An arithmetic sequence $a$ is a sequence where $a_n = a_{n-1} + c$ for some constant $c$ and initial value $a_1$.
\end{defn}

\begin{exmp}
    If $a_1 = 1$, and $c = 2$, we get the odd natural numbers: \[1, 3, 5, 7, 9, 11, \ldots\]
\end{exmp}

\begin{prop}\label{finite-arithmetic-sum}
    The sum of a arithmetic sequence from $a_1$ to $a_n$ is \[\frac{n(a_1 + a_n)}{2}.\]
\end{prop}

\begin{defn}
    A geometric sequence $a$ is a sequence where $a_n = ra_{n-1}$ for some constant $r \neq 1$ and initial value $a_0$.
\end{defn}

\begin{prop}\label{finite-geometric-sum}
    The sum of a geometric sequence from $a_0$ to $a_n$ is \[a_0\frac{r^n - 1}{r - 1}.\]
\end{prop}

\begin{proof}
    Let $S = a_0(1 + r + r^2 + \cdots + r^n)$ be the sum of a finite geometric sequence from $a_0$ to $a_n$. Then $rs = a_0(r + r^2 + r^3 \cdots + r^{n+1})$, and so
    \begin{align*}
        rs - s &= a_0(r^{n+1} + (r^n - r^n) + \cdots + (r - r) - 1)\\
        s(r - 1) &= a_0(r^{n+1} - 1) \\
    s &= a_0\frac{r^{n+1} - 1}{r - 1}.
    \end{align*}
\end{proof}

\begin{prop}\label{infinite-geometric-sum}
    The sum of a geometric sequence with constant $0 \leq r < 1$ is \[\frac{a_0}{1-r}.\]
\end{prop}

\begin{proof}
    \begin{align*}
        \sum_{k=0}^{\infty}a_k &= \lim_{n \to\infty}\sum_{k=0}^{n}a_k = \lim_{n \to\infty}a_0\frac{r^n - 1}{r - 1} \\
        &= \frac{a_0}{r-1}\lim_{n \to\infty}\left(r^n - 1\right) = \frac{a_0}{r-1}\left(-1\right).
    \end{align*}
\end{proof}

\begin{defn}
    A \emph{recurrence relation} is an equation that recursively defines a sequence in terms of the preceding terms, and certain initial conditions.
\end{defn}

\begin{defn}
    A recurrence relation of \emph{order k} is an equation that recursively defines each term in terms of the previous $k$ terms. That is, if $u$ is the generated sequence, an order $k$ recurrence relation defines \[u_n = \varphi(n, u_{n-1}, u_{n-2}, \ldots, u_{n-k}) \,\textrm{for}\, n \geq k,\] where $\varphi$ is some function. Since $\varphi$ defines the sequence for $n \geq k$, $k$ initial values for $u_0$ through $u_{k-1}$ are needed.
\end{defn}

\begin{exmp}
    $a_n = 2a_{n-1}$ is a first-order (order 1) recurrence relation. If we define $a_0 = 1$, then it generates the sequence $a = 1, 2, 4, 8, \ldots$.
\end{exmp}

\begin{exmp}
    The Fibonacci sequence, defined by $a_0 = 0$, $a_1 = 1$, and $a_n = a_{n-1} + a_{n-2}$ for $n \geq 2$, is a second-order recurrence relation. This sequence is \[0, 1, 1, 2, 3, 5, 8, 13, 21, 34, 55, \ldots.\]
\end{exmp}

\begin{defn}
    A \emph{linear} recurrence relation is one whose recursive equation is a polynomial of degree $1$ --- that is, each term is a linear combination of the previous $k$ terms, plus a constant: \[u_n = s_1u_{n-1} + s_2u_{n-2} + \cdots + s_ku_{n-k} + b \,\textrm{for}\, n \geq k.\]
\end{defn}

\begin{defn}
    A \emph{homogeneous} linear recurrence relation is one whose constant is $0$: \[u_n = s_1u_{n-1} + s_2u_{n-2} + \cdots + s_ku_{n-k} \,\textrm{for}\, n \geq k.\] Linear recurrences with a non-zero constant are known as \emph{non-homogeneous}.
\end{defn}

\begin{prop}\label{linear-combinations-are-recurrence-solutions}
    A homogeneous linear recurrence is linear in its generated sequences. That is, given a linear recurrence relation $\varphi$, let $a$ and $b$ be sequences generated by $\varphi$ under some initial conditions. Then all linear combinations of $a$ and $b$ are also sequence generated by $\varphi$ for some initial conditions.
\end{prop}

\begin{proof}
    Let $u_n = s_1u_{n-1} + s_2u_{n-2} + \cdots + s_ku_{n-k}$ be a linear recurrence relation of order $k$, and let $a$ and $b$ be sequences generated under some initial conditions. Then since
    \begin{align*}
        s_1(a_{n-1} + b_{n_1}) + s_2(a_{n-1} + b_{n_1}) + \cdots + s_k(a_{n-1} + b_{n_1}) & = \\ (s_1a_{n-1} + s_2a_{n-2} + \cdots + s_ka_{n-k}) + (s_1b_{n-1} + s_2b_{n-2} + \cdots + s_kb_{n-k}) & = \\
        a_n + b_n, &
    \end{align*}
    and
    \begin{align*}
        s_1(ra_{n-1}) + s_2(ra_{n-1}) + \cdots + s_k(ra_{n-k}) & = \\
        r(s_1a_{n-1} + s_2a_{n-1} + \cdots + r(s_ka_{n-k})) & = \\
        ra_n, &
    \end{align*}
    the relation is linear in its generated sequences.
\end{proof}

\begin{defn}
    The \emph{solution} to a recurrence relation of order $k$ is an expression for the $n$th term of the generated sequences as a function of $n$, parameters of the generating equation, and the $k$ initial conditions, but not involving any other terms of the sequence.
\end{defn}

\begin{exmp}
    Let $u_n = 2u_{n-1}$ be a first-order homogeneous linear recurrence relation. Then $u_n = u_02^n$ is a solution to the recurrence relation.
\end{exmp}

\begin{thm}
    Let $u_n = su_{n-1} + t$ be a first-order non-homogeneous linear recurrence relation. If $s = 1$, then a solution is \[u_n = u_0 + nt \,\textrm{for}\, n \geq 1.\] Otherwise, a solution is \[u_n = s^n(u_0 + \frac{t}{s-1}) - \frac{t}{s-1} \,\textrm{for}\, n \geq 1.\]
\end{thm}

\begin{rmk}
    When $s \neq 1$, the solution to a first-order non-homogeneous linear recurrence relation is of the form $u_n = c_1s^n + c_2$, where $c_1$ and $c_2$ are fully determined by $s$, $t$, and $u_0$.
\end{rmk}

\begin{proof}\proofbreak
    If $s = 1$, and $u_{n-1} = u_0 + (n-1)t$, then $u_n = 1(u_0 + (n-1)t) + t = u_0 + nt$.

    If $s \neq 1$ and $u_{n-1} = s^{n-1}(u_0 + \frac{t}{s-1}) - \frac{t}{s-1}$, then
    \begin{align*}
        u_n = s(s^{n-1}(u_0 + \frac{t}{s-1}) - \frac{t}{s-1})) + t & = \\
        s^n(u_0 + \frac{t}{s-1}) - \frac{st}{s-1} + \frac{(s-1)t}{s-1} & = \\
        s^n(u_0 + \frac{t}{s-1}) - \frac{t}{s-1} &.
    \end{align*}
    Therefore, given an initial condition $u_0$, these are valid formulas for $u_n$ for all $n \geq 1$.
\end{proof}

\begin{thm}
    Let $u_n = s_1u_{n-1} + s_2u_{n-2}$ be a second-order homogeneous linear recurrence relation. Let $r_1, r_2$ be the (possible complex) roots of $x^2 - s_1x - s_2$.

    If $r_1 \neq r_2$, then $u_n = r_1^n$ and $u_n = r_2^n$ are linearly independent solutions.

    If $r_1 = r_2$, let $r = r_1 = r_2$. Then $u_n = r^n$ and $u_n = nr^n$ are both solutions, and furthermore are linearly independent if $r \neq 0$.
\end{thm}

\begin{proof}\proofbreak
    If $u_{n-1} = r^{n-1}$ and $u_{n-2} = r^{n-2}$ where $r$ is a root of $x^2 - s_1x - s_2$, then $u_n = s_1r^{n-1} + s_2r^{n-2}$. Since $r^2 - s_1r - s_2 = 0$, it follows that $u_n = r^{n-2}r^2 = r^n$.

    If $r = r_1 = r_2$, $u_{n-1} = nr^{n-1}$, and $u_{n-2} = nr^{n-2}$, then $u_n = s_1nr^{n-1} + s_2nr^{n-2}$. Since $r^2 - s_1r - s_2 = 0$, it follows that $u_n = nr^{n-2}r^2 = nr^n$.
\end{proof}

\begin{thm}
    Let $u_n = s_1u_{n-1} + s_2u_{n-2}$ be a second-order homogeneous linear recurrence relation. Let $r_1, r_2$ be the (possible complex) roots of $x^2 - s_1x - s_2$.

    If $r_1 \neq r_2$, all possible solutions can be expressed as $u_n = c_1r_1^n + c_2r_2^n$  for some $c_1, c_2$, and $c_1$ and $c_2$ are fully determined by $s_1$, $s_2$, $u_0$, and $u_1$.

    If $r = r_1 = r_2$, then all possible solutions can be expressed as $u_n = c_1r^n + c_2nr^n$ for some $c_1, c_2$, and $c_1$ and $c_2$ are fully determined by $s_1$, $s_2$, $u_0$, and $u_1$.
\end{thm}

\begin{proof}
    Since the sequence generated by a specific $s_1, s_2$ is fully determined by $u_0$ and $u_1$, if every possible $u_0$ and $u_1$ can be expressed by a specific form of a solution, than all possible solutions can be expressed in that form.

    If $r_1 \neq r_2$, then by Proposition \ref{linear-combinations-are-recurrence-solutions}, $u_n = c_1r_1^n + c_2r_2^n$ is a solution. Given $s_1, s_2$, and $u_0, u_1$, then $u_0 = c_1r_1^0 + c_2r_2^0 = c_1 + c_2$, and $u_1 = c_1r_1 + c_2r_2$. This can be expressed as
    \[\begin{amatrix}{2}
        1 &1 &u_0 \\
        r_1 &r_2 &u_1 \\
    \end{amatrix}.\] Since $r_1 \neq r_2$ this matrix is full-rank, and so has a single unique solution for $c_1, c_2$. Therefore, all possible solutions can be expressed in this form $r_1 \neq r_2$.

    If $r = r_1 = r_2$ and $r \neq 0$, then by Proposition \ref{linear-combinations-are-recurrence-solutions}, $u_n = c_1r^n + c_2nr^n$ is a solution. Given $s_1, s_2$, and $u_0, u_1$, then $u_0 = c_1r^0 + c_2(0)r^0 = c_1$, and $u_1 = c_1r + c_2(1)r$, so $c_2 = \frac{u_1 - u_0r}{r}$. Therefore, all possible solutions can be expressed in this form when $r_1 = r_2 \neq 0$.

    If $r = 0$, then $u_n = c_1r^n$, so $u_0 = c_1$, and $u_n = 0$ for all $n \geq 1$, so all possible solutions can be expressed in this form when $r = 0$.
\end{proof}

\section{Cantor's Theorem}

\begin{thm}\label{pigeonhole}
    Pigeonhole principle.

    Let $A, B$ be finite sets, and let $f: A \to B$. If $|A| > |B|$, then $f$ cannot be injective, and if $|A| < |B|$ then $f$ cannot be surjective.
\end{thm}

\begin{prop}
    Let $A, B$ be finite sets, and let $f: A \to B$. If $f$ is a bijection, then $|A| = |B|$.
\end{prop}

\begin{proof}
    If $|A| \neq |B|$, then either $|A| > |B|$ or $|A| < B$. Therefore, either $f$ is not injective or $f$ is not surjective by the pigeonhole principle \ref{pigeonhole}.
\end{proof}

\begin{prop}
    Let $A, B$ be finite sets, and let $a = |A|$ and $b = |B|$. Let $F$ be all functions $f: A \to B$, $I \subseteq F$ be all injective functions, and $S \subseteq F$ be all surjective functions. Then
    \begin{itemize}
        \item $|F| = b^a$,
        \item $|I| = (b)_a = \frac{b!}{(b-a)!}$,
        \item $|S| = \sum_{i=0}^b(-1)^i\binom{b}{i}(b-i)^a$.
    \end{itemize}
\end{prop}

\begin{thm}{Cantor's Theorem}\label{cantors-theorem}
    Let $A$ be a set, and let $f: A \to 2^A$. Then $f$ cannot be onto.
\end{thm}

\begin{proof}
    Let $B = \left\{x \in A \compbar x \notin f(x) \right\} \in 2^A$. For the sake of contradiction, assume that $f$ is onto, and so there must exist some $a \in A$ such that $f(a) = B$. We also know that either $a \in B$ or $a \notin B$.

    If $a \in B$, then by the construction of $B$ we have $a \notin f(a)$. However, $f(a) = B$, so this is a contradiction. Similarly, $a \notin B$ must imply $a \in f(a)$.
\end{proof}

\begin{defn}
    We say that a set $S$ is \emph{countable} if there exists a bijection between $S$ and some $A$ where either $A = \N$ or $A = \{0, 1, 2, 3, 4, \ldots, N\} \subseteq \N$ for some $N \in \N$. In the first case, we say that $S$ is \emph{countably infinite}, otherwise we say that $S$ is \emph{finite}. If no such bijection exists, then we say that $S$ is \emph{uncountable} or \emph{uncountably infinite}.
\end{defn}

\begin{thm}{Countability of $\Q$}\label{rationals-countable}\proofbreak
    The rational numbers $\Q$ are countable infinite.
\end{thm}

\begin{proof}
    We know that we can represent every $q \in Q$ as $q = a/b$ such that $a \in \Z$ and $b \in \N^{+}$, and $a$ and $b$ have no factors in common. Consider the following function $f: \Q^{\geq 0} \to \N$:
    \begin{itemize}
        \item $f(0/1) = 0$,
        \item $f(1/1) = 1$,
        \item $f(1/2) = 2$,
        \item $f(1/3) = 3$,
        \item $f(2/3) = 4$,
        \item $...$
        \item $f(a/b) = f((a-1)/b) + 1$ if $a > 1$, otherwise $f(a/b) = f((b-2)/(b-1)) + 1$.
    \end{itemize}

    Since $a-1 < a$ and $b-1 < b$, it follows we have a prodecure for computing $f(a/b)$ based solvely on $f(c/d)$ such that $d < b$ or $d = b$ and $a < c$, so this function is well-defined. Since we can invert the above function, it is a bijection from the non-negative rationals onto $\N$. Now consider $g: \Q \to \N$ where $g(q) = 2f(q)$ if $q \geq 0$ else $g(q) = 2(q) - 1$, which is a bijection between $Q$ and $\N$.
\end{proof}

\begin{lemma}\label{diagonal-construction}\proofbreak
    Let $T$ be the set of all binary strings. For any enumeration $s_1, \ldots, s_n, \ldots \in T$, there exists a binary string in $T$ that is not in $s_i$.
\end{lemma}

\begin{proof}
    Consider the infinite binary string $w$ defined by $w_i = 1 - (s_i)_i$ --- that is, it differs at the $i$th index from the same position of the $i$th string $s_i$. Therefore, for any $s_i$ we know that $w \neq s_i$ and so $w$ is not contained in the enumeration.
\end{proof}

\begin{thm}{Uncountability of $\R$, Cantor's Diagonal Argument}\label{reals-uncountable}\proofbreak
    The real numbers $\R$ are uncountable.
\end{thm}

\begin{proof}
    We can construct a bijection $b$ between $\R$ and the set of all infinite binary strings. Assume, for the sake of contradiction, that $\R$ is countable, so there exists a bijection $f$ between $\R$ and $\N$. Consider the enumerator of binary strings given by $s_{i} = b(f^{-1}(i))$, where $b(f^{-1}(i))$ is the binary string representing the real number that is in bijection with $i$. By Lemma \ref{diagonal-construction}, we know there exists a binary string $w$ such that $w \neq s_i$ for all $i$. However $f(b^{-1}(w)) = i$ for some $i \in \N$, and so $s_i = w$. This is a contradiction, and so no bijection between $\R$ and $\N$ exists. Furthermore, no \emph{injection} exists since $\Q \subsetneq \R$, and so no bijection between $\R$ and any subset of $\N$ exists. Therefore, the real numbers must be \emph{uncountably} infinite.
\end{proof}
