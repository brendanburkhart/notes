\chapter{Discrete Mathematics}
\label{ch:discrete}

\section{Boolean Algebra}

Boolean algebra is the algebra dealing exclusively with the values \textsc{true} and \textsc{false}.

The primary operations of Boolean algebra are \emph{negation} (also called \emph{not}) denoted by $\neg$, \emph{conjunction} (also called \emph{and}) denoted by $\land$, and \emph{disjunction} (also called \emph{or}) denoted by $\lor$.

Since Boolean algebra has only two elements, it is possible to enumerate all variable combinations for a function. This is often done in the form of a truth table --- a table listing the values of variables and the corresponding function value as rows. For example, Table \ref{primary-operations} gives a combined truth table for negation, conjunction, and disjunction. It also serves as the definition of these operations.

\begin{center}
    \captionof{table}{Truth table of primary operations}
    \label{primary-operations}
    \begin{tabularx}{\linewidth}{|X|X|X|X|X|}
        \hline
        \thead{$X$} & \thead{$Y$} & \thead{$\neg X$} & \thead{$X \land Y$} & \thead{$X \lor Y$} \\
        \hline
        \textsc{true} & \textsc{true} & \textsc{false} & \textsc{true} & \textsc{true} \\
        \hline
        \textsc{true}  & \textsc{false} & \textsc{false} & \textsc{false} & \textsc{true} \\
        \hline
        \textsc{false} & \textsc{true} & \textsc{true} & \textsc{false} & \textsc{true} \\
        \hline
        \textsc{false} & \textsc{false} & \textsc{true} & \textsc{false} & \textsc{false} \\
        \hline
    \end{tabularx}
\end{center}

\begin{defn}\label{implies}
    A statement $P$ \emph{implies} (also $\implies$) statement $Q$ if $Q$ is \textsc{true} any time that $P$ is \textsc{true}. When $P$ is \textsc{false}, $Q$ can be \textsc{true} or \textsc{false}. If $P$ is always \textsc{false}, then it implies all statements. $P \implies Q$ is equivalent to $Q \impliedby P$.
\end{defn}

\begin{defn}\label{iff}
    Statement $P$ \emph{if and only if} (also \emph{iff} and $\iff$) statement $Q$ if $P \implies Q$ and $Q \implies P$.
\end{defn}

\begin{thm}{Boolean properties}\label{boolean-algebraic-properties}\proofbreak
    Conjunction and disjunction are commutative:
    \begin{enumerate}
        \item $x \land y = y \land x$
        \item $x \lor y = y \lor x$
    \end{enumerate}

    Conjunction and disjunction are associative:
    \begin{enumerate}
        \item $(x \land y) \land z = x \land (y \land z)$
        \item $(x \lor y) \lor z = x \lor (y \lor z)$
    \end{enumerate}

    Conjunction and disjunction are distributive:
    \begin{enumerate}
        \item $x \land (y \lor z) = (x \land y) \lor (x \land z)$
        \item $x \lor (y \land z) = (x \lor y) \land (x \lor z)$
    \end{enumerate}

    \textup{\textsc{true}} is the identity element (see \ref{identity}) for conjunction and \textup{\textsc{false}} is the identity element for disjunction:
    \begin{enumerate}
        \item $x \land \textup{\textsc{true}} = x$
        \item $x \lor \textup{\textsc{false}} = x$
    \end{enumerate}
\end{thm}

\begin{thm}{Additional properties}\label{additional-boolean-properties}\proofbreak
    \begin{enumerate}
        \item $\neg(\neg x) = x$
        \item $x \land x = x$
        \item $x \lor x = x$
        \item $x \land \neg x = \textup{\textsc{false}}$
        \item $x \lor \neg x = \textup{\textsc{true}}$
    \end{enumerate}
\end{thm}

\begin{thm}{De Morgan's Laws}\label{demorgan-boolean}\proofbreak
    \begin{enumerate}
        \item $\neg(x \land y) = (\neg x) \lor (\neg y)$
        \item $\neg(x \lor y) = (\neg x) \land (\neg y)$
    \end{enumerate}
\end{thm}

\begin{rmk}
    All properties in Theorem \ref{boolean-algebraic-properties}, Theorem \ref{additional-boolean-properties}, and Theorem \ref{demorgan-boolean} can be proved by simply writing out the corresponding truth tables.
\end{rmk}

\section{Sets}

\begin{defn}\label{set}
    A set is an unordered group of distinct elements.
\end{defn}

\begin{exmp}
    $\{1, 2, 3\}$ is a set containing three elements: $1$, $2$, and $3$.
\end{exmp}

\begin{rmk}
    $\{1, 2, 3, 3\}$ is also set containing three elements, since the elements of a set are distinct.
\end{rmk}

\begin{defn}\label{empty-set}
    The empty set (denoted $\emptyset$) is the unique set having no elements.
\end{defn}

One of the most fundamental operations of sets is the ``element of'' operation, denoted by $\in$. $x \in X$ is \textsc{true} precisely when $x$ is an element of the set $X$. Note that sets can be elements of other sets. $x \notin X$ is used to denote ``not an element of''.

\begin{exmp}
    $\left\{\{1, 2\}, \{\}\right\}$ is a set containing two elements: the set $\{1, 2\}$, and the empty set.
\end{exmp}

Set comprehensions, or set builder notation, is a method of precisely defining a set. It can take various forms, such as enumerating all (or implying such) the elements of a set (e.g. $\{1, 2\}$ or $\{1, 2, \ldots, 5\}$). Or it can be used to build a set from another, such as $\left\{2n \compbar n \in \N\right\}$, which says make a set by taking every natural number and doubling it --- these are, of course, the even natural numbers. Set comprehensions can be made more complicated by including a predicate, for example $\left\{n \in \N \compbar n \neq n^2 \right\}$ --- all natural numbers which are not their own square.

Two sets are equal when they contain precisely the same elements. For example, if we let $A = \{1, 1, 5, 2\}$ and $B = \{2, 2, 2, 1, 5\}$, then $A = B$ since for every element $x$ in $A$, $x$ is also in $B$ and vice versa.

\begin{defn}\label{subset}
    A set $T$ is a \emph{subset} of a set $S$ when every element of $T$ is also an element of $S$. This relationship is denoted $T \subseteq S$. $S$ is also referred to as a \emph{super set} of $T$.
\end{defn}

\begin{exmp}
    $\{1, 2, 3\}$ is a subset of $\{1, 2, 3\}$.
\end{exmp}

\begin{rmk}
    If a set $A$ is a subset of set $B$, and $B$ is a subset of $A$, then the sets must be equal. Showing that $A \subseteq B$ and $B \subseteq A$ is a common way to prove that two sets are equal.
\end{rmk}

\begin{defn}\label{proper-subset}
    A set $T$ is a \emph{proper subset} of a set $S$ when every element of $T$ is also an element of $S$, but not vice versa --- that is, the sets are not equal. This relationship is denoted $T \subset S$.
\end{defn}

\begin{exmp}
    $\{1\}$ is a proper subset of $\{1, 2, 3\}$.
\end{exmp}

\begin{defn}\label{intersection}
    The intersection of sets $A$ and $B$ is the set $\left\{x \compbar x \in A \land x \in B \right\}$. It is denoted by $A \intersection B$.
\end{defn}

\begin{exmp}
    $\{1, 2, 3, 4\} \intersection \{3, 4, 5\} = \{3, 4\}$.
\end{exmp}

\begin{defn}
    The union of sets $A$ and $B$ is the set $\left\{x \compbar x \in A \lor x \in B\right\}$. It is denoted by $A \union B$.
\end{defn}

\begin{exmp}
    $\{1, 2, 3, 4\} \union \{3, 4, 5\} = \{1, 2, 3, 4, 5\}$.
\end{exmp}

\begin{rmk}
    If $A$ and $B$ are sets, then $(A \intersection B) \subseteq (A \union B)$. $A = B$ if and only if $(A \intersection B) = (A \union B)$.
\end{rmk}

\begin{defn}
    The complement of set $A$ with respect to some super set $U$ is
    \[\complementof{A} = \left\{x \in U \compbar x \notin A\right\}.\]
\end{defn}

\begin{exmp}
    Let $U = \{1, 2, 3, 4, 5\}$, and $A = \{1, 2\}$. Then $\complementof{A} = \{3, 4, 5\}$.
\end{exmp}

\begin{exmp}
    Let $U = \Z$, and $A$ the even numbers. Then $\complementof{A}$ is the set of the odd numbers.
\end{exmp}

\begin{rmk}
    $\complementof{A} \union A = U$. $\complementof{A} \intersection A = \emptyset$.
\end{rmk}

\begin{defn}\label{set-difference}
    The set difference of sets $A$ and  $B$, denoted $A \setminus B$ or $A - B$, is the set containing all elements of $A$ which are not elements of $B$. $A \setminus B = \left\{x \in A \compbar x \notin B\right\}$.
\end{defn}

\begin{defn}\label{symmetric-difference}
    The symmetric difference of sets $A$ and $B$, denoted $A \triangle B$, is defined to be the set $(A \setminus B) \union (B \setminus A)$.
\end{defn}

\begin{rmk}
    $(A \triangle B)' = (A \intersection B)$ when $U = A \union B$.
\end{rmk}

\begin{thm}\label{set-distributive-rule}\proofbreak
    \begin{enumerate}
        \item $A \land (B \lor C) = (A \land B) \lor (A \land C)$
        \item $A \lor (B \land C) = (A \lor B) \land (A \lor C)$
    \end{enumerate}
\end{thm}

\begin{thm}{De Morgan's Laws}\label{demorgan-set}\proofbreak
    \begin{enumerate}
        \item $\complementof{(A \land B)} = \complementof{A} \lor \complementof{B}$
        \item $\complementof{(A \lor B)} = \complementof{A} \land \complementof{B}$
    \end{enumerate}
\end{thm}

\section{Constructions}

The \emph{von Neumann} construction is one of several ways to construct the natural numbers. It defines $0 = \emptyset$, and defines a function (called the \emph{successor} function) $S(a) = a \union \{a\}$ for every set $a$. Along with the axiom of infinity from Zermelo-Fraenkel set theory, this defines the set of natural numbers.

Using this construction, each natural number is the set of all preceding natural numbers:
\begin{itemize}
    \item $0 = \{\}$
    \item $1 = 0 \union \{0\} = \{\{\}\}$
    \item $2 = 1 \union \{1\} = \{\{\}, \{\{\}\}\}$
    \item $3 = 2 \union \{2\} = \{\{\}, \{\{\}\}, \{\{\}, \{\{\}\}\}\}$
\end{itemize}

Notice that for natural numbers $n, m$, the number of elements of $n$ is $n$, and that $n \leq m \iff n \subseteq m$.

While sets are unordered groups of distinct elements, lists (also called $n$-tuples) are ordered groups of elements which are not necessarily distinct. An ordered pair $(a, b)$ is a list of a length two (a tuple), where $a$, and $b$ are elements of some set.

\begin{defn}\label{tuple}
    An ordered pair $(a, b)$ is a tuple of elements of some set.
\end{defn}

Ordered pairs (and $n$-tuples more generally) can be represented as sets themselves --- the pair $(a, b)$ can be represented as the set $\left\{a, \{a, b\}\right\}$.

\begin{defn}\label{cartesian-product}
    The Cartesian product of two sets $A$ and $B$ is denoted $A \times B$. It is equal to $\left\{\left(a, b\right) \compbar a \in A, b \in B \right\}$.
\end{defn}

\section{Binary Operations}

\begin{defn}
    A \emph{binary operation} is a mathematical operation of arity two, where both domains and the codomain are the same set.
\end{defn}

\begin{exmp}
    Addition on $\R$ is a binary operation, as are multiplication and subtraction.
\end{exmp}

\begin{exmp}
    Let $S$ be the set of all functions $f : \R \to \R$. For any $f, g \in S$, let $f \circ g$ be the function in $S$ defined by $(f \circ g)(x) = f(g(x))$. This is a binary operation on $S$, called function composition.
\end{exmp}

\begin{defn}
    A binary operation $\circ$ on a set $S$ is commutative if $x \circ y = y \circ x$ for all $x, y \in S$.
\end{defn}

\begin{defn}
    A binary operation $\circ$ on a set $S$ is associative if $x \circ (y \circ z) = (x \circ y) \circ z$ for all $x, y, z \in S$.
\end{defn}

\begin{defn}\label{identity}
    Let $\circ$ be a binary operation on a set $S$. Then an element $e$ of $S$ is called a left identity if $e \circ a = a$ for all $a$ in $S$, and a right identity if $a \circ e = a$ for all $a$ in $S$. If $e$ is both a left and a right identity, then it is simply an identity.
\end{defn}

\begin{thm}
    Let $\circ$ be a binary operation on a set $S$. If $\circ$ has both a left identity and a right identity, then those identities are the same.
\end{thm}

\begin{proof}
    Let $e_1$ be a left identity for $\circ$ and $e_2$ be a right identity. Then $e_1 = e_1 \circ e_2 = e_2$, so $e_1 = e_2$.
\end{proof}

\begin{cor}
    If a binary operation has both a left identity and a right identity, it has only a single unique identity.
\end{cor}

\begin{defn}
    Let $\circ$ be a binary operation on a set $S$, and $e$ be an identity element. An element $x$ of a set $S$ is \emph{invertible} if there exists some $x' \in S$ such that $x \circ x' = e$. $x'$ is the \emph{inverse} of $x$.
\end{defn}

\begin{thm}
    Let $\circ$ be a binary associative operation on a set $S$, and $u \in S$ be an invertible element. Then for all $x, y \in S$, $(x \circ u = y \circ u) \implies (x = y)$.
\end{thm}

\begin{proof}
    Let $u'$ be the inverse of the invertible element $u$. Then $(x \circ u) = (y \circ u) \implies (x \circ u) \circ u' = (y \circ u) \circ u'$. Since $\circ$ is associative, this implies that $x \circ (u \circ u') = y \circ (u \circ u')$. Since $u \circ u' = e$, we have $x \circ e = y \circ e$, and so $x = y$.
\end{proof}

\begin{cor}
    Let $\circ$ be a binary associative operation on a set $S$, and $u \in S$ be an invertible element. Then for all $x, y \in S$, $(u \circ x = u \circ y) \implies (x = y)$.
\end{cor}

\section{Relations}

\begin{defn}
    A \emph{relation} $R$ on sets $A$ and $B$ is a set of ordered pairs such that $R \subset A \times B$. For $a \in A, b \in B$, we say that $a$ is related to $b$, denoted by $a R b$, if $(a, b) \in R$.
\end{defn}

\begin{exmp}
    Let $A = \{1, 2, 3, 4\}$, and let $R = \{(1, 1), (2, 2), (3, 3), (4, 4)\}$. $R$ is a relation from $A$ to $A$ that states $a \,R\, a$ if and only if $a = a$. That is, $R$ is an example of a typical equality relation.
\end{exmp}

\begin{exmp}
    $<$, $\leq$, $=$, $>$, $\geq$ are all examples of relations.
\end{exmp}

\begin{defn}
    A relation $R$ on a set $X$ is, for all $x, y, z \in X$:
    \begin{itemize}
        \item \emph{Reflexive} if $x R x$.
        \item \emph{Irreflexive} if $\neg(x R x)$.
        \item \emph{Symmetric} if $x R y \iff y R x$.
        \item \emph{Antisymmetric} if $x R y$ and $y R x$ implies $x = y$.
        \item \emph{Transitive} if $x R y$ and $y R z$ implies $x R z$.
    \end{itemize}
\end{defn}

\begin{exmp}\proofbreak
    \begin{itemize}
        \item $\geq$ is a reflexive relation on $\R$.
        \item $<$ is an irreflexive relation on $\R$.
        \item $a \mid b$ is a symmetric relation on $\R$.
        \item $a \mid b$ is an antisymmetric relation on $\N$.
        \item $\leq$ is a transitive relation on $\R$.
    \end{itemize}
\end{exmp}

\begin{defn}\label{equivalence-relation}
    An \emph{equivalence relation} is a \emph{reflexive}, \emph{symmetric}, and \emph{transitive} relation.
\end{defn}

\begin{defn}
    Let $A, B$ be two sets, and let $R$ be a relation from $A$ to $B$. Then the \emph{inverse} of $R$, denoted by $R^{-1}$, is a relation from $B$ to $A$ given by \[R^{-1} \,= \left\{(b, a) \compbar (a, b) \in R \right\}.\]
\end{defn}

\begin{prop}
    Let $A, B$ be sets, and let $R$ be a relation from $A$ to $B$. Then $\left(R^{-1}\right)^{-1} = R$.
\end{prop}

\begin{proof}
    If $R = \emptyset$, then $R^{-1} = \emptyset$, so $\left(R^{-1}\right)^{-1} = R$.

    Let $(a, b) \in R$. Then by definition, we know $(b, a) \in R^{-1}$, so $(a, b) \in \left(R^{-1}\right)^{-1}$. Therefore, $R \subseteq \left(R^{-1}\right)^{-1}$. Now let $(a, b) \in \left(R^{-1}\right)^{-1}$, so it must be that $(b, a) \in R^{-1}$, so $(a, b) \in R$. Therefore, $\left(R^{-1}\right)^{-1} \subseteq R$, so it follows that $R = \left(R^{-1}\right)^{-1}$.
\end{proof}

\begin{defn}\label{partition}
    A \emph{partition} $P$ of a set $X$ is a set of subsets of $X$ such that:
    \begin{itemize}
        \item $\emptyset \notin P$ --- that is, none of those subsets are empty.
        \item $X = \bigcup_{A\in P}A$ --- that is, every element of $X$ is in a subset.
        \item $(\forall A, B \in P) A \neq B \implies A \intersection B = \emptyset$ --- that is, no element of $X$ is in more than one subset.
    \end{itemize}
    Every $p \in P$ is called a \emph{part} of $P$.
\end{defn}

\begin{exmp}
    Let $X = \{1, 2, 3, 4, 5\}$. Then $P = \left\{\{1, 2, 3\}, \{4, 5\}\right\}$ is a partition of $X$ as every element of $X$ is an element of an element of $P$, no element of $X$ is an element of more than one element of $P$, and no element of $P$ is empty.
\end{exmp}

\begin{exmp}
    The empty set has a unique partition --- the empty set.
\end{exmp}

\begin{defn}
    For a non-empty $X$, $P = \{X\}$ is the \emph{trivial} partition of $X$.
\end{defn}

\begin{defn}\label{equivalence-class}
    Let $S$ be a set and $R$ an equivalence relation on $S$. Then the \emph{equivalence class} of an element $a$ in $S$, denoted by $\left[a\right]$, is the set $\left\{x \in S \compbar x R a\right\}$.
\end{defn}

\begin{lemma}\label{equiv-class-non-empty}
    Let $S$ be a set, $R$ be an equivalence relation on $S$, and $a \in S$. Then $a \in [a]$, and so $[a] \neq \emptyset$.
\end{lemma}

\begin{proof}
    By reflexivity, $a R a$, so it follows that $a \in [a]$.
\end{proof}

\begin{lemma}\label{equiv-class-equal}
    Let $S$ be a set, $R$ be an equivalence relation on $S$, and $a, b \in S$. Then $a R b$ if and only if $[a] = [b]$.
\end{lemma}

\begin{proof}\proofbreak
    ($\implies$) Suppose $a R b$, and let $x \in [a]$. Then $x R a$, and so by transitivity $x R b$, so $x \in [b]$. Therefore, $[a] \subseteq [b]$. If $x \in [b]$, then $x R b$. By symmetry, $b R a$, and so by transitivity, $x R a$, implying that $x \in [a]$. Therefore, $[b] \subseteq [a]$, and so $[a] = [b]$.

    ($\impliedby$) Assume that $[a] = [b]$. Let $x \in [a]$. Then $x \in [b]$, $x R a$, and $x R b$. It follows that by symmetry $a R x$, and by transitivity $a R b$.
\end{proof}

\begin{thm}Equivalence classes form a partition\label{equiv-classes-form-partition}\proofbreak
    Let $S$ be a set, and $R$ an equivalence relation on $S$. If $X$ is the set of all equivalence classes of elements in $S$, then $X$ is a partition of $S$.
\end{thm}

\begin{proof}\proofbreak
    \begin{itemize}
        \item By \ref{equiv-class-non-empty}, there is no $p \in X$ such that $p = \emptyset$, so $\emptyset \notin X$.
        \item Since $a \in S \implies a \in \left[a\right]$, we know that $S \subseteq \bigcup_{A\in X}A$. Since $a \in \bigcup_{A\in X}A$ implies that $a \in A$ for some $A \in X$. Therefore, $a \in S$, so we know that $\bigcup_{A\in X}A \subseteq S$. Therefore, $S = \bigcup_{A\in X}A$.
        \item Let $a, b \in S$ such that $\left[a\right] \intersection \left[b\right] \neq \emptyset$. Then there exists some $x \in (\left[a\right] \intersection \left[b\right])$, so $x R a$ and $x R b$. By symmetry $a R x$, and then by transitivity $a R b$. Then by \ref{equiv-class-equal} we have $\left[a\right] = \left[b\right]$. Therefore, $(\forall A, B \in P) A \neq B \implies A \intersection B = \emptyset$.
    \end{itemize}
    By Definition \ref{partition}, $X$ is a partition of $S$.
\end{proof}

\section{Binomial Coefficients}

\begin{defn}\label{binomial-coefficient}
Let $n, k \in \N$. The \emph{binomial coefficient} $\binom{n}{k}$ is the number of subsets with cardinality $k$ of a set with cardinality $n$.
\end{defn}

\begin{exmp}\proofbreak
    \begin{multicols}{3}
        \begin{itemize}
            \item $\binom{0}{0} = 1$
            \item $\binom{1}{0} = 1$
            \item $\binom{1}{1} = 1$
            \item $\binom{2}{1} = 2$
        \end{itemize}

        \columnbreak

        \begin{itemize}
            \item $\binom{3}{2} = 3$
            \item $\binom{4}{2} = 6$
            \item $\binom{5}{0} = 1$
            \item $\binom{5}{1} = 5$
        \end{itemize}

        \columnbreak

        \begin{itemize}
            \item $\binom{5}{2} = 10$
            \item $\binom{5}{3} = 10$
            \item $\binom{5}{4} = 5$
            \item $\binom{5}{5} = 1$
        \end{itemize}
    \end{multicols}
\end{exmp}

\begin{prop}\label{binomial-complement}
    Let $n, k \in \N$ with $0 \leq k \leq n$. Then \[\binom{n}{k} = \binom{n}{n-k}.\]
\end{prop}

\begin{proof}
    Any selection of $k$ elements from a set with $n$ elements is equivalent to the selection of the other $n-k$ elements.
\end{proof}

\begin{exmp}
    $\binom{5}{2} = \binom{5}{3}$, and $\binom{3}{1} = \binom{3}{2}$.
\end{exmp}

\begin{defn}\label{pascals-triangle}
    \emph{Pascal's Triangle} is a particular way of writing out what is essentially a table of binomial coefficients that highlights several properties. Rows represent successive values for $n$, and columns represent $k$ --- the first row contains only a single column where $k=0$, in the second row the columns are $k=0$ and $k=1$, and so on.
    \begin{center}
        \begin{tabular}{rccccccccc}
            $n=0$:&    &    &    &    &  1\\\noalign{\smallskip\smallskip}
            $n=1$:&    &    &    &  1 &    &  1\\\noalign{\smallskip\smallskip}
            $n=2$:&    &    &  1 &    &  2 &    &  1\\\noalign{\smallskip\smallskip}
            $n=3$:&    &  1 &    &  3 &    &  3 &    &  1\\\noalign{\smallskip\smallskip}
            $n=4$:&  1 &    &  4 &    &  6 &    &  4 &    &  1\\\noalign{\smallskip\smallskip}
        \end{tabular}
    \end{center}
\end{defn}

\begin{rmk}
    Notice that $\binom{n}{0} = \binom{n}{n} = 1$, since there is only one set with cardinality $0$ (the empty set), and the only subset with cardinality $n$ of a set $X$ itself with cardinality $n$ is $X$. Alternatively, since $\binom{n}{0} = 1$, by Proposition \ref{binomial-complement} it follows that $\binom{n}{0} = \binom{n}{n}$ and so $\binom{n}{n} = 1$.
\end{rmk}

\begin{rmk}
    Note that each entry in Pascal's Triangle is the sum of the two elements immediately above it.
\end{rmk}

\begin{prop}{Pascal's Identity}\label{pascals-identity}
    Let $n \in \N$ with $0 < k < n$. Then $\binom{n}{k} = \binom{n-1}{k-1} + \binom{n-1}{k}$.
\end{prop}

\begin{proof}
    Let $X$ be a set with $|X| = n$. By Definition \ref{binomial-coefficient}, $\binom{n}{k}$ is the number of subsets with $k$ elements of $X$. Fix some $x \in X$. Then there are $\binom{n-1}{k}$ subsets of $X$ that do not include $x$, and $\binom{n-1}{k-1}$ which do, so $\binom{n}{k} = \binom{n-1}{k-1} + \binom{n-1}{k}$.
\end{proof}

\begin{thm}{Binomial theorem}\label{binomial-theorem}\proofbreak
    Let $F$ be a field. Then for any $x, y \in F$ and $n \in \Z_{\geq 1}$, \[\left(x + y\right)^n = \sum_{k=0}^n x^ky^{n-k}\binom{n}{k}.\]
\end{thm}

\begin{proof}
    We will prove this by induction. Our base case is when $n = 1$, so $(x + y)^n = x + y$. Then \[\sum_k=0^1x^ky^{1-k}\binom{1}{k} = x^0y^1\binom{1}{0} + x^1y^0\binom{1}{1} = x + y,\] so the base case is true.
    Now assume that for some $x, y \in F$ and $n \geq 1$ we have \[\left(x + y\right)^n = \sum_{k=0}^n x^ky^{n-k}\binom{n}{k}.\] Then since $(x + y)^{n+1} = (x+y)(x+y)^n$, we have $(x = y)^{n+1} = x(x+y)^n + y(x+y)^n$.
    Therefore, \[\left(x + y\right)^{n+1} = x\left(\sum_{k=0}^n x^ky^{n-k}\binom{n}{k}\right) + y\left(\sum_{k=0}^n x^ky^{n-k}\binom{n}{k}\right),\] so it follows that \[\left(x + y\right)^{n+1} = \left(\sum_{k=0}^n x^{k+1}y^{n-k}\binom{n}{k}\right) + \left(\sum_{k=0}^n x^ky^{n-k+1}\binom{n}{k}\right).\]
    We can now re-index the left sum to make the form of the terms match up better: \[\left(x + y\right)^{n+1} = \left(\sum_{k=1}^{n+1} x^{k}y^{n-k+1}\binom{n}{k-1}\right) + \left(\sum_{k=0}^n x^ky^{n-k+1}\binom{n}{k}\right).\]
    Now we can factor and match up all the terms except for the last term of the left sum, and the first term of the right sum.
    \[\left(x + y\right)^{n+1} = y^{n+1}\binom{n}{0} + \left(\binom{n}{k-1} + \binom{n}{k}\right)\left(\sum_{k=0}^{n} x^{k}y^{n-k+1}\right) + x^{n+1}\binom{n}{n}.\]
    By Pascal's Identity \ref{pascals-identity}, we then have
    \[\left(x + y\right)^{n+1} = y^{n+1}\binom{n}{0} + \binom{n+1}{k}\left(\sum_{k=0}^{n} x^{k}y^{n-k+1}\right) + x^{n+1}\binom{n}{n}.\]
    We can then move the two extra terms into the summation:
    \[\left(x + y\right)^{n+1} = \sum_{k=0}^{n+1}x^{k}y^{n+1-k}\binom{n+1}{k}.\]
    Therefore, the induction step is complete, and so by the induction principle we are done.
\end{proof}

\begin{prop}
    Let $n, k \in \N$ with $0 \leq k \leq n$. Then \[\binom{n}{k} = \frac{n!}{k!(n-k)!}.\]
\end{prop}

\begin{proof}
    When selecting a specific subset with $k$ elements, there are $n$ choices for the first element, $n-1$ for the second, and so on until $n-k+1$ for the last. Therefore, there are $\frac{n!}{(n-k)!}$ different ways to select the subsets. Each subset can have its elements selected in $k!$ different orders, so there are $\frac{n!}{k!(n-k!)}$ such subsets.
\end{proof}

\begin{defn}
    Let $n, x_1, x_2, \ldots, x_k \in \N$ such that $x_1 + x_2 + \cdots + x_k = n$. The \emph{multinomial coefficient} \[\binom{n}{x_1, x_2, \ldots, x_k}\] is the number of distinct ways subsets $X_1, X_2, \ldots, X_k$ with cardinalities $x_1, x_2, \ldots, x_k$ respectively can be selected from a set with $n$ elements such that $X_i \intersection X_j = \emptyset$ for all $i \neq j$. Note that this necessarily implies $\bigdisjointunion_{i}X_i$ is the entire set elements are selected from.
\end{defn}

\begin{prop}
    \begin{align*}
        \binom{n}{x_1, x_2, \ldots, x_k} &= \frac{n!}{x_1!x_2!\cdots x_k!} \\ &= \binom{x_1 + x_2 + \cdots + x_k}{x_1}\binom{x_2 + \cdots + x_k}{x_2}\cdots\binom{x_k}{x_k}.
    \end{align*}
\end{prop}

\begin{prop}
    Let $n, k \in \N$. The number of $k$-tuples of positive integers that sum to $n$ is
    \[\binom{n + k - 1}{k - 1}.\]
\end{prop}

\begin{rmk}
    This also gives the numbers of ways to partition $n$ indistinguishable balls into $k$ distinct boxes.
\end{rmk}

\section{Recurrence Relations}

\begin{defn}
    A sequence is an ordered collection of elements.
\end{defn}

\begin{exmp}
    The well-known Fibonacci sequence is the infinite sequence \[0, 1, 1, 2, 3, 5, 8, 13, 21, 34, 55, \ldots\]
    which is characterized by $F_{n} = F_{n-2} + F_{n-1}$, $F_0 = 0$, and $F_1 = 1$.
\end{exmp}

\begin{defn}
    An arithmetic sequence $a$ is a sequence where $a_n = a_{n-1} + c$ for some constant $c$ and initial value $a_1$.
\end{defn}

\begin{exmp}
    If $a_1 = 1$, and $c = 2$, we get the odd natural numbers: \[1, 3, 5, 7, 9, 11, \ldots\]
\end{exmp}

\begin{prop}\label{finite-arithmetic-sum}
    The sum of a arithmetic sequence from $a_1$ to $a_n$ is \[\frac{n(a_1 + a_n)}{2}.\]
\end{prop}

\begin{defn}
    A geometric sequence $a$ is a sequence where $a_n = ra_{n-1}$ for some constant $r \neq 1$ and initial value $a_0$.
\end{defn}

\begin{prop}\label{finite-geometric-sum}
    The sum of a geometric sequence from $a_0$ to $a_n$ is \[a_0\frac{r^n - 1}{r - 1}.\]
\end{prop}

\begin{proof}
    Let $S = a_0(1 + r + r^2 + \cdots + r^n)$ be the sum of a finite geometric sequence from $a_0$ to $a_n$. Then $rs = a_0(r + r^2 + r^3 \cdots + r^{n+1})$, and so
    \begin{align*}
        rs - s &= a_0(r^{n+1} + (r^n - r^n) + \cdots + (r - r) - 1)\\
        s(r - 1) &= a_0(r^{n+1} - 1) \\
    s &= a_0\frac{r^{n+1} - 1}{r - 1}.
    \end{align*}
\end{proof}

\begin{prop}\label{infinite-geometric-sum}
    The sum of a geometric sequence with constant $0 \leq r < 1$ is \[\frac{a_0}{1-r}.\]
\end{prop}

\begin{proof}
    \begin{align*}
        \sum_{k=0}^{\infty}a_k &= \lim_{n \to\infty}\sum_{k=0}^{n}a_k = \lim_{n \to\infty}a_0\frac{r^n - 1}{r - 1} \\
        &= \frac{a_0}{r-1}\lim_{n \to\infty}\left(r^n - 1\right) = \frac{a_0}{r-1}\left(-1\right).
    \end{align*}
\end{proof}

\begin{defn}
    A \emph{recurrence relation} is an equation that recursively defines a sequence in terms of the preceding terms, and certain initial conditions.
\end{defn}

\begin{defn}
    A recurrence relation of \emph{order k} is an equation that recursively defines each term in terms of the previous $k$ terms. That is, if $u$ is the generated sequence, an order $k$ recurrence relation defines \[u_n = \varphi(n, u_{n-1}, u_{n-2}, \ldots, u_{n-k}) \,\textrm{for}\, n \geq k,\] where $\varphi$ is some function. Since $\varphi$ defines the sequence for $n \geq k$, $k$ initial values for $u_0$ through $u_{k-1}$ are needed.
\end{defn}

\begin{exmp}
    $a_n = 2a_{n-1}$ is a first-order (order 1) recurrence relation. If we define $a_0 = 1$, then it generates the sequence $a = 1, 2, 4, 8, \ldots$.
\end{exmp}

\begin{exmp}
    The Fibonacci sequence, defined by $a_0 = 0$, $a_1 = 1$, and $a_n = a_{n-1} + a_{n-2}$ for $n \geq 2$, is a second-order recurrence relation. This sequence is \[0, 1, 1, 2, 3, 5, 8, 13, 21, 34, 55, \ldots.\]
\end{exmp}

\begin{defn}
    A \emph{linear} recurrence relation is one whose recursive equation is a polynomial of degree $1$ --- that is, each term is a linear combination of the previous $k$ terms, plus a constant: \[u_n = s_1u_{n-1} + s_2u_{n-2} + \cdots + s_ku_{n-k} + b \,\textrm{for}\, n \geq k.\]
\end{defn}

\begin{defn}
    A \emph{homogeneous} linear recurrence relation is one whose constant is $0$: \[u_n = s_1u_{n-1} + s_2u_{n-2} + \cdots + s_ku_{n-k} \,\textrm{for}\, n \geq k.\] Linear recurrences with a non-zero constant are known as \emph{non-homogeneous}.
\end{defn}

\begin{prop}\label{linear-combinations-are-recurrence-solutions}
    A homogeneous linear recurrence is linear in its generated sequences. That is, given a linear recurrence relation $\varphi$, let $a$ and $b$ be sequences generated by $\varphi$ under some initial conditions. Then all linear combinations of $a$ and $b$ are also sequence generated by $\varphi$ for some initial conditions.
\end{prop}

\begin{proof}
    Let $u_n = s_1u_{n-1} + s_2u_{n-2} + \cdots + s_ku_{n-k}$ be a linear recurrence relation of order $k$, and let $a$ and $b$ be sequences generated under some initial conditions. Then since
    \begin{align*}
        s_1(a_{n-1} + b_{n_1}) + s_2(a_{n-1} + b_{n_1}) + \cdots + s_k(a_{n-1} + b_{n_1}) & = \\ (s_1a_{n-1} + s_2a_{n-2} + \cdots + s_ka_{n-k}) + (s_1b_{n-1} + s_2b_{n-2} + \cdots + s_kb_{n-k}) & = \\
        a_n + b_n, &
    \end{align*}
    and
    \begin{align*}
        s_1(ra_{n-1}) + s_2(ra_{n-1}) + \cdots + s_k(ra_{n-k}) & = \\
        r(s_1a_{n-1} + s_2a_{n-1} + \cdots + r(s_ka_{n-k})) & = \\
        ra_n, &
    \end{align*}
    the relation is linear in its generated sequences.
\end{proof}

\begin{defn}
    The \emph{solution} to a recurrence relation of order $k$ is an expression for the $n$th term of the generated sequences as a function of $n$, parameters of the generating equation, and the $k$ initial conditions, but not involving any other terms of the sequence.
\end{defn}

\begin{exmp}
    Let $u_n = 2u_{n-1}$ be a first-order homogeneous linear recurrence relation. Then $u_n = u_02^n$ is a solution to the recurrence relation.
\end{exmp}

\begin{thm}
    Let $u_n = su_{n-1} + t$ be a first-order non-homogeneous linear recurrence relation. If $s = 1$, then a solution is \[u_n = u_0 + nt \,\textrm{for}\, n \geq 1.\] Otherwise, a solution is \[u_n = s^n(u_0 + \frac{t}{s-1}) - \frac{t}{s-1} \,\textrm{for}\, n \geq 1.\]
\end{thm}

\begin{rmk}
    When $s \neq 1$, the solution to a first-order non-homogeneous linear recurrence relation is of the form $u_n = c_1s^n + c_2$, where $c_1$ and $c_2$ are fully determined by $s$, $t$, and $u_0$.
\end{rmk}

\begin{proof}\proofbreak
    If $s = 1$, and $u_{n-1} = u_0 + (n-1)t$, then $u_n = 1(u_0 + (n-1)t) + t = u_0 + nt$.

    If $s \neq 1$ and $u_{n-1} = s^{n-1}(u_0 + \frac{t}{s-1}) - \frac{t}{s-1}$, then
    \begin{align*}
        u_n = s(s^{n-1}(u_0 + \frac{t}{s-1}) - \frac{t}{s-1})) + t & = \\
        s^n(u_0 + \frac{t}{s-1}) - \frac{st}{s-1} + \frac{(s-1)t}{s-1} & = \\
        s^n(u_0 + \frac{t}{s-1}) - \frac{t}{s-1} &.
    \end{align*}
    Therefore, given an initial condition $u_0$, these are valid formulas for $u_n$ for all $n \geq 1$.
\end{proof}

\begin{thm}
    Let $u_n = s_1u_{n-1} + s_2u_{n-2}$ be a second-order homogeneous linear recurrence relation. Let $r_1, r_2$ be the (possible complex) roots of $x^2 - s_1x - s_2$.

    If $r_1 \neq r_2$, then $u_n = r_1^n$ and $u_n = r_2^n$ are linearly independent solutions.

    If $r_1 = r_2$, let $r = r_1 = r_2$. Then $u_n = r^n$ and $u_n = nr^n$ are both solutions, and furthermore are linearly independent if $r \neq 0$.
\end{thm}

\begin{proof}\proofbreak
    If $u_{n-1} = r^{n-1}$ and $u_{n-2} = r^{n-2}$ where $r$ is a root of $x^2 - s_1x - s_2$, then $u_n = s_1r^{n-1} + s_2r^{n-2}$. Since $r^2 - s_1r - s_2 = 0$, it follows that $u_n = r^{n-2}r^2 = r^n$.

    If $r = r_1 = r_2$, $u_{n-1} = nr^{n-1}$, and $u_{n-2} = nr^{n-2}$, then $u_n = s_1nr^{n-1} + s_2nr^{n-2}$. Since $r^2 - s_1r - s_2 = 0$, it follows that $u_n = nr^{n-2}r^2 = nr^n$.
\end{proof}

\begin{thm}
    Let $u_n = s_1u_{n-1} + s_2u_{n-2}$ be a second-order homogeneous linear recurrence relation. Let $r_1, r_2$ be the (possible complex) roots of $x^2 - s_1x - s_2$.

    If $r_1 \neq r_2$, all possible solutions can be expressed as $u_n = c_1r_1^n + c_2r_2^n$  for some $c_1, c_2$, and $c_1$ and $c_2$ are fully determined by $s_1$, $s_2$, $u_0$, and $u_1$.

    If $r = r_1 = r_2$, then all possible solutions can be expressed as $u_n = c_1r^n + c_2nr^n$ for some $c_1, c_2$, and $c_1$ and $c_2$ are fully determined by $s_1$, $s_2$, $u_0$, and $u_1$.
\end{thm}

\begin{proof}
    Since the sequence generated by a specific $s_1, s_2$ is fully determined by $u_0$ and $u_1$, if every possible $u_0$ and $u_1$ can be expressed by a specific form of a solution, than all possible solutions can be expressed in that form.

    If $r_1 \neq r_2$, then by Proposition \ref{linear-combinations-are-recurrence-solutions}, $u_n = c_1r_1^n + c_2r_2^n$ is a solution. Given $s_1, s_2$, and $u_0, u_1$, then $u_0 = c_1r_1^0 + c_2r_2^0 = c_1 + c_2$, and $u_1 = c_1r_1 + c_2r_2$. This can be expressed as
    \[\begin{amatrix}{2}
        1 &1 &u_0 \\
        r_1 &r_2 &u_1 \\
    \end{amatrix}.\] Since $r_1 \neq r_2$ this matrix is full-rank, and so has a single unique solution for $c_1, c_2$. Therefore, all possible solutions can be expressed in this form $r_1 \neq r_2$.

    If $r = r_1 = r_2$ and $r \neq 0$, then by Proposition \ref{linear-combinations-are-recurrence-solutions}, $u_n = c_1r^n + c_2nr^n$ is a solution. Given $s_1, s_2$, and $u_0, u_1$, then $u_0 = c_1r^0 + c_2(0)r^0 = c_1$, and $u_1 = c_1r + c_2(1)r$, so $c_2 = \frac{u_1 - u_0r}{r}$. Therefore, all possible solutions can be expressed in this form when $r_1 = r_2 \neq 0$.

    If $r = 0$, then $u_n = c_1r^n$, so $u_0 = c_1$, and $u_n = 0$ for all $n \geq 1$, so all possible solutions can be expressed in this form when $r = 0$.
\end{proof}

\section{Cantor's Theorem}

\begin{thm}\label{pigeonhole}
    Pigeonhole principle.

    Let $A, B$ be finite sets, and let $f: A \to B$. If $|A| > |B|$, then $f$ cannot be injective, and if $|A| < |B|$ then $f$ cannot be surjective.
\end{thm}

\begin{prop}
    Let $A, B$ be finite sets, and let $f: A \to B$. If $f$ is a bijection, then $|A| = |B|$.
\end{prop}

\begin{proof}
    If $|A| \neq |B|$, then either $|A| > |B|$ or $|A| < B$. Therefore, either $f$ is not injective or $f$ is not surjective by the pigeonhole principle \ref{pigeonhole}.
\end{proof}

\begin{prop}
    Let $A, B$ be finite sets, and let $a = |A|$ and $b = |B|$. Let $F$ be all functions $f: A \to B$, $I \subseteq F$ be all injective functions, and $S \subseteq F$ be all surjective functions. Then
    \begin{itemize}
        \item $|F| = b^a$,
        \item $|I| = (b)_a = \frac{b!}{(b-a)!}$,
        \item $|S| = \sum_{i=0}^b(-1)^i\binom{b}{i}(b-i)^a$.
    \end{itemize}
\end{prop}

\begin{thm}{Cantor's Theorem}\label{cantors-theorem}
    Let $A$ be a set, and let $f: A \to 2^A$. Then $f$ cannot be onto.
\end{thm}

\begin{proof}
    Let $B = \left\{x \in A \compbar x \notin f(x) \right\} \in 2^A$. For the sake of contradiction, assume that $f$ is onto, and so there must exist some $a \in A$ such that $f(a) = B$. We also know that either $a \in B$ or $a \notin B$.

    If $a \in B$, then by the construction of $B$ we have $a \notin f(a)$. However, $f(a) = B$, so this is a contradiction. Similarly, $a \notin B$ must imply $a \in f(a)$.
\end{proof}

\begin{defn}
    We say that a set $S$ is \emph{countable} if there exists a bijection between $S$ and some $A$ where either $A = \N$ or $A = \{0, 1, 2, 3, 4, \ldots, N\} \subseteq \N$ for some $N \in \N$. In the first case, we say that $S$ is \emph{countably infinite}, otherwise we say that $S$ is \emph{finite}. If no such bijection exists, then we say that $S$ is \emph{uncountable} or \emph{uncountably infinite}.
\end{defn}

\begin{thm}{Countability of $\Q$}\label{rationals-countable}\proofbreak
    The rational numbers $\Q$ are countable infinite.
\end{thm}

\begin{proof}
    We know that we can represent every $q \in Q$ as $q = a/b$ such that $a \in \Z$ and $b \in \N^{+}$, and $a$ and $b$ have no factors in common. Consider the following function $f: \Q^{\geq 0} \to \N$:
    \begin{itemize}
        \item $f(0/1) = 0$,
        \item $f(1/1) = 1$,
        \item $f(1/2) = 2$,
        \item $f(1/3) = 3$,
        \item $f(2/3) = 4$,
        \item $...$
        \item $f(a/b) = f((a-1)/b) + 1$ if $a > 1$, otherwise $f(a/b) = f((b-2)/(b-1)) + 1$.
    \end{itemize}

    Since $a-1 < a$ and $b-1 < b$, it follows we have a prodecure for computing $f(a/b)$ based solvely on $f(c/d)$ such that $d < b$ or $d = b$ and $a < c$, so this function is well-defined. Since we can invert the above function, it is a bijection from the non-negative rationals onto $\N$. Now consider $g: \Q \to \N$ where $g(q) = 2f(q)$ if $q \geq 0$ else $g(q) = 2(q) - 1$, which is a bijection between $Q$ and $\N$.
\end{proof}

\begin{lemma}\label{diagonal-construction}\proofbreak
    Let $T$ be the set of all binary strings. For any enumeration $s_1, \ldots, s_n, \ldots \in T$, there exists a binary string in $T$ that is not in $s_i$.
\end{lemma}

\begin{proof}
    Consider the infinite binary string $w$ defined by $w_i = 1 - (s_i)_i$ --- that is, it differs at the $i$th index from the same position of the $i$th string $s_i$. Therefore, for any $s_i$ we know that $w \neq s_i$ and so $w$ is not contained in the enumeration.
\end{proof}

\begin{thm}{Uncountability of $\R$, Cantor's Diagonal Argument}\label{reals-uncountable}\proofbreak
    The real numbers $\R$ are uncountable.
\end{thm}

\begin{proof}
    We can construct a bijection $b$ between $\R$ and the set of all infinite binary strings. Assume, for the sake of contradiction, that $\R$ is countable, so there exists a bijection $f$ between $\R$ and $\N$. Consider the enumerator of binary strings given by $s_{i} = b(f^{-1}(i))$, where $b(f^{-1}(i))$ is the binary string representing the real number that is in bijection with $i$. By Lemma \ref{diagonal-construction}, we know there exists a binary string $w$ such that $w \neq s_i$ for all $i$. However $f(b^{-1}(w)) = i$ for some $i \in \N$, and so $s_i = w$. This is a contradiction, and so no bijection between $\R$ and $\N$ exists. Furthermore, no \emph{injection} exists since $\Q \subsetneq \R$, and so no bijection between $\R$ and any subset of $\N$ exists. Therefore, the real numbers must be \emph{uncountably} infinite.
\end{proof}
