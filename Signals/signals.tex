\documentclass[12pt]{article}

\usepackage{amsmath}
\usepackage{amssymb}
\usepackage{amsthm}
\usepackage{centernot}
\usepackage{fullpage}
\usepackage{makecell}
\usepackage{tabularx}
\usepackage[hypcap=false]{caption}
\usepackage{tikz, tkz-euclide}
\usetikzlibrary{decorations.pathreplacing,arrows}
\usetikzlibrary{quotes,angles,calc,intersections}

\usepackage{titling}
\usepackage{pdfpages}
\usepackage{enumitem}
\usepackage{multicol}
\usepackage{bm}

\usepackage{linear}
\usepackage{common}

\begin{document}

\title{Signals and Systems}
\author{Brendan Burkhart}
\maketitle

\tableofcontents
\newpage

\section{Introduction}

Signals can be broadly classified into discrete and continuous signals. For example, a continuous signal might be a function of a real variable representing time, or real variables representing a position in space, while a discrete signal is indexed by one or more integer variables. The variable(s) the signal is a function of are the \emph{independent variable(s)} of the signal, the values of the function are the dependent variables. For example, the voltage through a resistor could be the dependent variable of a signal where time is the independent variable.

Signals in the real world are usually continuous signals, but are often discretized for representing in computers and data into discrete signals. A \emph{digital} signal is a discrete signal where the dependent variable(s) have finite precision, which is what signals in a digital representation must be.

The value of a continuous signal $f$ at time $t \in \R$ is commonly denoted by $f(t)$, while a discrete signal $f$ at step $n \in \N$ would be denoted by $f[n]$. That is, parentheses are used for continuous signals, and square brackets for discrete.

Systems are processes, either physical ones like electronics and mechanics, or virtual ones like a software algorithm, that have input and output signals, generating the output signals based on the input signals and any internal state.

\end{document}
