\chapter{Optimization}
\label{ch:optimization}

\section{Introduction}

objective functions, constraints, feasible region

Given a feasible region $S \subseteq \R^n$, the objective function is a function $f: S \to \R$.

Maximize plant height due to nutrients $x_1, x_2$ given by $f(x_1, x_2) = 1 + x_1^2(x_2 - 1)^3e^{-x_1-x_2}$, such that $x_1, x_2 \geq 0$, $x_1 + x_2 \geq 6$, and $x_2 \geq \log x_1$. This is maximized by $(x_1, x_2) = (2, 4)$.

\begin{figure}[ht!]
    \centering
    \begin{tikzpicture}[scale=1.0]
        \begin{axis}[
            axis x line=middle,
            axis y line=middle,
            ymin=0,ymax=8,ylabel=$y$,
            xmin=0,xmax=8,xlabel=$x$
        ]
            \addplot[domain=0:6, blue, dashed, ultra thick] {6-\x};
            \addplot[domain=1:8, blue, dashed, ultra thick] {ln(\x)};
            \addplot[domain=0:6, blue, dashed, ultra thick] {0};
            \addplot[domain=0:6, blue, dashed, ultra thick, variable=\y] ({0}, {\y});
        \end{axis}
    \end{tikzpicture}
\caption{Example Feasible Region}
\label{fig:example-feasible-region}
\end{figure}

\begin{itemize}
    \item unbounded, e.g. minimize $\log x$ such that $0 < x \leq 7$
    \item bounded without solution, e.g. minimize $\log x$ such that $1 < x \leq 7$
    \item infeasible, e.g. minimize $\log x$ such that $1 < x \leq 0.5$
    \item interior point of the feasible region, e.g. minimize $f(x) = 3 + (x - 2)^2$ such that $1 \leq x \leq 3$. $x^* = 2$, $f'(2) = 0$
    \item boundary point of the feasible region, e.g. minimize $3 + (x - 2)^2$ such that $x \leq 10$. $x^* = 10$, but $f'(10) \neq 0$.
\end{itemize}

local minimums, local maximums, unique vs many solutions. Strict?

\begin{defn}
    For any $x, y \in \R^n$, the Euclidean length of $x$ is
    \[\norm{x} = \sqrt{x\cdot x} = \sqrt{\sum_{i=1}^{n}x_i^2},\]
    and the Euclidean distance between $x$ and $y$ is $\norm{x - y}$.
\end{defn}

\begin{defn}
    For all $x \in \R^n$ and $\varepsilon > 0$, the \emph{$\varepsilon$-neighborhood} of $x$ is
    \[N_{\varepsilon}(x) = \left\{y \in \R^n \compbar \norm{x - y} < \varepsilon \right\}.\]
\end{defn}

\begin{exmp}
    In $\R^1$, $N_{3}(7)$ is $(4, 10)$.
\end{exmp}

\begin{defn}
    For any $S \subseteq \R^n$ and $x \in S$, we say that $x$ is an \emph{interior point} of $S$ if there exists an $\varepsilon$ neighborhood of $x$ $N_{\varepsilon}(x) \subseteq S$. If every $N_{\varepsilon}(x)$ contains a point inside $S$ and a point not inside $S$, we say that $x$ is a \emph{boundary point} of $x$.
\end{defn}

\begin{defn}
    A set $S \subseteq \R^n$ is \emph{open} if every point in $S$ is an interior point of $S$, and \emph{closed} is $S$ contains every boundary point of $S$.
\end{defn}

\begin{exmp}
    In $\R^1$, any non-empty interval $[a, b]$ is closed, and and non-empty $(a, b)$ is open.
\end{exmp}

\begin{exmp}
    Both $\emptyset$ and $\R^n \subseteq \R^n$ are both open and closed.
\end{exmp}

continuous function on closed subset of $\R^n$ always attains minimum/maximum?

\begin{prop}
    Let $S \subseteq \R^n$. Then $S$ is open if and only if $\R^n - S$ is closed.
\end{prop}

\begin{defn}
    Let $S \subseteq \R^n$ and $f: \R^n \to \R$. Consider $x^* \in S$. We say that $x^*$ is a \emph{global minimizer} if for all $y \in S$, $f(x^*) \leq f(y)$, and a \emph{strict} global minimizer if for all $y \in S - \{x^*\}$, $f(x^*) < f(y)$.
\end{defn}

\begin{defn}
    Let $S \subseteq \R^n$ and $f: \R^n \to \R$. Consider $x^* \in S$. We say that $x^*$ is a \emph{local minimizer} if there exists an $\varepsilon$-neighborhood $N_{\varepsilon}(x^*)$ such that for all $y \in N_{\varepsilon}(x) \intersection S$, $f(x^*) \leq f(y)$, and a \emph{strict} local minimizer if for all $y \in \left(N_{\varepsilon}(x) \intersection S\right) - \{x^*\}$, $f(x^*) < f(y)$.
\end{defn}

stationary points := $\nabla f(x) = \vec{0}$

\begin{rmk}
    Let $S \subseteq \R^n$, $x^*$ be an interior point of $S$, and $f: S \to \R$ be a sufficiently smooth continuous function. If $x^* \in S$ is a local minimizer, then the \emph{gradient} of $f(x^*)$ is $\vec{0}$. However, $\nabla f(x^*) = \vec{0}$ does not imply that $x^*$ is a local minimizer. Note that
    \[\nabla f(x^*) = \vec{0},\;\nabla^2 f(x^*) \neq 0\] does imply that $x^*$ is a local minimizer/maximizer.
\end{rmk}
