\documentclass[12pt]{article}

\usepackage{amsmath}
\usepackage{amssymb}
\usepackage{amsthm}
\usepackage{centernot}
\usepackage{fullpage}
\usepackage{makecell}
\usepackage{tabularx}
\usepackage[hypcap=false]{caption}
\usepackage{tikz}
\usepackage{titling}
\usepackage{pdfpages}
\usepackage{enumitem}
\usepackage{multicol}

\usepackage{common}

\begin{document}

\title{Discrete Mathematics}
\author{Brendan Burkhart}
\maketitle

\tableofcontents
\newpage

\section{Binomial Coefficients}

\begin{defn}\label{binomial-coefficient}
Let $n, k \in \N$. The \emph{binomial coefficient} $n \choose k$ is the number of subsets with cardinality $k$ of a set with cardinality $n$.
\end{defn}

\begin{exmp}\proofbreak
    \begin{multicols}{2}
        \begin{itemize}
            \item $\binom{0}{0} = 1$
            \item $\binom{1}{0} = 1$
            \item $\binom{1}{1} = 1$
            \item $\binom{2}{1} = 2$
            \item $\binom{3}{2} = 3$
            \item $\binom{4}{2} = 6$
        \end{itemize}

        \columnbreak

        \begin{itemize}
            \item $\binom{5}{0} = 1$
            \item $\binom{5}{1} = 5$
            \item $\binom{5}{2} = 10$
            \item $\binom{5}{3} = 10$
            \item $\binom{5}{4} = 5$
            \item $\binom{5}{5} = 1$
        \end{itemize}
    \end{multicols}
\end{exmp}

\begin{prop}\label{binomial-complement}
    Let $n, k \in \N$ with $0 \leq k \leq n$. Then \[{n \choose k} = {n \choose n-k}.\]
\end{prop}

\begin{proof}
    Any selection of $k$ elements from a set with $n$ elements is equivalent to the selection of the other $n-k$ elements.
\end{proof}

\begin{exmp}
    $\binom{5}{2} = \binom{5}{3}$, and $\binom{3}{1} = \binom{3}{2}$.
\end{exmp}

\begin{defn}\label{pascals-triangle}
    \emph{Pascal's Triangle} is a particular way of writing out what is essentially a table of binomial coefficients that highlights several properties. Rows represent successive values for $n$, and columns represent $k$ --- the first row contains only a single column where $k=0$, in the second row the columns are $k=0$ and $k=1$, and so on.
    \begin{center}
        \begin{tabular}{rccccccccc}
            $n=0$:&    &    &    &    &  1\\\noalign{\smallskip\smallskip}
            $n=1$:&    &    &    &  1 &    &  1\\\noalign{\smallskip\smallskip}
            $n=2$:&    &    &  1 &    &  2 &    &  1\\\noalign{\smallskip\smallskip}
            $n=3$:&    &  1 &    &  3 &    &  3 &    &  1\\\noalign{\smallskip\smallskip}
            $n=4$:&  1 &    &  4 &    &  6 &    &  4 &    &  1\\\noalign{\smallskip\smallskip}
        \end{tabular}
    \end{center}
\end{defn}

\begin{rmk}
    Notice that $\binom{n}{0} = \binom{n}{n} = 1$, since there is only one set with cardinality $0$ (the empty set), and the only subset with cardinality $n$ of a set $X$ itself with cardinality $n$ is $X$. Alternatively, since $\binom{n}{0} = 1$, by Proposition \ref{binomial-complement} it follows that $\binom{n}{0} = \binom{n}{n}$ and so $\binom{n}{n} = 1$.
\end{rmk}

\begin{rmk}
    Note that each entry in Pascal's Triangle is the sum of the two elements immediately above it.
\end{rmk}

\begin{prop}{Pascal's Identity}\label{pascals-identity}
    Let $n \in \N$ with $0 < k < n$. Then $\binom{n}{k} = \binom{n-1}{k-1} + \binom{n-1}{k}$.
\end{prop}

\begin{proof}
    Let $X$ be a set with $|X| = n$. By Definition \ref{binomial-coefficient}, $\binom{n}{k}$ is the number of subsets with $k$ elements of $X$. Fix some $x \in X$. Then there are $\binom{n-1}{k}$ subsets of $X$ that do not include $x$, and $\binom{n-1}{k-1}$ which do, so $\binom{n}{k} = \binom{n-1}{k-1} + \binom{n-1}{k}$.
\end{proof}

\begin{thm}{Binomial theorem}\label{binomial-theorem}\proofbreak
    Let $F$ be a field. Then for any $x, y \in F$ and $n \in \Z_{\geq 1}$, \[\left(x + y\right)^n = \sum_{k=0}^n x^ky^{n-k}{n \choose k}.\]
\end{thm}

\begin{proof}
    We will prove this by induction. Our base case is when $n = 1$, so $(x + y)^n = x + y$. Then \[\sum_k=0^1x^ky^{1-k}\binom{1}{k} = x^0y^1\binom{1}{0} + x^1y^0\binom{1}{1} = x + y,\] so the base case is true.
    Now assume that for some $x, y \in F$ and $n \geq 1$ we have \[\left(x + y\right)^n = \sum_{k=0}^n x^ky^{n-k}{n \choose k}.\] Then since $(x + y)^{n+1} = (x+y)(x+y)^n$, we have $(x = y)^{n+1} = x(x+y)^n + y(x+y)^n$.
    Therefore, \[\left(x + y\right)^{n+1} = x\left(\sum_{k=0}^n x^ky^{n-k}\binom{n}{k}\right) + y\left(\sum_{k=0}^n x^ky^{n-k}\binom{n}{k}\right),\] so it follows that \[\left(x + y\right)^{n+1} = \left(\sum_{k=0}^n x^{k+1}y^{n-k}\binom{n}{k}\right) + \left(\sum_{k=0}^n x^ky^{n-k+1}\binom{n}{k}\right).\]
    We can now re-index the left sum to make the form of the terms match up better: \[\left(x + y\right)^{n+1} = \left(\sum_{k=1}^{n+1} x^{k}y^{n-k+1}\binom{n}{k-1}\right) + \left(\sum_{k=0}^n x^ky^{n-k+1}\binom{n}{k}\right).\]
    Now we can factor and match up all the terms except for the last term of the left sum, and the first term of the right sum.
    \[\left(x + y\right)^{n+1} = y^{n+1}\binom{n}{0} + \left(\binom{n}{k-1} + \binom{n}{k}\right)\left(\sum_{k=0}^{n} x^{k}y^{n-k+1}\right) + x^{n+1}\binom{n}{n}.\]
    By Pascal's Identity \ref{pascals-identity}, we then have
    \[\left(x + y\right)^{n+1} = y^{n+1}\binom{n}{0} + \binom{n+1}{k}\left(\sum_{k=0}^{n} x^{k}y^{n-k+1}\right) + x^{n+1}\binom{n}{n}.\]
    We can then move the two extra terms into the summation:
    \[\left(x + y\right)^{n+1} = \sum_{k=0}^{n+1}x^{k}y^{n+1-k}\binom{n+1}{k}.\]
    Therefore, the induction step is complete, and so by the induction principle we are done.
\end{proof}

\begin{prop}
    Let $n, k \in \N$ with $0 \leq k \leq n$. Then \[\binom{n}{k} = \frac{n!}{k!(n-k)!}.\]
\end{prop}

\end{document}
