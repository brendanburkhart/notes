\setchaptergraphic{
    \begin{tikzpicture}[
        shorten >=1pt,>={Stealth[round]},
        node distance=2.5 cm,
        every state/.style={draw=blue!50,very thick,fill=blue!20}
    ]
        \node[initial, state]   (A)                    {$q_0$};
        \node[state, accepting] (B) [right of=A]       {$q_1$};
        \node[state, accepting] (C) [above right of=B] {$q_2$};
        \node[state]            (R) [below right of=B] {$q_3$};

        \path[->] (A) edge [above]                  node [align=center] {0,1} (B)
                  (B) edge [above left, bend left]  node [align=center] {1} (C)
                      edge [below left]             node [align=center] {0} (R)
                  (C) edge [below right, bend left] node [align=center] {0,1} (B)
                  (R) edge [loop right]             node [align=center] {0,1} ();
    \end{tikzpicture}
}

\chapter{Computation Theory}
\label{ch:computation}

\section{Finite Automata}

\begin{defn}
    An \emph{alphabet} is a non-empty set whose elements are referred to as \emph{symbols}. \emph{Strings} (or \emph{words}) over a given alphabet are finite sequences of symbols from the alphabet. A \emph{language} is a set of strings over an alphabet.
\end{defn}

\begin{rmk}
    Alphabets and languages may be non-finite, but strings are generally considered to only include finite sequences. The empty string will be denoted by $\emptystring$.
\end{rmk}

\begin{defn}
    Let $\Sigma$ be an alphabet. Then the language consisting of all strings over $\Sigma$ of length $n$ is denoted by $\Sigma^{n}$. The \emph{Kleene closure} of $\Sigma$ is
    \begin{align*}
        \Sigma^{*} = \bigunion_{i\in\N}\Sigma^{i}.
    \end{align*}
\end{defn}

\begin{defn}
    Given languages $A$ and $B$, the \emph{concatenation} of $A$ and $B$, denoted by $A \concat B$, is
    \begin{align*}
        A \concat B = \left\{\alpha\beta : \alpha \in A, \beta \in B \right\},
    \end{align*}
    where $\alpha\beta$ is the string formed by concatenating $\alpha$ and $\beta$.
\end{defn}

\begin{defn}
    Let $L$ be a language. We can also apply the Kleene star operator to this language, creating its Kleene closure $L^*$ by defining
    \begin{align*}
        L^0 &= \{\emptystring\}, \\
        L^{i+1} &= L^{i} \concat L, \\
        L^{*} &= \bigunion_{i \geq 0}L^{i}.
    \end{align*}
\end{defn}

\begin{defn}
    A (deterministic) \emph{finite automaton} is a $5$-tuple $(Q, \Sigma, \delta, q_0, F)$, where
    \begin{itemize}
        \item $Q$ is a finite set of \emph{states},
        \item $\Sigma$ is an alphabet,
        \item $\delta: Q \times \Sigma \to Q$ is the \emph{transition function},
        \item $q_0 \in Q$ is the \emph{start state},
        \item $F \subseteq Q$ are the \emph{accept states}.
    \end{itemize}
    An automaton starts in state $q_0$, and is given an input consisting over a string over $\Sigma$. For each symbol $s$ in the input, the automaton moves from state $q$ to state $\delta(q, s)$. Once the end of the input is reached, if the state $q$ of the automaton is an accept state --- that is, $q \in F$, then the automaton is said to have \emph{accepted} the input.
\end{defn}

\begin{defn}
    The \emph{language accepted by} a finite automaton $M$ is a language $A$ consisting of all strings over $\Sigma$ that are accepted by the finite automaton. We denote this by $L(M) = A$. A \emph{regular language} is any language accepted by some finite automaton.
\end{defn}

\begin{rmk}
    Finite automata can be represented as a graph, where states are represented as nodes, and transitions are edges between states, labelled by the symbol that triggers the transition. When graphically representing a finite automaton as a graph, accept states may be represented as double-circled nodes.
\end{rmk}

\begin{exmp}
    \begin{figure}[ht!]
        \centering
        \begin{tikzpicture}[
            shorten >=1pt,>={Stealth[round]},
            node distance=2 cm,
            every state/.style={draw=blue!50,very thick,fill=blue!20}
        ]
            \node[initial, state]   (S) {$q_0$};
            \node[state]            (0) [right of=S] {$q_1$};
            \node[state]            (00) [right of=0] {$q_2$};
            \node[state, accepting] (001) [right of=00] {$q_3$};
    
            \path[->] (S)   edge [above, bend left]     node [align=center] {0} (0)
                      (S)   edge [loop below]           node [align=center] {1} ()
                      (0)   edge [above, bend left]     node [align=center] {0} (00)
                            edge [above, bend left]     node [align=center] {1} (S)
                      (00)  edge [loop above]           node [align=center] {0} ()
                            edge [above, bend left]     node [align=center] {1} (001)
                      (001) edge [below, bend left]     node [align=center] {0} (0)
                            edge [above, bend right=60] node [align=center] {1} (S);
        \end{tikzpicture}
        \caption{Example finite automaton}
        \label{fig:example-finite-automaton}
    \end{figure}

    In the finite automaton shown in Figure \ref{fig:example-finite-automaton}, $Q = \left\{q_0, q_1, q_2, q_3\right\}$, $\Sigma = \left\{0, 1\right\}$, $F = \{q_3\}$, and $\delta$ is
    \begin{center}
        \begin{tabular}{|c||c|c|}
            \hline
            \thead{$\delta$} & \thead{$0$}    & \thead{$1$} \\
            \hline\hline
            \textsc{$q_0$}   & \textsc{$q_1$} & \textsc{$q_0$} \\
            \hline
            \textsc{$q_1$}   & \textsc{$q_2$} & \textsc{$q_0$} \\
            \hline
            \textsc{$q_2$}   & \textsc{$q_2$} & \textsc{$q_3$} \\
            \hline
            \textsc{$q_3$}   & \textsc{$q_1$} & \textsc{$q_0$} \\
            \hline
        \end{tabular}.
    \end{center}

    This language accepted by this finite automaton is the set of all binary strings which end with \texttt{001}. We can use an inductive proof on the length of the longest input to prove this.
\end{exmp}

\begin{prop}\label{regular-language-complement}
    Let $A$ be a regular language, then $\complementof{A}$ is also a regular lanaguage.
\end{prop}

\begin{proof}
    Let $M = (Q, \Sigma, \delta, s_0, F)$ be a finite automaton that recognizes $A$, and define $M' = (Q, \Sigma, \delta, s_0, Q \setminus F)$. For any $w \in \Sigma^{*}$, we can see that $M$ and $M'$ reach the same final state, let this be $s_f$. Since $s_f \in F$ if and only if $s_f \in Q \setminus F$, it follows that $M'$ accepts $w$ if and only if $M$ \emph{does not}, therefore the regular language that $M'$ recognizes is precisely $\complementof{A}$.
\end{proof}

\begin{thm}\label{regular-language-union-intersection}
    Let $A$ and $B$ be regular languages, then $A \union B$ and $A \intersection B$ are also regular languages.
\end{thm}

\begin{proof}
    Let $\Sigma$ be the union of the alphabets of $A$ and $B$. Since $A$ and $B$ are regular, we know that there exists finite automata $M_A = \left(Q_1, \Sigma, \delta_1, s_1, F_1\right)$ and $M_B = \left(Q_2, \Sigma, \delta_2, s_2, F_2\right)$. Now we construct a new finite automaton that runs $M_A$ and $M_B$ in parallel. To do this, consider the finite automaton $M = \left(Q, \Sigma, \delta, s, F\right)$, where $Q = Q_1 \times Q_2$, $s = (s_1, s_2)$, and  $\delta: (Q_1 \times Q_2) \times \Sigma \to Q_1 \times Q_2$ is defined by $\delta\left((p_1, p_2), a\right) = \left(\delta_1\left(p_1, a\right), \delta_2\left(p_2, a\right)\right)$.
    
    If $F = F_1 \times F_2$, then $M$ accepts precisely the strings accepted by both $M_A$ and $M_B$ --- that is, $L(M) = A \intersection B$. If instead $F = \left(F_1 \times Q_2\right) \union \left(Q_1 \times F_2\right)$, then $M$ accepts the strings accepted by at least one of $M_A$ and $M_B$ --- that is, $L(M) = A \union B$.
\end{proof}

\begin{defn}
    A \emph{non-deterministic} finite automaton is a $5$-tuple $(Q, \Sigma, \delta, q_0, F)$, where
    \begin{itemize}
        \item $Q$ is a finite set of \emph{states},
        \item $\Sigma$ is an alphabet,
        \item $\delta: Q \times \left(\Sigma \union \{\emptystring\}\right) \to 2^{Q}$ is the \emph{transition function},
        \item $q_0 \in Q$ is the \emph{start state},
        \item $F \subseteq Q$ are the \emph{accept states}.
    \end{itemize}
    A non-deterministic finite automaton $M$ accepts an input $w$ over $\Sigma$ when there exists $w_{\emptystring}$ that causes $M$ to end in an accept state, where $w_{\emptystring}$ is constructed by inserting any finite number of $\emptystring$ into $w$.
\end{defn}

\begin{rmk}
    Standard deterministic finite automata may be referred to as DFAs, while non-deterministic finite automata may be referred to as NFAs.
\end{rmk}

\begin{rmk}
    A NFA can essentially be viewed as a DFA which can split into multiple parallel copies of itself.
\end{rmk}

\begin{defn}
    Given a transition function $\delta: Q \times \left(\Sigma \union \{\emptystring\}\right) \to 2^{Q}$ and $R \subseteq Q$,
    the \emph{$\emptystring$-closure} of $R$ is the set of all states reachable from $R$ via zero or more \emph{$\emptystring$-transitions}. Formally, we recursively define
    \begin{align*}
        E_0\left(R\right) &= R, \\
        E_{i+1}\left(R\right) &= \left\{q \in Q : \exists r \in E_{i}\left(R\right)\left[q \in \delta\left(r, \emptystring\right)\right]\right\}.
    \end{align*}
    and then the $\emptystring$-closure of $R$ is
    \begin{align*}
        E\left(R\right) = \bigunion_{0\leq i < n}E_{i}\left(R\right),
    \end{align*}
    where $n = \abs{Q}$.
\end{defn}

\begin{lemma}\label{epsilon-closure}
    Consider an NFA with states $Q$ and transition function $\delta$, and consider $s\in Q$ and $R \subseteq Q$. Then $s \in E(R)$ if and only if $s$ is reachable from some $q \in R$ via zero or more $\emptystring$-transitions.
\end{lemma}

\begin{proof}\proofbreak
    ($\implies$) If $s \in E(R)$, then $s \in E_{i}(R)$ for some $0 \leq i < n$. If $i = 0$, then $s \in R$ and so $s$ is reachable from $q = s \in R$ via zero $\emptystring$-transitions. If $i > 0$, then there exists $r \in E_{i=1}(R)$ such that $s \in \delta(r, \emptystring)$, and so by induction there exists $q \in R$ such that $s$ is reachable from $q$ via zero or more $\emptystring$-transitions.

    ($\impliedby$) If $s$ is reachable from $q_0 \in R$ via zero of more $\emptystring$-transitions, then clearly there exists a sequence $q_0, q_1, \ldots, q_{k}, s$ such that $q_{i+1} \in \delta(q_i, \emptystring)$ and $s \in \delta(q_k, \emptystring)$. Therefore, $q_i \in E_i(R)$, and so $s \in E_{k+1}$. Notice that if $q_i = q_j$, we can take reach $s$ from $q_0$ via the subsequence $q_0, q_1, \ldots, q_i, q_{j+1}, q_{j+2}, \ldots, q_k, s$. Since $n = \abs{Q}$, it follows by the pigeonhole principle that such a sequence with at most $n$ states must exist, and so $k + 2 \leq n$, which in turn implies $s \in E_{i = k+1}(R)$ for some $0 \leq i < n$, and so $s \in E(R)$.
\end{proof}

\begin{lemma}\label{epsilon-closure-union}
    Consider an NFA with states $Q$ and transition function $\delta$, and consider $A \subseteq Q$ and $B \subseteq Q$. Then
    \begin{align*}
        E(A) \union E(B) = E(A \union B).
    \end{align*}
\end{lemma}

\begin{proof}
    If $x \in E(A) \union E(B)$, then since $A$ and $B$ are arbitrary we have without loss of generality that $x \in E(A)$. It follows by Lemma \ref{epsilon-closure} that $x$ is reachable via $\emptystring$-transitions from some $a \in A$, and since $a \in A \union B$ we then have $x \in E(\union B)$.

    If $x \in E(A \union B)$, then by Lemma \ref{epsilon-closure} we know that $x$ is reachable via $\emptystring$-transitions from some $a \in A$ or some $b \in B$. Then $x \in E(A)$ or $x \in E(B)$, and so $x \in E(E) \union E(B)$.
\end{proof}

\begin{thm}\label{dfa-nfa-equivalence}
    Let $N$ be a non-deterministic finite automaton, and $L$ be the language it accepts. Then $L$ is a regular language. Equivalently, there is an equivalent deterministic finite automaton for every non-deterministic finite automaton.
\end{thm}

\begin{proof}
    Let $V = \left(Q, \Sigma, \delta, q_0, F\right)$ be a non-deterministic finite automaton. We will construct an equivalent deterministic finite automaton $M$. Let $M = \left(Q', \Sigma, \delta', q_0', F'\right)$, where
    \begin{itemize}
        \item $Q' = 2^Q$,
        \item For all $R \in Q'$, define $\delta'(R, a) = E\left(\bigunion_{r \in R}\delta(r, a)\right)$,
        \item $q_0' = \left\{q_0\right\}$,
        \item $F' = \left\{R \in Q \compbar R \intersection F \neq \emptyset\right\}$.
    \end{itemize}

    For any $w \in \Sigma^*$ such that $V$ accepts $w$, by we can construct $w_{\emptystring}$ from $w$ by inserting some number of $\emptystring$'s into $w$ such that $w_{\emptystring}$ causes $V$ to run to an accept state. Let $q_0, q_1, \ldots, q_n$ be the states that $w_{\emptystring}$ causes $V$ to visit. I claim that if $V$ is in state $q_i$ after seeing $w_1, \ldots, w_i$ then $M$ is in state $R$ where $q_i \in R$. We will prove this by induction. The base case is true by construction since $V$ starts in state $q_0$ and $M$ in state $\{q_0\}$. The induction step simply amounts to showing that if $q_i \in R$ then $q_{i+1} \in \delta'(R, w_{i+1})$.
    \begin{align*}
        R' = \delta'(R, w_{i+1}) = E\left(\bigunion_{r\in R}\delta(r, w_{i+1})\right).
    \end{align*}
    Since $w_i$ causes $V$ to move from $q_i$ to $q_{i+1}$, it follows that there is a (possibly empty) sequence of $\emptystring$-transitions from $q_i$ to some state $a$, then $d \in \delta(a, w_i)$, and then a (again possibly empty) sequence of $\emptystring$-transitions from $d$ to $q_{i+1}$. By Lemma \ref{epsilon-closure}, we know that $a \in R$, and therefore $d \in \bigunion_{r \in R}\delta(r, w_i)$, and so again using Lemma \ref{epsilon-closure} we can conclude that $q_{i+i} \in R'$.

    We have therefore shown that $q_n \in R_f$, where $R_f$ is the final state of $M$. Since $V$ accepts $w$ we know that $q_n \in F$ and so $R_f \intersection F \neq \empty$ which implies $R_f \in F'$, and so $M$ also accepts $w$.

    Now consider instead $w$ accepted by $M$. Let $R_0 = \{q_0\}, R_1, \ldots, R_n$ be the sequence of states visited by $M$. Since $w$ is accepted by $M$, we know $R_n \intersection F \neq \emptyset$, so take $q_n \in R_n \intersection F$. It follows that there exists some $q_{n-1} \in R_{n-1}$ such that $q_n \in E(\delta(q_{n-1}, w_{i+1}))$ by Lemmas \ref{epsilon-closure} and \ref{epsilon-closure-union}, and so inductively there must exist a sequence of states of $V$ $q_0, \ldots, q_n$ such that $q_n \in F$ and $q_{i+1}$ is reachable from $q_i$ via $w_i$ and some number of $\emptystring$-transitions. Let $w_{\emptystring}$ be constructed from $w$ by inserting the relevant $\emptystring$'s, and then $w_{\emptystring}$ causes $V$ to end in an accept state, and so $w$ is accepted by $V$ by definition.
\end{proof}

\begin{rmk}
    In addition to a proof of the equivalence of deterministic and non-deterministic finite automata, this is an explicit method to construct a deterministic finite automaton from a non-deterministic one.
\end{rmk}

\begin{thm}\label{regular-language-concatenation}
    Let $A$ and $B$ be regular languages, then $A \concat B$ is also regular.
\end{thm}

\begin{proof}
    Let $A = (Q_A, \Sigma, \delta_A, q_A, F_A)$ and $B = (Q_B, \Sigma, \delta_B, q_B, F_B)$ be deterministic finite automata recognizing language $A$ and $B$ respectively. We now construct a non-deterministic finite automaton which recognizes $A \concat B$. Consider $M = \left(Q, \Sigma, \delta, q_0, F\right)$, where
    \begin{itemize}
        \item $Q = Q_A \union Q_B$,
        \item For $a \in \Sigma$, define $\delta(q, a) = \begin{dcases}
            \{\delta_A(q, a)\}, &q \in Q_A \\
            \{\delta_B(q, a)\}, &q \in Q_B \\
        \end{dcases}$ and $\delta(q, \emptystring) = \begin{dcases}
            \{q_B\}, &q \in F_A \\
            \emptyset, &\textrm{otherwise} \\
        \end{dcases}$,
        \item $q_0 = q_A$,
        \item $F = F_B$.
    \end{itemize}
    Notice that if $w = \alpha\beta$, then the input $\alpha\emptystring\beta$ will cause $M$ to end in an accept state, and so any such $w$ is accepted. If $w$ is accepted, then since the only transition possible from states in $Q_A$ to those in $Q_B$ (which contains all possible accept states) is an $\emptystring$-transition from a state in $F_A$, and so $w = \alpha\beta$ for some $\alpha \in A$ and $\beta \in B$.

    Since $M$ recognizes $A \concat B$, by Theorem \ref{dfa-nfa-equivalence} we know that $A \concat B$ must be a regular language.
\end{proof}

\begin{thm}\label{regular-language-kleene-star}
    Let $A$ be a regular language, then $A^{*}$ is also regular.
\end{thm}

\begin{proof}
    Given a deterministic finite automaton $M = (Q, \Sigma, \delta, q_0, F)$, we make a number of small modifications in order to produce a non-deterministic finite automaton $N$ recognizing $A^{*}$ and thus proving it is regular. Specifically, we add $\emptystring$-transitions from every accept state of $M$ back to $q_0$ --- this allows $N$ to recognize any string you can form by concatenating strings from $A$. We also introduce a new start state with a single $\emptystring$-transition to the original start state, and make this new start state an accept state, thus allowing $N$ to also recognize $\emptystring$.

    Formally, we define $N = (Q_N, \Sigma, \delta_N, q_N, F_N)$ where
    \begin{itemize}
        \item $Q_N = Q \union \{q_N\}$,
        \item $\delta_N(q, a) = \begin{dcases}
            \delta(q, a), &q \in Q \\
            \{\}, &q = q_N \\
        \end{dcases}$ and $\delta_N(q, \emptystring) = \begin{dcases}
            \delta(q, \emptystring) \union \{q_0\}, &q \in F, q \neq q_N \\
            \delta(q, \emptystring), &q \not\in F, q \neq q_N \\
            \{q_0\}, &q = q_N \\
        \end{dcases}$,
        \item $F_N = F \union \{q_N\}$.
    \end{itemize}
\end{proof}

\begin{defn}
    The \emph{regular operations} are union, intersection, complement, concatenation, and star, under which we have shown the set of regular languages is closed.
\end{defn}

\section{Regular Expressions}

\begin{defn}
    A \emph{regular expression} is a sequence of regular operations applied to regular languages. Formally, given an alphabet $\Sigma$, we recursively define the regular expressions over $\Sigma$ as:
    \begin{itemize}
        \item $a$, for any $a \in \Sigma$,
        \item $\emptystring$,
        \item $\emptyset$,
        \item $R_1 \union R_2$, where $R_1$ and $R_2$ are themselves regular expressions,
        \item $R_1 \concat R_2$, where $R_1$ and $R_2$ are themselves regular expressions,
        \item $R_1^{*}$, where $R_1$ is a regular expression,
    \end{itemize}
\end{defn}

\begin{defn}
    For a regular expression $R_1$, we also define $R_1^{+} = R_1R_1^{*}$.
\end{defn}

\begin{exmp}
    The regular expression $\{1\}^{*}(\{0\} \concat \{1\}^{+})^{*}$, also written in a more condensed form as $1^{*}(01^{+})^{*}$, is the language of consisting of binary strings where every $0$ is followed by at least one $1$.
\end{exmp}

\begin{lemma}\label{regex-to-regular-language}
    The language described by any regular expression is a regular language.
\end{lemma}

\begin{proof}
    Let $R$ be a regular expression that describes a language $L$ over an alphabet $\Sigma$. We recursively construct a non-deterministic finite automaton that recognizes $L$ in order to prove that $L$ must be regular. First, any regular expression by definition must have one of the six forms listed in our definition.

    If $R = a$, then $L = \{a\}$ by definition, which is recognized by the non-deterministic finite automaton:
    \begin{figure}[ht!]
        \centering
        \begin{tikzpicture}[
            shorten >=1pt,>={Stealth[round]},
            node distance=2 cm,
            every state/.style={draw=blue!50,very thick,fill=blue!20}
        ]
            \node[initial, state]   (S)              {$q_1$};
            \node[state, accepting] (A) [right of=S] {$q_2$};
    
            \path[->] (S) edge [above] node [align=center] {a} (A);
        \end{tikzpicture}
    \end{figure}

    If $R = \emptystring$, it can be recognized by the non-deterministic finite automaton:
    \begin{figure}[ht!]
        \centering
        \begin{tikzpicture}[
            shorten >=1pt,>={Stealth[round]},
            node distance=2 cm,
            every state/.style={draw=blue!50,very thick,fill=blue!20}
        ]
            \node[initial, state, accepting] (S) {};
        \end{tikzpicture}
    \end{figure}

    If $R = \emptyset$, it can be recognized by the non-deterministic finite automaton:
    \begin{figure}[ht!]
        \centering
        \begin{tikzpicture}[
            shorten >=1pt,>={Stealth[round]},
            node distance=2 cm,
            every state/.style={draw=blue!50,very thick,fill=blue!20}
        ]
            \node[initial, state] (S) {};
        \end{tikzpicture}
    \end{figure}

    Formally, these finite automata are
    \begin{itemize}
        \item $(\{q_0, q_1\}, \Sigma, \delta, q_0, \{q_1\})$ where $\delta(q_0, a) = \{q_1\}$ and $\delta(q, r) = \emptyset$ otherwise,
        \item $(\{q_0\}, \Sigma, \delta, q_0, \{q_0\})$ where $\delta(q, r) = \emptyset$,
        \item $(\{q_0\}, \Sigma, \delta, q_0, \emptyset)$ where $\delta(q, r) = \emptyset$.
    \end{itemize}

    From here, we proceed by induction. In the base case, $R$ is one of the above expressions and has been shown to be regular. Now assuming that $R_1$ and $R_2$ are regular, it follows that
    \begin{itemize}
        \item $R_1 \union R_2$ is regular by Theorem \ref{regular-language-union-intersection},
        \item $R_1 \concat R_2$ is regular by Theorem \ref{regular-language-concatenation},
        \item and $R_1^{*}$ is regular by Theorem \ref{regular-language-kleene-star}.
    \end{itemize}
    Since any regular expression $R$ must be recursively composed of these six forms, it follows that any regular expression describes a regular language.
\end{proof}

\begin{rmk}
    We can explicitly convert any regular expression into a non-deterministic finite automaton.
\end{rmk}

\begin{exmp}
    Consider the regular expression $(0 \union 1)^{*}010$. We can construct, piece by piece, an NFA to recognize the language described by this regular expression as demonstrated in Figure \ref{fig:regex-nfa-construction-example}.
    \begin{figure}[ht!]
        \centering
        \begin{tikzpicture}[
            shorten >=1pt,>={Stealth[round]},
            node distance=1.5 cm,
            every state/.style={draw=RoyalBlue!50,very thick,fill=RoyalBlue!20}
        ]
            \node[initial, state]   (S)              {};
            \node[state, accepting] (0) [right of=S] {};

            \path[->] (S) edge [above] node [align=center] {$0$} (0);
        \end{tikzpicture}

        \begin{tikzpicture}[
            shorten >=1pt,>={Stealth[round]},
            node distance=1.5 cm,
            every state/.style={draw=red!50,very thick,fill=red!20}
        ]
            \node[initial, state]   (S)              {};
            \node[state, accepting] (1) [right of=S] {};

            \path[->] (S) edge [above] node [align=center] {$1$} (1);
        \end{tikzpicture}

        \begin{tikzpicture}[
            shorten >=1pt,>={Stealth[round]},
            node distance=1.5 cm,
            every state/.style={draw=blue!50,very thick,fill=blue!20},
            rednode/.style={draw=red!50,very thick,fill=red!20},
            bluenode/.style={draw=RoyalBlue!50,very thick,fill=RoyalBlue!20},
            purplenode/.style={draw=Fuchsia!50,very thick,fill=Fuchsia!20},
        ]
            \node[initial, state, purplenode] (S)                     {};
            \node[state, bluenode]            (E0) [above right of=S] {};
            \node[state, rednode]             (E1) [below right of=S] {};
            \node[state, accepting, bluenode] (A0) [right of=E0]      {};
            \node[state, accepting, rednode]  (A1) [right of=E1]      {};

            \path[->] (S)  edge [below] node [align=center] {$\emptystring$} (E0)
                           edge [above] node [align=center] {$\emptystring$} (E1)
                      (E0) edge [above] node [align=center] {$0$} (A0)
                      (E1) edge [above] node [align=center] {$1$} (A1);
        \end{tikzpicture}

        \begin{tikzpicture}[
            shorten >=1pt,>={Stealth[round]},
            node distance=1.5 cm,
            every state/.style={draw=blue!50,very thick,fill=blue!20},
            rednode/.style={draw=red!50,very thick,fill=red!20},
            bluenode/.style={draw=RoyalBlue!50,very thick,fill=RoyalBlue!20},
            purplenode/.style={draw=Fuchsia!50,very thick,fill=Fuchsia!20},
            yellownode/.style={draw=BurntOrange!50,very thick,fill=BurntOrange!20},
        ]
            \node[initial, state, accepting, yellownode] (0)                     {};
            \node[state, purplenode]                     (S)  [right of=0]       {};
            \node[state, bluenode]                       (E0) [above right of=S] {};
            \node[state, rednode]                        (E1) [below right of=S] {};
            \node[state, accepting, bluenode]            (A0) [right of=E0]      {};
            \node[state, accepting, rednode]             (A1) [right of=E1]      {};

            \path[->] (0)  edge [above]             node [align=center] {$\emptystring$} (S)
                      (S)  edge [below]             node [align=center] {$\emptystring$} (E0)
                           edge [above]             node [align=center] {$\emptystring$} (E1)
                      (E0) edge [above]             node [align=center] {$0$} (A0)
                      (E1) edge [above]             node [align=center] {$1$} (A1)
                      (A0) edge [above, out=125, in=90] node [align=center] {$\emptystring$} (S)
                      (A1) edge [below, out=-125, in=-90]  node [align=center] {$\emptystring$} (S);
        \end{tikzpicture}

        \begin{tikzpicture}[
            shorten >=1pt,>={Stealth[round]},
            node distance=1.5 cm,
            every state/.style={draw=ForestGreen!50,very thick,fill=ForestGreen!20},
        ]
            \node[initial, state]    (S)                {};
            \node[state]             (A)  [right of=S]  {};
            \node[state]             (AE) [right of=A]  {};
            \node[state]             (B)  [right of=AE] {};
            \node[state]             (BE) [right of=B]  {};
            \node[state, accepting]  (A2) [right of=BE] {};

            \path[->] (S)  edge [above] node [align=center] {$a$}           (A)
                      (A)  edge [above] node [align=center] {$\emptystring$} (AE)
                      (AE) edge [above] node [align=center] {$b$}           (B)
                      (B)  edge [above] node [align=center] {$\emptystring$} (BE)
                      (BE) edge [above] node [align=center] {$a$}           (A2);
        \end{tikzpicture}

        \begin{tikzpicture}[
            shorten >=1pt,>={Stealth[round]},
            node distance=1.5 cm,
            every state/.style={draw=blue!50,very thick,fill=blue!20},
            rednode/.style={draw=red!50,very thick,fill=red!20},
            bluenode/.style={draw=RoyalBlue!50,very thick,fill=RoyalBlue!20},
            purplenode/.style={draw=Fuchsia!50,very thick,fill=Fuchsia!20},
            yellownode/.style={draw=BurntOrange!50,very thick,fill=BurntOrange!20},
            greennode/.style={draw=ForestGreen!50,very thick,fill=ForestGreen!20},
        ]
            \node[initial, state, yellownode]  (0)                     {};
            \node[state, purplenode]           (S)  [right of=0]       {};
            \node[state, bluenode]             (E0) [above right of=S] {};
            \node[state, rednode]              (E1) [below right of=S] {};
            \node[state, bluenode]             (A0) [right of=E0]      {};
            \node[state, rednode]              (A1) [right of=E1]      {};
            \node[state, greennode]            (2S)  [below of=A1]     {};
            \node[state, greennode]            (2A)  [right of=2S]     {};
            \node[state, greennode]            (2AE) [right of=2A]     {};
            \node[state, greennode]            (2B)  [right of=2AE]    {};
            \node[state, greennode]            (2BE) [right of=2B]     {};
            \node[state, accepting, greennode] (2A2) [right of=2BE]    {};

            \path[->] (0)   edge [above]                   node [align=center] {$\emptystring$} (S)
                      (S)   edge [below]                   node [align=center] {$\emptystring$} (E0)
                            edge [above]                   node [align=center] {$\emptystring$} (E1)
                      (E0)  edge [above]                   node [align=center] {$0$}           (A0)
                      (E1)  edge [above]                   node [align=center] {$1$}           (A1)
                      (A0)  edge [above, out=135, in=90]   node [align=center] {$\emptystring$} (S)
                      (A1)  edge [below, out=-135, in=-90] node [align=center] {$\emptystring$} (S)
                      (2S)  edge [above]                   node [align=center] {$a$}           (2A)
                      (2A)  edge [above]                   node [align=center] {$\emptystring$} (2AE)
                      (2AE) edge [above]                   node [align=center] {$b$}           (2B)
                      (2B)  edge [above]                   node [align=center] {$\emptystring$} (2BE)
                      (2BE) edge [above]                   node [align=center] {$a$}           (2A2)
                      (0)   edge [below, bend right]       node [align=center] {$\emptystring$} (2S)
                      (A0)  edge [right, bend left]        node [align=center] {$\emptystring$} (2S)
                      (A1)  edge [right]                   node [align=center] {$\emptystring$} (2S);
        \end{tikzpicture}

        \caption{Step-by-step construction of an NFA to recognize $(0 \union 1)^{*}010$}
        \label{fig:regex-nfa-construction-example}
    \end{figure}
\end{exmp}

\begin{defn}
    A \emph{generalized non-deterministic finite automaton} (GNFA) is a variant of the non-deterministic finite automaton which consumes multiple symbols of the input at a time, by non-deterministically matching the input against regular expressions. We will only consider GNFAs in a \emph{standard form}, which has exactly one start state, one accept state, as well as outgoing transitions from the start state, ingoing transitions from every state to the accept state, and transitions between every other state. Formally, it is a $5$-tuple $(Q, \Sigma, \delta, q_{\textrm{start}}, q_{\textrm{accept}})$, where
    \begin{itemize}
        \item $Q$ is a finite set of \emph{states},
        \item $\Sigma$ is an alphabet,
        \item $\delta: (Q - q_{\textrm{accept}}) \times \left(Q - q_{\textrm{start}}\right) \to \mathcal{R}(\Sigma)$, where $\mathcal{R}(\Sigma)$ is the set of all regular expressions over $\Sigma$.
        \item $q_{\textrm{start}} \in Q$ is the \emph{start state},
        \item $q_{\textrm{accept}} \in Q$ is the \emph{accept state}.
    \end{itemize}

    A GNFA accepts $w \in \Sigma^{*}$ if $w = w_1w_2\cdots w_k$ where $w_i \in \Sigma^{*}$ and there exists $q_0, q_1, \ldots, q_k$ such that:
    \begin{itemize}
        \item $q_0 = q_{\textrm{start}}$,
        \item $q_k = q_{\textrm{accept}}$,
        \item $w_i \in L(R_i)$ where $R_i = \delta\left(q_{i-1}, q_i\right)$.
    \end{itemize}
\end{defn}

\begin{lemma}\label{dfa-to-gnfa}
    Any regular language can be converted into a generalized non-deterministic finite automaton.
\end{lemma}

\begin{proof}
    Consider a regular language $L$ that is recognized by a deterministic finite automaton $M = (Q_0, \Sigma, \delta_0, q_{s}, F)$. From this, we construct an equivalent GNFA $G = (Q, \Sigma, \delta, q_{\textrm{start}}, q_{\textrm{accept}})$, where $Q = Q_0 \union \{q_{\textrm{start}}, q_{\textrm{accept}}\}$. We add $\emptystring$-transitions from every state in $F$ to $q_{\textrm{accept}}$, and a $\emptystring$-transition from $q_{\textrm{start}}$ to $q_{s}$. Every transtions from state $q_0$ to state $q_1$ is replaced with a regular expression describing their union, and for states without transitions we use the regular expression $\emptyset$. Formally, let
    \begin{align*}
        s_{\alpha\beta} &= \left\{s \in \Sigma : \delta_0\left(\alpha, s\right) = b\right\}, \\
        \delta(\alpha, \beta) &= \begin{dcases}
            \varepsilon, &\alpha \in F, \beta = q_{\textrm{accept}} \\
            \varepsilon, &\alpha = q_{\textrm{start}}, \beta = q_{s} \\
            s_{\alpha\beta}, &\textrm{otherwise}
        \end{dcases}.
    \end{align*}

    This GNFA recognizes an equivalent language since a transition between $q_i$ and $q_{i+1}$ for symbol $s \in \Sigma$ exists precisely when $\delta(q_i, q_{i+1})$ is a union containing $s$.
\end{proof}

\begin{lemma}\label{gnfa-to-regex}
    The language recognized by any generalized non-deterministic finite automaton can be described by a regular expression.
\end{lemma}

\begin{proof}
    $G = (Q, \Sigma, \delta, q_{\textrm{start}}, q_{\textrm{accept}})$ be a GNFA. We know that $\abs{Q} \geq 2$ since it must contain at least the start and accept states. If $\abs{Q} = 2$, then $\delta(q_{\textrm{start}}, q_{\textrm{accept}})$ is a regular expression that describes the language recognized by $G$. In the case that $\abs{Q} > 2$, we will show how to produce an equivalent GNFA $G'$ with exactly one less state. Since $\abs{Q} > 2$, there must exist some $q \in Q$ such that $q \neq q_{\textrm{start}}$ and $q \neq q_{\textrm{accept}}$. Let $Q' = Q \setminus \{q\}$, and let
    \begin{align*}
        \delta'(\alpha, \beta) = \delta(\alpha, \beta) \union \left[\delta(\alpha, q) \concat \delta(q, q)^{*} \concat \delta(q, \beta)\right].
    \end{align*}
    Consider any $w$ accepted by $G$, so by definition there exists $w_1w_2\cdots w_k$ and $q_0, q_1, \ldots, q_k$ such that $w_i \in \delta(q_{i-1}, q_i)$. Construct $r_0, r_1, \ldots, r_m$ from $q_0, q_1, \ldots, q_k$ by removing any $q$'s, and then construct $w = v_1v_2\cdots v_m$ by concatenating the appropriate $w_i$'s. If no $q$'s were removed between $r_{i-1}$ and $r_i$ then trivially $v_i = w_{\ell} \in \delta(q_{\ell-1}, q_{\ell}) \subseteq \delta'(r_{i-1}, r_i)$. If any $q$'s were removed, then $\delta'(r_{i-1}, r_i)$ contains the concatenation of the relevant regular expressions, and so $v_i \in \delta'(r_{i-1}, r_i)$, so $w$ is also accepted by $G'$.

    If $w = w_1w_2\cdots w_l$ is accepted by $G'$, then anytime of the transitions added via union is used, we can simply insert $q$'s back into the state sequence and break $w_i$ into the necessary pieces, so $w$ must be accepted by $G$.
\end{proof}

\begin{thm}
    A language can be described by a regular expression if and only if it is a regular language.
\end{thm}

\begin{proof}
    If a language can be described by a regular expression, then by Lemma \ref{regex-to-regular-language} it must be regular. If a language is regular, by Lemma \ref{dfa-to-gnfa} a generalized non-deterministic finite automaton can be constructed to recognize it, and then by Lemma \ref{gnfa-to-regex} the automaton can be converted into an equivalent regular expression.
\end{proof}

\section{Non-Regular Languages}

\begin{lemma}{Pumping Lemma}{\label{pumping-lemma}}\proofbreak
    Let $A$ be a regular language. There exists $p \in \N$, the \emph{pumping length}, such that if $s \in A$ has length at least $p$, then we can divide $s$ into three pieces $s = xyz$ satisfying
    \begin{itemize}
        \item $\abs{y} > 0$,
        \item $\abs{xy} \leq p$,
        \item $xy^{i}z \in A$ for all $i \geq 0$.
    \end{itemize}
\end{lemma}

\begin{proof}
    Let $M = (Q, \Sigma, \delta, q_s, F)$ be a deterministic finite automaton recognizing $A$. Let $p = \abs{Q}$. Then for any $s = s_1s_2 \cdots s_n \in A$, where $n \geq p$, let $r_1, r_2, \ldots, r_n, r_{n+1}$ be the sequence of states that $M$ visits in input $s$. Since $n+1 \geq p + 1 > p$, by the pigeonhole principle \ref{pigeonhole}, it follows that at least one state must be visited twice within the first $p+1$ states --- that is, that there must be $i > j$ such that $r_i = r_j$ and $i \leq p + 1$.

    Then, take $x = s_1s_2 \cdots s_{i-1}$, $y = s_{i}s_{i+1} \cdots s_{j-1}$, and $z = s_{j}s_{j+1}\cdots s_n$. Notice that since $i > j$, $j-1 \geq i$, and so $\abs{y} > 0$. Furthermore, $\abs{xy} = j-1$ and since $j < i \leq p + 1$, it follows that $j-1 < i-1 \leq p$, and so $\abs{xy} \leq p$. Finally, since $r_{i}r_{i+1}\cdots r_{j}$ is a cycle that occurs on input $y$, we clearly have $xy^{i}z \in A$ for all $i \geq 0$.
\end{proof}

\begin{exmp}
    Consider the language $A = \{0^n1^n | n \geq 0\}$. Assume, for the sake of contradiction, that $A$ is regular and so by the pumping lemma \ref{pumping-lemma} there must exist a pumping length $p$. We consider $s = 0^{p}1^{p} \in A$, and by the pumping lemma are guaranteed $xyz = 0^{p}1^{p}$ since $\abs{s} = 2p \geq p$. In particular, we have $\abs{xy} \leq p$, and so $xy = 0^{t}$ for some $t \leq p$. It follows that $y = 0^{s}$ for some $0 < s \leq p$, and so by the pumping lemma we must have $xy^{i}z = 0^{p+(i-1)s}1^{p} \in A$. However, this is false by the definition of $A$ whenever $i \neq 1$, and so we have a contradiction. It follows that $A$ cannot be a regular language.
\end{exmp}

\begin{exmp}
    Consider the language $C$ consisting of all binary strings containing an equal number of $0$s and $1$s. Notice that $A = C \intersection 0^{*}1^{*}$ --- that is, $A$ is precisely the subset of $0^{*}1^{*}$ where the number of $0$s and $1$s are equal. If $C$ was regular, then $A$ would be regular since regular languages are closed under intersection. Therefore, $C$ cannot be a regular language.
\end{exmp}

\section{Grammar}

\begin{defn}
    grammar, variables (can be changed), terminals
\end{defn}

\begin{defn}
    Context-free grammar
\end{defn}

\begin{thm}
    Convert DFA into a grammar
\end{thm}

\begin{exmp}
    Convert DFA into a grammar
\end{exmp}
