\setchaptergraphic{}

\chapter{Statistics}
\label{ch:statistics}

\section{Simple Random Sampling}

Consider a finite population of $N$ distinct objects, each with two associated measurements $x$ and $y$. We denote the $k$th object by $z_k = ((x_k, y_k), k)$. Let $\underline{\pi}(z_k) = (x_k, y_k)$, let $\pi_x(z_k) = x_k$, let $\pi_y(z_k) = y_k$, and finally let $\pi_{\mathrm{ind}}(z_k) = k$.

\subsection{Population Parameters}

Population parameters are \emph{deterministic} functions of the measurements associated with objects in a population.

For example, the population mean of the $y$-measurements is
\begin{align*}
    \mu_y = \frac{1}{N}\sum_{k=1}^{N}\pi_{y}(z_k),
\end{align*}
and the population variance of the $y$-measurements is
\begin{align*}
    \sigma_{y}^2 = \frac{1}{N}\sum_{k=1}^{N}\left(y_k - \mu_y\right)^2.
\end{align*}

The population covariance between the $x$ and $y$ measurements is
\begin{align*}
    \sigma_{xy} = \frac{1}{N}\sum_{k=1}^{N}\left(x_k-\mu_x\right)\left(y_k-\mu_y\right).
\end{align*}

\subsection{Sample Parameters}

Suppose we have a finite sample $Z_1, \ldots, Z_n$ of the entire population. Functions of such samples are known as \emph{statistics}, and will be used to construct useful \emph{estimators} of our population parameters. For example, the \emph{sample mean} can be used to estimate the population mean, and is defined by
\begin{align*}
    \samplemeanof{X} = \frac{1}{n}\sum_{k=1}^{n}\pi_{x}(Z_k),\;\; \samplemeanof{Y} = \frac{1}{n}\sum_{k=1}^{n}\pi_{y}(Z_k).
\end{align*}

\begin{defn}
    An \emph{estimator} is a deterministic function of the sample data.
\end{defn}

What do we mean by \emph{useful estimators}? Suppose $\theta$ is a population parameter, and $\hat{\theta}$ is an estimator for $\theta$. We say that
\begin{itemize}
    \item $\hat{\theta}$ is \emph{unbiased} for $\theta$ if $E(\hat{\theta}) = \theta$,
    \item $\hat{\theta}$ is \emph{consistent} for $\theta$ if for all $\varepsilon > 0$, the limit as $n$ approaches infinity of $P\left(\abs{\hat{\theta} - \theta} > \varepsilon\right)$ is zero.
\end{itemize}
We can then define useful estimators to simply be those that are unbiased and consistent.

\begin{prop}
    Let $Z_1, \ldots, Z_n$ be a finite sample drawn uniformly at random \emph{with replacement} from the entire population. Then for any $k_1, \ldots, k_n \in {1, \ldots, N}$,
    \begin{align*}
        P\left(Z_1=z_{k_1}, \ldots, Z_n=z_{k_n}\right) = \prod_{i=1}^{n}P(Z_{i}=z_{k_i}) = \left(\frac{1}{N}\right)^{n}.
    \end{align*}
    The draws are independent and identically distributed.
\end{prop}

\begin{prop}
    Now consider instead $Z_1, \ldots, Z_n$ drawn uniformly at random \emph{without replacement} from the entire population. Then for any distinct $k_1, \ldots, k_n \in {1, \ldots, N}$, where $n \leq N$,
    \begin{align*}
        P\left(Z_1=z_{k_1}, \ldots, Z_n=z_{k_n}\right) = \frac{(N-n)!}{N!}.
    \end{align*}
    The draws are exchangeable but dependent.
\end{prop}

\begin{rmk}
    We can view draws without replacement from a population with $N$ objects as being deterministic viewing of a specific permutation of the population.
\end{rmk}

\begin{exmp}
    Consider some $f$, and let
    \begin{align*}
        \beta = \frac{1}{N}\sum_{k=1}^{N}f(z_k),
    \end{align*}
    and a corresponding estimator
    \begin{align*}
        \frac{1}{n}\sum_{i=1}^{n}f(Z_i).
    \end{align*}
    Note that this estimator is unbiased by $\mathcal{LOTUS}$ \ref{lotus}.
\end{exmp}

\begin{prop}
    If we are sampling $X_1, \ldots, X_n$ with replacement, then $\varianceof{\samplemeanof{X}} = \frac{\sigma_x^2}{n}$.
\end{prop}

\begin{proof}
    \begin{align*}
        \varianceof{\samplemeanof{X}} &= \varianceof{\frac{1}{n}\sum_{k=1}^{n}\pi_{x}(Z_k)} = \sum_{i=1}^{n}\sum_{j=1}^{n}\frac{1}{n^2}\covarianceof{X_i}{X_j} \\
        &= \sum_{i=1}^{n}\frac{1}{n^2}\covarianceof{X_i}{X_i} + 2\sum_{i=1}^{n}\sum_{j=i+1}^{n}\frac{1}{n^2}\covarianceof{X_i}{X_j} \\
        &= \sum_{i=1}^{n}\frac{1}{n^2}\covarianceof{X_i}{X_i} + 2\sum_{i=1}^{n}\sum_{j=i+1}^{n}\frac{1}{n^2}\left(0\right) \\
        &= \frac{1}{n^2}\sum_{i=1}^{n}\varianceof{X_i} = \frac{1}{n}\varianceof{X_i} = \frac{\sigma_x^2}{n}.
    \end{align*}
\end{proof}

\begin{rmk}
    We have seen that $\samplemeanof{X}$ is unbiased, and it must be consistent by the Weak Law of Large Numbers \ref{wlln}.
\end{rmk}

\begin{prop}
    If we are sampling $X_1, \ldots, X_n$ without replacement, then $\varianceof{\samplemeanof{X}} = \frac{\sigma_x^2}{n} + \frac{\covarianceof{X_1}{X_2}}{n}\left(n-1\right)$.
\end{prop}

\begin{proof}
    \begin{align*}
        \varianceof{\samplemeanof{X}} &= \varianceof{\frac{1}{n}\sum_{k=1}^{n}\pi_{x}(Z_k)} = \sum_{i=1}^{n}\sum_{j=1}^{n}\frac{1}{n^2}\covarianceof{X_i}{X_j} \\
        &= \sum_{i=1}^{n}\frac{1}{n^2}\covarianceof{X_i}{X_i} + 2\sum_{i=1}^{n}\sum_{j=i+1}^{n}\frac{1}{n^2}\covarianceof{X_i}{X_j} \\
        &= \sum_{i=1}^{n}\frac{1}{n^2}\covarianceof{X_i}{X_i} + \frac{\covarianceof{X_1}{X_2}}{n^2}\left(n^2 - n\right) \\
        &= \frac{\sigma_x^2}{n} + \frac{\covarianceof{X_1}{X_2}\left(n-1\right)}{n}.
    \end{align*}
\end{proof}

\begin{cor}
    \begin{align*}
        \covarianceof{X_1}{X_2} = -\frac{\sigma_x^2}{N-1}
    \end{align*}
\end{cor}

\begin{proof}
    We know that when drawing without replacement, if $n = N$ then $\varianceof{\samplemeanof{X}} = 0$. Therefore,
    \begin{align*}
        0 = \frac{\sigma_x^2}{N} + \frac{\covarianceof{X_1}{X_2}\left(N-1\right)}{N}.
    \end{align*}
\end{proof}

\begin{cor}
    \begin{align*}
        \varianceof{\samplemeanof{X}} = \frac{\sigma_x^2}{n}\left(\frac{N-n}{N-1}\right).
    \end{align*}
\end{cor}

\begin{prop}
    If we are sampling $X_1, \ldots, X_n$ without replacement, then $\hat{\sigma}_x^2$ is a biased estimator.
\end{prop}

\begin{proof}
    \begin{align*}
        \hat{\sigma}_x^2 = \frac{1}{n}\sum_{i=1}^{n}\left(X_i - \samplemeanof{X}\right)^{2} = \frac{1}{n}\left[\sum_{i=1}^{n}X_i^2 - 2\sum_{i=1}^{n}X_i\samplemeanof{X} + \sum_{i=1}^{n}\samplemeanof{X}^{2}\right] = \frac{1}{n}\sum_{i=1}^{n}X_i^2 - \samplemeanof{X}^2.
    \end{align*}
    \begin{align*}
        \expectationof{\hat{\sigma}_x^2} &= \expectationof{\frac{1}{n}\sum_{i=1}^{n}X_i^2} - \expectationof{\samplemeanof{X}^2} = \frac{n\left(\sigma_x^2 + \mu_x^2\right))}{n} - \left(\varianceof{\samplemeanof{X}} + \mu_x^2\right) \\
        &= \left(\sigma_x^2 + \mu_x^2\right) - \frac{\sigma_x^2}{n}\left(\frac{N-n}{N-1}\right) - \mu_x^2 = \sigma_x^2 - \frac{\sigma_x^2}{n}\left(\frac{N-n}{N-1}\right) \\
        &= \sigma_x^2\left[\frac{N\left(n-1\right)}{n\left(N-1\right)}\right].
    \end{align*}
    Clearly $\expectationof{\hat{\sigma}_x^2} \neq \sigma_x^2$ when $n \neq N$.
\end{proof}

\section{Dichotomous Populations}

\begin{defn}
    A \emph{dichotomous} random variable is one which has only two possible values, often $0$ and $1$.
\end{defn}

\begin{exmp}
    Consider a population
    \begin{align*}
        \left\{z_1, \ldots, z_k\right\},
    \end{align*}
    where $z_i = \left((x_i, y_i), i\right)$, and $x_i$ is a dichotomous random variables. Note that the second moment is equal to the first moment, because $0^2 = 0$ and $1^2 = 1$. Therefore, $\sigma_x^2 = \mu_x - \mu_x^2 = \mu_x\left(1 - \mu_x\right)$. Then using $\hat{p_x} = \samplemeanof{X}$,
    \begin{align*}
        \varianceof{\hat{p}_x} = \frac{\sigma_x^2}{2} = \frac{p_x\left(1-p_x\right)}{n}.
    \end{align*}
    However, we need to know $p_x$ to calculate this variance, and if we already knew $p_x$ we likely wouldn't be looking at the variance of an estimator of $p_x$. Can we put a bound on $\varianceof{\hat{p}_x}$ without knowing $p_x$? Since $0 \leq p_{x} \leq 1$, we know that $p_{x}\left(1 - p_{x}\right) \leq \frac{1}{4}$, and so
    \begin{align*}
        \varianceof{\hat{p}_x} \leq \frac{1}{4n}.
    \end{align*}

    Now suppose $y_i$ is also dichotomous, and so
    \begin{align*}
        \sigma_{xy} = \frac{1}{N}\sum_{k=1}^{N}\left(x_k - \mu_x\right)\left(y_k - \mu_y\right) = \frac{1}{N}\sum_{k=1}^{N}x_ky_k - \mu_x\mu_y.
    \end{align*}
    Note that if $\covarianceof{X_1}{Y_1} = 0$, since $\covarianceof{X_1}{Y_1} = \sigma_{xy}$, we have
    \begin{align*}
        \frac{1}{N}\sum_{k=1}^{N}x_ky_k = \mu_x\mu_y,
    \end{align*}
    which is equivalent to $P(X_1 = 1, Y_1 = 1) = P(X_1 = 1)P(Y_1 = 1)$, and so zero covaraince between $X_1$ and $Y_1$ implies they must be independent when they are Bernoulli random variables.
\end{exmp}

\section{Approximation Methods}

Suppose $X: \Omega \to \R$ is a random variable with mean $\mu$ and variance $\sigma^2$, and $g: \R \to \R$ is a deterministic, $C^2$ function. What can we say about $g(X)$?

Let $f$ be the density of $X$, then by $\mathcal{LOTUS}$ \ref{lotus} we have
\begin{align*}
    \expectationof{g(X)} = \int_{\R}g(x)f(x)dx.
\end{align*}
Suppose we know that $X$ is very likely ``close to'' $\mu$, that is, that $\sigma^2$ is small. Consider the Taylor expansion of $g$ about $\mu$. We know that for some $\xi$ between $x$ and $\mu$, we have
\begin{align*}
    g(x) = g(\mu) + g'(\mu)(x-\mu) + \frac{g''(\xi)}{2}(x-\mu)^2
\end{align*}
If $x$ is near $\mu$, then we can use $\xi \approx \mu$. If we can guarantee certain conditions to be explored later, we can guarantee that
\begin{align*}
    \expectationof{g(X)} &\approx \expectationof{g(\mu) + g'(\mu)(x-\mu) + \frac{g''(\mu)}{2}(x-\mu)^2} \\
    &= g(\mu) + g'(\mu)\expectationof{X - \mu} + \frac{g''(\mu)}{2}\expectationof{\left(X - u\right)^2} = g(\mu) + \frac{g''(\mu)}{2}\sigma^2.
\end{align*}

Now let's expand our focus to random variables $X$ and $Y$, and a deterministic function $h: \R \times \R \to \R$. We know that
\begin{align*}
    h(x, y) &\approx h(\mu_x, \mu_y) + \frac{\partial h}{\partial x}\left(\mu_x, \mu_y\right)(x - \mu_x) + \frac{\partial h}{\partial y}\left(\mu_x, \mu_y\right)(y - \mu_y) \\
    &+ \frac{\partial^2h}{2\partial x^2}(\mu_x, \mu_y)\left(x - \mu_x\right)^2 + \frac{\partial^2h}{2\partial y^2}(\mu_x, \mu_y)\left(y - \mu_y\right)^2 + \frac{\partial^2h}{\partial x\partial y}(\mu_x, \mu_y)\left(x - \mu_x\right)\left(y - \mu_y\right).
\end{align*}
Note that
\begin{align*}
    \expectationof{h(x, y)} \approx \expectationof{h(\mu_x, \mu_y)}
\end{align*}

\begin{thm}
    Let $\theta$ be a random variable, and $\hat{\theta}$ be an estimator for $\theta$. Then the \emph{mean squared error} of $\hat{\theta}$ is
    \begin{align*}
        \expectationof{\left(\hat{\theta} - \theta\right)^2} = \varianceof{\hat{\theta}} + \biasof{\hat{\theta}}^2.
    \end{align*}
\end{thm}

\begin{proof}
    \begin{align*}
        \expectationof{\left(\hat{\theta} - \theta\right)^2} &= \expectationof{\left(\hat{\theta} - \expectationof{\hat{\theta}} + \expectationof{\hat{\theta}} - \theta\right)^2} \\
        &= \expectationof{\left(\hat{\theta} - \expectationof{\hat{\theta}}\right)^2 + 2\left(\hat{\theta} - \expectationof{\hat{\theta}}\right)\left(\expectationof{\hat{\theta}} - \theta\right) + \left(\expectationof{\hat{\theta}} - \theta\right)^2} \\
        &= \expectationof{\left(\hat{\theta} - \expectationof{\hat{\theta}}\right)^2} + 2\expectationof{\hat{\theta} - \expectationof{\hat{\theta}}}\left(\expectationof{\hat{\theta}} - \theta\right) + \expectationof{\left(\expectationof{\hat{\theta}} - \theta\right)^2}.
    \end{align*}
    Since $\expectationof{\hat{\theta} - \expectationof{\hat{\theta}}} = \expectationof{\hat{\theta}} - \expectationof{\hat{\theta}} = 0$, we have
    \begin{align*}
        \expectationof{\left(\hat{\theta} - \theta\right)^2} = \expectationof{\left(\hat{\theta} - \expectationof{\hat{\theta}}\right)^2} + \left(\expectationof{\hat{\theta}} - \theta\right)^2.
    \end{align*}
\end{proof}

\begin{exmp}
    Consider a bivariate population $z_i = ((x_i, y_i), i)$ for $1 \leq i \leq N$. We take a sample $Z_1, \ldots, Z_n$ (without replacement) from this population, and compute the sample means $\samplemeanof{X}$ and $\samplemeanof{Y}$. Suppose we want to estimate $\mu_y$, and happen to know $\mu_x$, where $\mu_x \neq 0$. Let
    \begin{align*}
        r = \frac{\mu_y}{\mu_y},
    \end{align*}
    which we can estimate as
    \begin{align*}
        R = \frac{\samplemeanof{Y}}{\samplemeanof{X}}.
    \end{align*}
    Then we can choose
    \begin{align*}
        \samplemeanof{Y}_R = \mu_xR = \mu_x\frac{\samplemeanof{Y}}{\samplemeanof{X}}.
    \end{align*}
    Note that $\samplemeanof{Y}_R$ is not $\samplemeanof{Y}$. Let us use our approximation method to obtain approximations of $\expectationof{R}$ and $\varianceof{R}$. We use $R = g(\samplemeanof{X}, \samplemeanof{Y})$, where $g(x, y) = y/x$. Then we have
    \begin{align*}
        g(x, y) &\approx g(\mu_x, \mu_y) + \frac{\partial g}{\partial x}(x - \mu_x) + \frac{\partial g}{\partial y}(y - \mu_y) \\
        &+ \frac{\partial^2g}{2\partial x^2}(x - \mu_x)^2 + \frac{\partial^2g}{2\partial y^2}(y - \mu_y)^2 + \frac{\partial^2g}{\partial xy}(x - \mu_x)(y - \mu_y) \\
        &= g(\mu_x, \mu_y) + \frac{1}{2}\frac{2\mu_y}{\mu_x^3}(x-\mu_x)^2 + \frac{1}{2}(0)(y-\mu_y)^2 - \frac{1}{\mu_x^2}(x-\mu_x)(y-\mu_y).
    \end{align*}
    Therefore,
    \begin{align*}
        \expectationof{R} &\approx g(\mu_x, \mu_y) + \frac{1}{2}\frac{2\mu_y}{\mu_x^2}\varianceof{\samplemeanof{X}} - \frac{1}{\mu_x^2}\covarianceof{\samplemeanof{X}}{\samplemeanof{Y}} \\
        &= \frac{\mu_y}{\mu_x} + \frac{\mu_y}{\mu_x^3}\frac{\sigma_x^2}{n}\left[\frac{N-n}{N-1}\right] - \frac{1}{\mu_x^2}\left[\frac{\sigma_{xy}}{n}\left(\frac{N-n}{N-1}\right)\right].
    \end{align*}

    For the variance, we instead use a first-order Taylor approximation. Note that, in general for
    \begin{align*}
        g(x,y) = c_1 + c_2(x - \mu_x) + c_2(y - \mu_y),
    \end{align*}
    the variance of $g$ is given by
    \begin{align*}
        \varianceof{c_1 + c_2(x - \mu_x) + c_2(y - \mu_y)} = c_2^2\varianceof{x} + c_3^2\varianceof{y}  + 2c_2c_3\covarianceof{x}{y}.
    \end{align*}
    Therefore,
    \begin{align*}
        \varianceof{R} &\approx \left(\frac{-\mu_y}{\mu_x^2}\right)^2\varianceof{\overline{X}} + \left(\frac{1}{\mu_x}\right)^2\varianceof{\overline{Y}} - 2\left(\frac{-\mu_y}{\mu_x^3}\right)\covarianceof{\overline{X}}{\overline{Y}} \\
        &= \left(\frac{-\mu_y}{\mu_x^2}\right)^2\frac{\sigma_{x}^2}{n}\left[\frac{N-n}{N-1}\right] + \left(\frac{1}{\mu_x}\right)^2\frac{\sigma_{y}^2}{n}\left[\frac{N-n}{N-1}\right] - 2\frac{\mu_y}{\mu_x^3}\frac{\sigma_{xy}}{n}\left[\frac{N-n}{N-1}\right].
    \end{align*}
\end{exmp}

\section{Sigma Algebras}

\begin{defn}
    A \emph{$\sigma$-algebra} is a set $X$ of sets such that $X$ contains $\emptyset$, and $X$ is closed under complements and countable unions.
\end{defn}

\begin{defn}
    The \emph{Borel $\sigma$-algebra} is the $\sigma$-algebra consisting of the closure of all opens subsets of $\R$.
\end{defn}

\section{Convergence of Random Variables}

\begin{defn}
    Let $(\Omega, \mathcal{F}, P)$ be a probability space. Let $\{X_n : n \in \N\}$ be a sequence of random variables defined on $\Omega$, and let $X: \Omega \to \R$ be a random variable.

    We say that $X_n$ converges \emph{pointwise everywhere} to $X$ if for all $\omega \in \Omega$, and for all $\varepsilon > 0$, there exists $N \in \N$ such that $d\left(X_n, X(\omega)\right) < \varepsilon$.

    We say that $X_n$ converges \emph{uniformly} to $X$ if for all $\varepsilon > 0$, there exists $N \in \N$ such that for all 
    $\omega \in \Omega$ we have $d\left(X_n, X(\omega)\right) < \varepsilon$.

    We say that $X_n$ converges \emph{pointwise almost surely} or \emph{pointwise almost everywhere} to $X$ if there exists $
    A \in \mathcal{F}$ where $P(A) = 0$ such that $X_n$ converges pointwise to $X$ on $A^{c}$.
\end{defn}

\begin{defn}
    We say that $X_n$ converges to $X$ \emph{in probability}, denoted by
    \begin{align*}
        X_n \to X
    \end{align*}
    if for all $\delta > 0$, the collection of sets
    \begin{align*}
        A_{n,\delta} = \left\{\omega \in \Omega \compbar d\left(X_n(\omega), X(\omega)\right) > \delta\right\}
    \end{align*}
    satisfies $P(A_{n,\delta}) \to 0$ as $n \to \infty$.
\end{defn}

\begin{defn}
    We say that $X_n$ converges to $X$ using $L^{p}$ convergence when
    \begin{align*}
        \expectationof{\abs{X_n(\omega) - X(\omega)}^{p}} \to 0
    \end{align*}
    as $n \to \infty$.
\end{defn}

\begin{defn}
    We say that $t$ is a \emph{continuity point} of $F$ if $F$ is continuous at $t$.
\end{defn}

\begin{defn}
    Let $X_n$ be a sequence of random variables, not necessarily defined on the same probability same. Let $X$ be a random variable with cumulative distribution function $F$. We say that $X_n$ converges to $X$ \emph{in distribution} if for every continuity point $t$ of $F$, the limit of $F_n(t)$ as $n \to \infty$ is $F(t)$.
\end{defn}

\begin{exmp}
    Let $\Omega = [0, 1]$, $\mathcal{F} = \mathcal{B}(\Omega)$, and $P$ be the uniform probability measure. Let $X_n(\omega)$ be an indicator of whether $\omega$ is contained in $[0, 1/n]$ or not. This sequence has pointwise convergence almost surely to $0$, but it is not pointwise everywhere nor uniform.
\end{exmp}

\begin{exmp}
    Let $\Omega = [0, 1]$, $\mathcal{F} = \mathcal{B}(\Omega)$, and $P$ be the uniform probability measure. Let $X_n(\omega)$ be equal to $n^2$ when $0 \leq \omega \leq 1/n$ and zero otherwise. This sequence has pointwise convergence almost surely to $0$, but it is not pointwise everywhere nor uniform.
\end{exmp}

\begin{thm}{Slutsky's Theorem}\label{slutsky}\proofbreak
    Let $X_n$ and $Y_n$ be sequences of random variables such that $X_n$ converges in distribution to a random variable $X$ and $Y_n$ converges in probability to a real constant $c$. Suppose $X_n$ and $y_n$ are defined on the same probability space, so that $X_n + Y_n$ and $X_n \cdot Y_n$ are well-defined. Then we have
    \begin{itemize}
        \item $X_n + Y_n$ converges in distribution to $X + c$,
        \item $X_nY_n$ converges in distribution to $X \cdot c$,
        \item if $c \neq 0$ then $X_n / Y_n$ converges in distribution to $X/c$.
    \end{itemize}
\end{thm}

\begin{lemma}
    Let $X_n$ be a sequence of random variables defined on a probability space $\Omega$. If $X_n$ converges in distribution to a real constant $c$, then $X_n$ converges in probability to $c$.
\end{lemma}

\begin{defn}
    The \emph{delta-method} takes an independent and identically distributed sequence of random variables $X_1, X_2, \ldots$, with $\expectationof{X_i = \mu}$ and $\varianceof{X_i} = \sigma^2$. Let $g: \R \to \R$ be a $C^2$ function, then
    \begin{align}
        g\left(\overline{X}\right)
    \end{align}
\end{defn}

\section{Paramnetric Estimation}

\subsection{Method of Moments}

\begin{defn}
    Consider a collection of independent and identically distributed random variables $X_1, \ldots, X_n$ with a common distribution function $F_{\theta}$, where $\theta \in \R^k$ is a non-random fixed but unknown parameter. We say that the set of all $X$ with distribution $F_{\theta}$ for some $\theta \in \R^k$ forms a \emph{parametric family}.
\end{defn}

\begin{rmk}
    If $\theta \in \R^{k}$ is known, then we know $F_{\theta}$.
\end{rmk}

\begin{exmp}
    Consider $X_1, \ldots, X_n, \ldots$ independent and identically distributed Poisson($\lambda$) random variables, where $\lambda > 0$. We have parameter space $\Theta = \{\lambda \in \R: \lambda > 0\}$, and
    \begin{align*}
        P(X_i = \ell) = \frac{e^{-\lambda}\lambda^{\ell}}{\ell!},\;\;\ell \geq 0.
    \end{align*}
\end{exmp}

\begin{exmp}
    Consider $X_1, \ldots, X_n, \ldots$ independent and identically distributed uniform random variables on $[0, \theta]$, where $\theta > 0$. Since the support of the distribution function is $[0, \frac{1}{\theta}]$, which is probabilistically equivalent to $(0, \frac{1}{\theta})$. Note that the support then depends on $\theta$.
\end{exmp}

\begin{defn}
    In the \emph{Method of Moments}, we find expressions for the moments of a distribution in terms of the parameters of the distribution, and then invert these functions to estimate the parameters in terms of the \emph{sample moments}.
\end{defn}

\begin{exmp}
    Consider $X_1, \ldots, X_n, \ldots$ independent and identically distributed exponential($\lambda$) random variables. Notice that
    \begin{align*}
        \expectationof{X_i} = \int_{0}^{\infty}x\lambda e^{-\lambda x}dx = \frac{1}{\lambda},
    \end{align*}
    and so $\lambda = 1/\mu$. We can therefore estimate
    \begin{align*}
        \hat{\lambda}_{\textrm{MoM}} = \frac{1}{\samplemeanof{X}}.
    \end{align*}

    Since $g(x) = 1/x$ is smooth when $x \neq 0$, and the Weak Law of Large Numbers \ref{wlln} guarantees $\samplemeanof{X}$ converges to $\mu$ in probability, it follows that $1/\samplemeanof{X}$ converges to $1/\mu = \lambda$ in probability, and so this estimator is \emph{consistent}.
\end{exmp}

\begin{exmp}
    Consider $X_1, \ldots, X_n, \ldots$ independent and identically distributed normal variables with distribution $N(\mu, \sigma^2)$, so $\theta = (\mu, \sigma^2)$. We find that
    \begin{align*}
        \mu &= \mu_1 \\
        \sigma^2 &= \mu_2 - \mu_1^2.
    \end{align*}
    In follows that we can estimate
    \begin{align*}
        \hat{\mu}_{\textrm{MoM}} &= \samplemeanof{X} \\
        \hat{\sigma^2}_{\textrm{MoM}} &= \varianceof{x} - \samplemeanof{X}^2.
    \end{align*}
\end{exmp}

\subsection{Maximum Likelihood Estimators}

Consider the joint density $f(x_1,\ldots,x_n|\theta)$ of $X_1, \ldots, X_n$. If we can maximize this value over $\theta$, we have an estimate of $\theta$ which has the ``maximum likelihood'' of us having observed the sample data we have.

\begin{defn}
    Let $X_1, \ldots, X_n, \ldots$ random variables drawn from a common distribution with parameter vector $\theta \in \R^k$. Let
    \begin{align*}
        \hat{\theta}(x_1, \ldots, x_n) = \argmax_{\theta}f(x_1,\ldots,x_n|\theta).
    \end{align*}
    We define the \emph{maximum likelihood estimator} of $\theta$ as
    \begin{align*}
        \theta(X_1, \ldots, X_n).
    \end{align*}
\end{defn}

\begin{prop}
    When $X_1, \ldots, X_n$ are independent and identically distributed,
    \begin{align*}
        \hat{\theta}(x_1, \ldots, x_n) = \argmax_{\theta}\sum_{i=1}^{n}\ln f(x_i|\theta).
    \end{align*}
\end{prop}

\begin{proof}
    Since $X_1, \ldots, X_n$ are independent and identically distributed, it follows that
    \begin{align*}
        f(x_1, \ldots, x_n | \theta) = \prod_{i=1}^{n}f(x_i,\theta).
    \end{align*}
    Since $\ln(x)$ is monotonic, the argument at the maximum of $f(x_1, \ldots, x_n|\theta)$ is the argument at the maximum of $\ln f(x_1, \ldots, x_n|\theta)$. Therefore,
    \begin{align*}
        \hat{\theta}(x_1, \ldots, x_n) = \argmax_{\theta}\ln\left[\prod_{i=1}^{n}f(x_i,\theta)\right] = \argmax_{\theta}\sum_{i=1}^{n}\ln f(x_i|\theta).
    \end{align*}
\end{proof}

\begin{exmp}
    Consider $X_1, \ldots, X_n, \ldots$ independent and identically distributed exponential($\lambda$) random variables. Then
    \begin{align*}
        f(x_1, \ldots, x_n|\lambda) = \lambda^n\exp\left(-\lambda\sum_{i=1}^{n}x_i\right),
    \end{align*}
    so
    \begin{align*}
        \ln f(x_1, \ldots, x_n|\lambda) = n\ln(\lambda) - \lambda\sum_{i=1}^{n}x_i.
    \end{align*}
\end{exmp}

\begin{exmp}
    Consider $X_1, \ldots, X_n, \ldots$ independent and identically distributed normal variables with distribution $N(\mu, \sigma^2)$, so $\theta = (\mu, \sigma^2)$. We find that
    \begin{align*}
        f\left(x_1, \ldots, x_n|\theta\right) = \left[\frac{1}{\sigma\sqrt{2}\pi}\right]^{n}\exp\left[-\frac{1}{2\sigma^2}\sum_{i=1}^{n}\left(x_i-\mu\right)^2\right],
    \end{align*}
    and so
    \begin{align*}
        \ln f\left(x_1, \ldots, x_n|\theta\right) = -n\ln(\sigma\sqrt{2}\pi) -\frac{1}{2\sigma^2}\sum_{i=1}^{n}\left(x_i-\mu\right)^2.
    \end{align*}
    Then we have
    \begin{align*}
        \begin{pmatrix}
            \frac{\partial \ln f\left(x_1, \ldots, x_n|\theta\right)}{\partial \mu} \\
            \frac{\partial \ln f\left(x_1, \ldots, x_n|\theta\right)}{\partial \sigma}
        \end{pmatrix} = \begin{pmatrix}
            \frac{1}{\sigma^2}\sum_{i=1}^{n}\left(x_i-\mu\right) \\
            \frac{-n}{\sigma} + \frac{1}{\sigma^3}\sum_{i=1}^{n}\left(x_i-\mu\right)^2.
        \end{pmatrix}
    \end{align*}
    Since our estimator is smooth, its maximum must occur when this derivative is zero. Therefore, we find
    \begin{align*}
        \sum_{i=1}^{n}(x_i - \mu) &= 0 \implies \hat{\mu}_{\textrm{MLE}} = \samplemeanof{X} \\
        \frac{-n}{\sigma} + \frac{1}{\sigma^3}\sum_{i=1}^{n}\left(x_i-\mu\right)^2 &= 0 \implies \hat{\sigma^2}_{\textrm{MLE}} = \frac{1}{n}\sum_{i=1}^{n}\left(X_i-\samplemeanof{X}\right)^2.
    \end{align*}
\end{exmp}

\begin{exmp}
    Consider $X_1, \ldots, X_n, \ldots$ independent and identically distributed uniform variables on $[0, \theta]$. Then
    \begin{align*}
        f(x_1, \ldots, x_n|\theta) = \left(\frac{1}{\theta}\right)^{n}1_{[0, \theta]}.
    \end{align*}
    We can see that $f(x_1, \ldots, x_n)$ is maximized when $\theta$ is as small as possible, but at least as large as any $X_i$ so that all indicators are one. It follows that
    \begin{align*}
        \hat{\theta}_{\textrm{MLE}} = \max_{i}(X_i).
    \end{align*}
    Contrast this with the Method of Moments estimator for $\theta$, which is
    \begin{align*}
        \hat{\theta}_{\textrm{MoM}} = 2\samplemeanof{X}.
    \end{align*}
\end{exmp}

\subsection{Fisher Information}

\begin{prop}
    Let $X \distributed f(x|\theta)$ be a random variable from a distribution with parameter $\theta \in \R^k$, and let
    \begin{align*}
        Y = \frac{\partial}{\partial \theta}\ln\left(f(X|\theta)\right).
    \end{align*}
    Under sufficient regularity conditions, $\expectationof{Y} = 0$.
\end{prop}

\begin{proof}
    Since $f(x|\theta)$ is a probability density, we know that
    \begin{align*}
        \int_{\Omega}f(x|\theta)d\theta = 1.
    \end{align*}
    It follows that
    \begin{align*}
        \frac{\partial}{\partial \theta}\left[\int_{\Omega}f(x|\theta)d\theta\right] = 0.
    \end{align*}
\end{proof}

\begin{thm}
    Cram\'er-Rao bound
\end{thm}

\begin{thm}
    Let $X_1, \ldots, X_n$ be independent and identically distributed random variables with common density $f(X|\theta)$, and let $\hat{\theta}$ be the maximum likelihood estimator for $\theta$. Then
    \begin{align*}
        \sqrt{nI(\theta)}\left[\hat{\theta} - \theta\right]
    \end{align*}
    converges in distribution to the standard normal distribution $N(0, 1)$.
\end{thm}

\begin{defn}
    Suppose we want to compare two estimators, $\hat{\theta}_1$ and $\hat{\theta}_2$. One possible metric is the mean squared error of each estimator. We call the ratio
    \begin{align*}
        \frac{\meansqerrof{\hat{\theta}_1}}{\meansqerrof{\hat{\theta}_2}}
    \end{align*}
    the \emph{relative efficiency} of $\hat{\theta_1}$ and $\hat{\theta}_2$. In the case that both estimators are unbiased, then since $\textrm{MSE}(\hat{\theta}) = \varianceof{\hat{\theta}} + \textrm{bias}(\hat{\theta})^2$ it follows that their relative efficiency is simply
    \begin{align*}
        \frac{\varianceof{\hat{\theta}_1}}{\varianceof{\hat{\theta}_2}}.
    \end{align*}
\end{defn}

\begin{thm}\label{sufficiency-characterization}
    Let $X_1, \ldots, X_n$ be independent and identically distributed random variables with common density $f(X|\theta)$ and let $T$ be an estimator. Then $T$ is sufficient for $\theta$ if and only if there exists $h(x_1, \ldots, x_n)$ such that
    \begin{align*}
        f(x_1, \ldots, x_n|\theta) = g(T, \theta)h(x_1, \ldots, x_n).
    \end{align*}
\end{thm}

\begin{exmp}
    Suppose we want to show that $T = \sum X_i$ is a sufficient statistics for $\lambda$ if $X_i \distributed \textrm{exp}(\lambda)$ are independent and identically distributed. We know that
    \begin{align*}
        f(x_1, \ldots, x_n|\theta) = \lambda^n\exp\left(-\lambda\sum_{i=1}^{n}x_i\right) = \lambda^n\exp\left(-\lambda T\right).
    \end{align*}
    Let
    \begin{align*}
        g(T,\lambda) = \lambda^n\exp\left(-\lambda T\right),
    \end{align*}
    by then Theorem \ref{sufficiency-characterization} we know that $T$ is sufficient for $\lambda$.
\end{exmp}

\begin{defn}
    Minimally sufficient statistic.
\end{defn}

\section{Bayesian Inference}

\begin{rmk}
    In the frequentist paradigm we consider a parametric model with parameter $\theta$, and attempt to estimate the true and fixed value of $\theta$ from our data.

    In the Bayesian paradigm, we instead view $\theta$ itself as a random variable, rather than as a fixed but unknown value. We view $\theta$ as having a \emph{prior} distributio (typically known), which may have been constructed to include information about the experimental process. If the prior distribution has (typically known) parameters, these are called \emph{hyperparameters}. We start from the prior distribution $f_{\theta}(\theta)$ and a likelihood function $f(x|\theta)$, and from this we compute a \emph{posterior distribution} $f_{\theta|X}(\theta|X)$.
\end{rmk}

\begin{defn}{Bayes' Rule}\proofbreak
    \begin{align*}
        P\left(\theta = \theta_0|X\right) = \frac{P\left(X|\theta_0\right)P(\theta_0)}{P\left(X|\theta_0\right)P(\theta_0) + P\left(X|\theta_1\right)P(\theta_1)}
    \end{align*}
\end{defn}

\begin{exmp}
    Suppose we wish to understand some independent and identically distributed Bernoulli($\theta$) data, where $0 \leq \theta \leq 1$. Suppose $\theta$ has the following \emph{discrete prior}
    \begin{align*}
        P\left(\theta = \frac{3}{4}\right) &= \frac{1}{3}, \\
        P\left(\theta = \frac{1}{4}\right) &= \frac{2}{3}.
    \end{align*}
    Next, suppose we observe data $X_1 = 1$, $X_2 = 1$, $X_3 = 0$, $X_4 = 1$, $X_5 = 1$. Then by Bayes' rule we have
    \begin{align*}
        P\left(\theta = 3/4|X\right) &= \frac{P(X|\theta=3/4)P(\theta=3/4)}{P(X|\theta=1/4)P(\theta=1/4) + P(X|\theta=3/4)P(\theta=3/4)} \\
        &= \frac{\left(3/4\right)^{4}(1-3/4)^{1}(1/3)}{\left(3/4\right)^{4}(1-3/4)^{1}(1/3) + \left(1/4\right)^{4}(1-1/4)^{1}(2/3)} \\
        &= \frac{3^3/4^5}{3^3/4^5 + 2/4^5} = \frac{27}{29}.
    \end{align*}
    It follows that $P(\theta = 1/4|X) = 1-P(\theta=3/4|X) = \frac{2}{29}$.
\end{exmp}

\begin{exmp}
    Suppose $X_1, \ldots, X_n$ are independent and identically distributed normal $N(\mu, \sigma^2)$ random variables, where $\mu$ is unknown while $\sigma^2$ is known.
\end{exmp}
