\documentclass[12pt]{article}

\usepackage{amsmath}
\usepackage{amssymb}
\usepackage{amsthm}
\usepackage{centernot}
\usepackage{fullpage}
\usepackage{makecell}
\usepackage{tabularx}
\usepackage[hypcap=false]{caption}
\usepackage{tikz, tkz-euclide}
\usetikzlibrary{decorations.pathreplacing,arrows}
\usetikzlibrary{quotes,angles,calc,intersections}

\usepackage{titling}
\usepackage{pdfpages}
\usepackage{enumitem}
\usepackage{multicol}
\usepackage{bm}

\usepackage{linear}
\usepackage{common}

\begin{document}

\title{Ordinary Differential Equations}
\author{Brendan Burkhart}
\maketitle

\tableofcontents
\newpage

\section{Classification}

\begin{defn}
    A \emph{differential equation} is an equation relating one or more independent variables, functions of those variables, and derivatives of those functions.
\end{defn}

\begin{exmp}\label{first-order-ode}
    \[\frac{\mathrm{d}x}{\mathrm{d}t} = x\]
\end{exmp}

\begin{defn}
    A \emph{solution} to a differential equation is an expression for the dependent functions of the differential equation in terms of the independent variables.
\end{defn}

Differential equations can be broadly classified into \emph{ordinary} and \emph{partial} differential equations.

\begin{defn}
    An \emph{ordinary} differential equation, or \emph{ODE} is a differential equation involving a single independent variable, and all derivatives are with respect to this variable.
\end{defn}

\begin{defn}
    A \emph{partial} differential equation, or \emph{PDE} is a differential equation involving functions of more than one independent variables, and derivatives may be with respect to any of those variables.
\end{defn}

\begin{defn}
    The \emph{general form} of an ODE is
    \[F(x, y, y', \ldots, y^{(n)}) = 0,\] that is some expression involving the independent variable $x$, the dependent variable $y$, and the derivatives of $y$.
\end{defn}

\begin{defn}
    The \emph{order} of a differential equation is the order of the highest derivative.
\end{defn}

\begin{exmp}
    \[\frac{\mathrm{d}^2\theta}{\mathrm{d}t^2} + \frac{g}{L}\sin\theta = 0\]
    This is a second order differential equation, while Example \ref{first-order-ode} is first order.
\end{exmp}

\begin{defn}
    A differential equation of a dependent variable $y$ and its derivatives is said to be \emph{linear} if it is an affine map with regard to $y$ and its derivatives.
\end{defn}

\begin{exmp}
    $\sin(x)y' + 2x^2y'' = x$ is linear.
\end{exmp}

\begin{exmp}
    $yy' + y'' = 0$ is non-linear.
\end{exmp}

\begin{defn}
    An \emph{autonomous} differential equation is one in which the independent variable does not explicitly appear. When the independent variable represents time in some way, these may also be called \emph{time-invariant}.
\end{defn}

\begin{exmp}
    $\frac{\mathrm{d}y}{\mathrm{d}x} = 5y - 20$ is an autonomous differential equation as the independent variable $x$ is not explicitly present.
\end{exmp}

\section{First order differential equations}

\begin{defn}
    The \emph{standard form} of a linear first order differential equation (with independent variable $t$ and dependent variable $y$) is \[\frac{\mathrm{d}y}{\mathrm{d}t} + p(t)y = g(t),\] for some arbitrary functions $p(t), g(t)$.
\end{defn}

\subsection{Autonomous equations}

Any linear first order autonomous ODE with independent variable $x$ and dependent variable $y$ can trivially be put into the form \[\frac{\mathrm{d}y}{\mathrm{d}x} = ay - b.\] This form can always be solved.

\begin{align*}
    \frac{\mathrm{d}y}{\mathrm{d}x} &= ay - b \\
    &= a(y - \frac{b}{a})
\end{align*}

Note that $\frac{\mathrm{d}}{\mathrm{d}x}{y - \frac{b}{a}} = \frac{\mathrm{d}}{\mathrm{d}x}y$. If $y \neq \frac{b}{a}$, it follows that $\ln(y - \frac{b}{a}) = ax + C$ for some constant of integration $C$, so $\abs{y - \frac{b}{a}} = ke^{ax}$ for some $k$. If $y > \frac{b}{a}$, then $y = \frac{b}{a} + ke^{ax}$, and similarly if $y < \frac{b}{a}$, then $y = \frac{b}{a} + ke^{ax}$ for some $k$. In the case that $y = \frac{b}{a}$, it follows that $\frac{\mathrm{d}y}{\mathrm{d}x} = 0$, so $y = \frac{b}{a} + ke^{ax}$ where $k = 0$.

Therefore, $y = \frac{b}{a} + ke^{ax}$ is a solution to any linear first order autonomous differential equation.

\subsection{Integrating factors}

Linear first order ODEs which are not autonomous can sometimes be solved through the use of an \emph{integrating factor}. We can write a linear first order ODE as \[y' + p(t)y = g(t),\] where $p(t), g(t)$ are arbitrary functions. Our goal here is to find a function $\mu(t)$, called the \emph{integrating factor}, such that
\[\frac{\mathrm{d}}{\mathrm{d}t}\mu(t)y = \mu(t)y' + p(t)\mu(t)y.\] That is, we want $\mu'(t) = p(t)\mu(t)$. Once such $\mu(t)$ is found, since \[\mu(t)y' + \mu(t)p(t)y = \mu(t)g(t),\] we clearly have \[\frac{\mathrm{d}}{\mathrm{d}t}\mu(t)y = \mu(t)g(t).\] It follows that \[y = \frac{1}{\mu(t)}\int_{t_0}^t \mu(s)g(s)ds + C\] for some constants $t_0$ and $C$.

We of course need to also be able to find such a $\mu(t)$. Note that since $\mu'(t) = p(t)\mu(t)$ we have \[\frac{\mu'(t)}{\mu(t)} = p(t).\] Since $\frac{\mathrm{d}}{\mathrm{d}t}\ln(\mu(t)) = \frac{\mu'(t)}{\mu(t)}$, it follows that $\ln(\mu(t)) = \int p(t)$. Therefore, $\mu(t) = e^{\int p(t)dt}$.

\begin{exmp}
    Using an integrating factor to solve $ty' + 2y = 4t^2$ with the initial condition $y(1) = 2$.

    First, we need the differential equation in the standard form $y' + p(t)y = g(t)$. Doing this, we obtain $y' + \frac{2}{t}y = 4t$, provided that $t \neq 0$. Then we have $\ln(\mu(t)) = \int \frac{2}{t}dt = 2\ln(t) = \ln(t^2)$, so $\mu(t) = t^2$. We know have $\frac{\mathrm{d}}{\mathrm{d}t}\mu(t)y = 4t^3$, which we may integrate to obtain $t^2y = t^4 + c$, and so $y = t^2 + \frac{c}{t^2}$. Applying the initial condition $y(1) = 2$, we have $2 = 1 + c$, so $c = 1$ and our final particular solution is \[y = t^2 + \frac{1}{t^2}, t > 0.\]
\end{exmp}

\subsection{Separable equations}

A separable first order differential equation is of the form
\[\frac{\mathrm{d}y}{\mathrm{d}t} = g(t)h(y),\] so that we may ``separate'' the independent and dependent variables to obtain
\[\frac{1}{h(y)}\frac{\mathrm{d}y}{\mathrm{d}t} = g(t).\] This can be often be integrated, with the particularly nice property that due to the chain rule, we can simply integrate the LHS with respect to $y$.
\[\int \frac{1}{h(y)}dy = \int g(t)dt.\] These integrals may or may not be computable, and even if they are the resulting equation may not be solvable for $y$. However, it can sometimes be done and is a simple and straightforward technique.

\end{document}
