\documentclass[12pt]{article}

\usepackage{amsmath}
\usepackage{amssymb}
\usepackage{amsthm}
\usepackage{centernot}
\usepackage{fullpage}
\usepackage{makecell}
\usepackage{tabularx}
\usepackage[hypcap=false]{caption}
\usepackage{tikz, tkz-euclide}
\usetikzlibrary{decorations.pathreplacing,arrows}
\usetikzlibrary{quotes,angles,calc,intersections}

\usepackage{titling}
\usepackage{pdfpages}
\usepackage{enumitem}
\usepackage{multicol}
\usepackage{bm}

\usepackage{linear}
\usepackage{common}

\begin{document}

\title{Ordinary Differential Equations}
\author{Brendan Burkhart}
\maketitle

\tableofcontents
\newpage

\section{Classification}

\begin{defn}
    A \emph{differential equation} is an equation relating one or more independent variables, functions of those variables, and derivatives of those functions.
\end{defn}

\begin{exmp}\label{first-order-ode}
    \[\frac{\mathrm{d}x}{\mathrm{d}t} = x\]
\end{exmp}

\begin{defn}
    A \emph{solution} to a differential equation is an expression for the dependent functions of the differential equation in terms of the independent variables.
\end{defn}

Differential equations can be broadly classified into \emph{ordinary} and \emph{partial} differential equations.

\begin{defn}
    An \emph{ordinary} differential equation, or \emph{ODE} is a differential equation involving a single independent variable, and all derivatives are with respect to this variable.
\end{defn}

\begin{defn}
    A \emph{partial} differential equation, or \emph{PDE} is a differential equation involving functions of more than one independent variables, and derivatives may be with respect to any of those variables.
\end{defn}

\begin{defn}
    The \emph{general form} of an ODE is
    \[F(x, y, y', \ldots, y^{(n)}) = 0,\] that is some expression involving the independent variable $x$, the dependent variable $y$, and the derivatives of $y$.
\end{defn}

\begin{defn}
    The \emph{order} of a differential equation is the order of the highest derivative.
\end{defn}

\begin{exmp}
    \[\frac{\mathrm{d}^2\theta}{\mathrm{d}t^2} + \frac{g}{L}\sin\theta = 0\]
    This is a second order differential equation, while Example \ref{first-order-ode} is first order.
\end{exmp}

\begin{defn}
    A differential equation of a dependent variable $y$ and its derivatives is said to be \emph{linear} if it is an affine map with regard to $y$ and its derivatives.
\end{defn}

\begin{exmp}
    $\sin(x)y' + 2x^2y'' = x$ is linear.
\end{exmp}

\begin{exmp}
    $yy' + y'' = 0$ is non-linear.
\end{exmp}

\begin{defn}
    An \emph{autonomous} differential equation is one in which the independent variable does not explicitly appear. When the independent variable represents time in some way, these may also be called \emph{time-invariant}.
\end{defn}

\begin{exmp}
    $\frac{\mathrm{d}y}{\mathrm{d}x} = 5y - 20$ is an autonomous differential equation as the independent variable $x$ is not explicitly present.
\end{exmp}

\section{First-order linear autonomous differential equations}

Any first-order linear autonomous differential equation with independent variable $x$ and dependent variable $y$ can trivially be put into the form \[\frac{\mathrm{d}y}{\mathrm{d}x} = ay - b.\] This form can always be solved.

\begin{align*}
    \frac{\mathrm{d}y}{\mathrm{d}x} &= ay - b \\
    &= a(y - \frac{b}{a})
\end{align*}

Note that $\frac{\mathrm{d}}{\mathrm{d}x}{y - \frac{b}{a}} = \frac{\mathrm{d}}{\mathrm{d}x}y$. If $y \neq \frac{b}{a}$, it follows that $\ln(y - \frac{b}{a}) = ax + C$ for some constant of integration $C$, so $\abs{y - \frac{b}{a}} = ke^{ax}$ for some $k$. If $y > \frac{b}{a}$, then $y = \frac{b}{a} + ke^{ax}$, and similarly if $y < \frac{b}{a}$, then $y = \frac{b}{a} + ke^{ax}$ for some $k$. In the case that $y = \frac{b}{a}$, it follows that $\frac{\mathrm{d}y}{\mathrm{d}x} = 0$, so $y = \frac{b}{a} + ke^{ax}$ where $k = 0$.

Therefore, $y = \frac{b}{a} + ke^{ax}$ is a solution to any first-order linear autonomous differential equation.

\end{document}
