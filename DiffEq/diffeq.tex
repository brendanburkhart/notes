\documentclass[12pt]{article}

\usepackage{amsmath}
\usepackage{amssymb}
\usepackage{amsthm}
\usepackage{centernot}
\usepackage{fullpage}
\usepackage{makecell}
\usepackage{tabularx}
\usepackage[hypcap=false]{caption}
\usepackage{tikz, tkz-euclide}
\usetikzlibrary{decorations.pathreplacing,arrows}
\usetikzlibrary{quotes,angles,calc,intersections}

\usepackage{titling}
\usepackage{pdfpages}
\usepackage{enumitem}
\usepackage{multicol}
\usepackage{bm}

\usepackage{linear}
\usepackage{common}

\begin{document}

\title{Ordinary Differential Equations}
\author{Brendan Burkhart}
\maketitle

\tableofcontents
\newpage

\section{Classification}

\begin{defn}
    A \emph{differential equation} is an equation relating one or more independent variables, functions of those variables, and derivatives of those functions.
\end{defn}

\begin{exmp}\label{first-order-ode}
    \[\frac{\mathrm{d}x}{\mathrm{d}t} = x\]
\end{exmp}

\begin{defn}
    A \emph{solution} to a differential equation is an expression for the dependent functions of the differential equation in terms of the independent variables.
\end{defn}

Differential equations can be broadly classified into \emph{ordinary} and \emph{partial} differential equations.

\begin{defn}
    An \emph{ordinary} differential equation, or \emph{ODE} is a differential equation involving a single independent variable, and all derivatives are with respect to this variable.
\end{defn}

\begin{defn}
    A \emph{partial} differential equation, or \emph{PDE} is a differential equation involving functions of more than one independent variables, and derivatives may be with respect to any of those variables.
\end{defn}

\begin{defn}
    The \emph{general form} of an ODE is
    \[F(x, y, y', \ldots, y^{(n)}) = 0,\] that is some expression involving the independent variable $x$, the dependent variable $y$, and the derivatives of $y$.
\end{defn}

\begin{defn}
    The \emph{order} of a differential equation is the order of the highest derivative.
\end{defn}

\begin{exmp}
    \[\frac{\mathrm{d}^2\theta}{\mathrm{d}t^2} + \frac{g}{L}\sin\theta = 0\]
    This is a second order differential equation, while Example \ref{first-order-ode} is first order.
\end{exmp}

\begin{defn}
    A differential equation of a dependent variable $y$ and its derivatives is said to be \emph{linear} if it is an affine map with regard to $y$ and its derivatives.
\end{defn}

\begin{exmp}
    $\sin(x)y' + 2x^2y'' = x$ is linear.
\end{exmp}

\begin{exmp}
    $yy' + y'' = 0$ is non-linear.
\end{exmp}

\begin{defn}
    An \emph{autonomous} differential equation is one in which the independent variable does not explicitly appear. When the independent variable represents time in some way, these may also be called \emph{time-invariant}.
\end{defn}

\begin{exmp}
    $\frac{\mathrm{d}y}{\mathrm{d}x} = 5y - 20$ is an autonomous differential equation as the independent variable $x$ is not explicitly present.
\end{exmp}

\section{First order differential equations}

\begin{defn}
    The \emph{standard form} of a linear first order differential equation (with independent variable $t$ and dependent variable $y$) is \[\frac{\mathrm{d}y}{\mathrm{d}t} + p(t)y = g(t),\] for some arbitrary functions $p(t), g(t)$.
\end{defn}

Any linear first order autonomous ODE with independent variable $x$ and dependent variable $y$ can trivially be put into the form \[\frac{\mathrm{d}y}{\mathrm{d}x} = ay - b.\] This form can always be solved.

\begin{align*}
    \frac{\mathrm{d}y}{\mathrm{d}x} &= ay - b \\
    &= a(y - \frac{b}{a})
\end{align*}

Note that $\frac{\mathrm{d}}{\mathrm{d}x}{y - \frac{b}{a}} = \frac{\mathrm{d}}{\mathrm{d}x}y$. If $y \neq \frac{b}{a}$, it follows that $\ln(y - \frac{b}{a}) = ax + C$ for some constant of integration $C$, so $\abs{y - \frac{b}{a}} = ke^{ax}$ for some $k$. If $y > \frac{b}{a}$, then $y = \frac{b}{a} + ke^{ax}$, and similarly if $y < \frac{b}{a}$, then $y = \frac{b}{a} + ke^{ax}$ for some $k$. In the case that $y = \frac{b}{a}$, it follows that $\frac{\mathrm{d}y}{\mathrm{d}x} = 0$, so $y = \frac{b}{a} + ke^{ax}$ where $k = 0$.

Therefore, $y = \frac{b}{a} + ke^{ax}$ is a solution to any linear first order autonomous differential equation.

\subsection{Linear equations}

Linear first order ODEs which are not autonomous can sometimes be solved through the use of an \emph{integrating factor}. We can write a linear first order ODE as \[y' + p(t)y = g(t),\] where $p(t), g(t)$ are arbitrary functions. Our goal here is to find a function $\mu(t)$, called the \emph{integrating factor}, such that
\[\frac{\mathrm{d}}{\mathrm{d}t}\mu(t)y = \mu(t)y' + p(t)\mu(t)y.\] That is, we want $\mu'(t) = p(t)\mu(t)$. Once such $\mu(t)$ is found, since \[\mu(t)y' + \mu(t)p(t)y = \mu(t)g(t),\] we clearly have \[\frac{\mathrm{d}}{\mathrm{d}t}\mu(t)y = \mu(t)g(t).\] It follows that \[y = \frac{1}{\mu(t)}\int_{t_0}^t \mu(s)g(s)ds + C\] for some constants $t_0$ and $C$.

We of course need to also be able to find such a $\mu(t)$. Note that since $\mu'(t) = p(t)\mu(t)$ we have \[\frac{\mu'(t)}{\mu(t)} = p(t).\] Since $\frac{\mathrm{d}}{\mathrm{d}t}\ln(\mu(t)) = \frac{\mu'(t)}{\mu(t)}$, it follows that $\ln(\mu(t)) = \int p(t)$. Therefore, $\mu(t) = e^{\int p(t)dt}$.

\begin{exmp}
    Using an integrating factor to solve $ty' + 2y = 4t^2$ with the initial condition $y(1) = 2$.

    First, we need the differential equation in the standard form $y' + p(t)y = g(t)$. Doing this, we obtain $y' + \frac{2}{t}y = 4t$, provided that $t \neq 0$. Then we have $\ln(\mu(t)) = \int \frac{2}{t}dt = 2\ln(t) = \ln(t^2)$, so $\mu(t) = t^2$. We know have $\frac{\mathrm{d}}{\mathrm{d}t}\mu(t)y = 4t^3$, which we may integrate to obtain $t^2y = t^4 + c$, and so $y = t^2 + \frac{c}{t^2}$. Applying the initial condition $y(1) = 2$, we have $2 = 1 + c$, so $c = 1$ and our final particular solution is \[y = t^2 + \frac{1}{t^2}, t > 0.\]
\end{exmp}

\subsection{Separable equations}

A separable first order differential equation is of the form
\[\frac{\mathrm{d}y}{\mathrm{d}t} = g(t)h(y),\] so that we may ``separate'' the independent and dependent variables to obtain
\[\frac{1}{h(y)}\frac{\mathrm{d}y}{\mathrm{d}t} = g(t).\] This can be often be integrated, with the particularly nice property that due to the chain rule, we can simply integrate the LHS with respect to $y$.
\[\int \frac{1}{h(y)}dy = \int g(t)dt.\] These integrals may or may not be computable, and even if they are the resulting equation may not be solvable for $y$. However, it can sometimes be done and is a simple and straightforward technique.

\subsection{Autonomous equations}

In general, any first order autonomous ODE is of the form $\frac{dy}{dt} = f(y)$. Note that this form is always separable.

\begin{rmk}
    Note that the slope field of every autonomous ODE has the same slope for any given value of $y$, regardless of $t$.
\end{rmk}

\begin{defn}
    Let $y' = f(y)$ be a first order autonomous ODE, and consider the set $\left\{y \in \R \compbar f(y) = 0 \right\}$. This is the \emph{critical point} set of the ODE.
\end{defn}

\begin{defn}
    Let $y' = f(y)$ be a first order autonomous ODE. For any $y_{\star}$ in the critical point set, $y(t) = y_{\star}$ is an \emph{equilibrium solution} to the ODE.
\end{defn}

\begin{exmp}
    Let $y' = y(1-y)$. The polynomial $y(1-y)$ is zero when $y = 0$ or $y = 1$, so $y(t) = 0$ and $y(t) = 1$ are equilibrium solutions.
\end{exmp}

\begin{rmk}
    In between equilibria, the sign of $y'$ does not change.
\end{rmk}

\begin{rmk}
    Given a first order autonomous ODE $y' = f(y)$, since the sign of $y'$ goes not change between critical points, we can use the sign of $y''$(which is also $f'(y)$) to learn more about the behavior of solutions around critical points.
\end{rmk}

\begin{defn}
    Let $y' = f(y)$ be a first order autonomous ODE. If $f(y) > 0$ at a critical point $y_{\star}$, then when $y > y_{\star}$ the solution increases and when $y < y_{\star}$ the solution decreases. That is, solutions near the equilibrium of the critical point diverge from it, so we call these \emph{unstable} equilibria. Similarly, if $f'(y_{\star}) < 0$ then nearby solutions converge towards $y_{\star}$ and so we call it a \emph{stable} equilibrium.
\end{defn}

\begin{rmk}
    Stable equilibria are also called \emph{sinks}, while unstable equilibria may be called \emph{sources}.
\end{rmk}

\begin{exmp}
    Let $y' = (1-y)^2(y-4)$, and so $y'' = -2(1-y)(y-4) + (1-y)^2$. Since both $y'$ and $y''$ are continuous on $\R$, by Theorem \ref{general-first-order-existence} we know that a unique solution must exist for any given initial value. It follows that no solution curves cross each other anywhere.

    Furthermore, we have see that the critical points are $1$ and $4$, and since $y''(4) = 9$ and $y''(1) = 0$ it follows that $4$ is an unstable equilibrium. We know that $1$ is an equilibrium as well, and that $y''' < 0$, so $1$ is semi-stable -- solutions converge to it from above, and diverge from it below.
\end{exmp}

\subsection{Existence and uniqueness of solutions}

\begin{thm}\label{linear-first-order-existence}
    Let $I = (\alpha, \beta)$ be an open interval containing $t = t_0$, and let $p, g$ be functions that are continuous on $I$. There exists a unique function $y(t)$ that satisfies both the differential equation
    \[\frac{dy}{dt} + p(t)y = g(t)\] and the initial condition $y(t_0) = y_0$ for some $y_0$.
\end{thm}

\begin{proof}
    Since the differential equation is linear, it can be solved using an integrating factor, and the initial condition uniquely determines the constant of integration.
\end{proof}

\begin{thm}\label{general-first-order-existence}
    Let the functions $f$ and $\frac{\partial{f}}{\partial{y}}$ be continuous in some rectangular region $\alpha < t < \beta, \gamma < y < \delta$ containing a point $(t_0, y_0)$. Then, for some interval $(t_0 - h, t_0 + h)$ within $(\alpha, \beta)$, there exists a unique function $y(t)$ that satisfies both the differential equation
    \[\frac{dy}{dt} = f(t, y)\] and the initial condition $y(t_0) = y_0$.
\end{thm}

\begin{cor}
    The continuity of $f$ alone is sufficient for the existence of solutions, but not their uniqueness.
\end{cor}

\begin{rmk}
    Note that Theorem \ref{general-first-order-existence} ensures two solutions cannot intersect each other, since the differential equation with initial condition corresponding to the intersection would be satisfied by both solutions.
\end{rmk}

\begin{exmp}
    Consider $\dot{z} = z(1-z)$. Since $z(1-z)$ is a polynomial, it is continuous for all $z$, and so is $\frac{d}{dz}z(1-z)$. By Theorem \ref{general-first-order-existence}, it therefore has a unique solution for any given initial value.
\end{exmp}

\begin{exmp}
    Consider $y(t) = y^{2/3}$. Since $y^{2/3}$ is continuous for all $y$, the \emph{existence} of a solution is guaranteed for all initial values by Theorem $\ref{general-first-order-existence}$. However, since $\frac{d}{dy}y^{2/3} = \frac{2}{3}y^{-1/3}$ is discontinuous at $y = 0$, the \emph{uniqueness} of the solution is only guaranteed for initial values $y \neq 0$.
\end{exmp}

\begin{exmp}
    Consider $\dot{x} = \sqrt{x}$. Since $\sqrt{x}$ is only continuous for $x > 0$, given the initial condition $x(0) = x_0$, neither the existence nor uniqueness of a solution is not guaranteed.
\end{exmp}

\subsection{Exact equations}

\begin{defn}
    A first order ODE of the form $M(x, y) + N(x, y)y' = 0$ (where $y$ is a function of $x$) is \emph{exact} if there exists some function $\psi(x, y)$ such that $\frac{d\psi}{dx} = M(x, y) + N(x, y)y'$.
\end{defn}

\begin{exmp}
    $2x + y^2 + 2xyy' = 0$ is an exact first order ODE, since $\psi(x, y) = x^2 + xy^2$ gives $\frac{d\psi}{dx} = 2x + y^2 + 2xyy' = 0$. It follows that $\psi(x, y) = \int{0}dx$, and so $x^2 + xy^2 = c$ for some constant $c$ is a solution.
\end{exmp}

\begin{thm}
    Let functions $M, N, M_y, N_x$ be continuous on a simply connected region in the $xy$ plane. Then $M(x, y) + N(x, y)y' = 0$ is an exact differential equation if and only if $M_x(x, y) = N_y(x, y)$ for all points on the region.
\end{thm}

\begin{proof}\proofbreak
    ($\implies$) If it is an exact differential equation, then by definition there exists some $\psi(x, y)$ such that $\psi_x(x, y) = M(x, y)$ and $\psi_y(x, y) = N(x, y)$. Therefore, $\psi_{xy}(x, y) = M_y(x, y)$ and $\psi_{yx} = N_x(x, y)$. Since $M_y, N_x$ are continuous are some region, $\psi$ is $C^2$ are that region, and so $\psi_{xy} = \psi_{yx}$, implying $M_y = N_x$.

    ($\impliedby$) Since $M$ is continuous on the region, we can integrate to obtain $\varPhi(x, y) = Q(x, y) + h(y)$, where $\frac{d}{dx}Q = M$ and $h(y)$ is an arbitrary function. We need $\frac{d}{dx}\varPhi(x, y) = M(x, y) + N(x, y)y'$, so we need to be able to choose $h(y)$ such that $\frac{d}{dy}\varPhi = N(x,y)$. We therefore have $\frac{d}{dy}\varPhi = \frac{d}{dy}Q + h'(y) = N(y)$, and so $h'(y) = N(y) - \frac{d}{dy}Q$. This equation is integrable (giving a solution for $h(y)$) when $N(x,y) - \frac{d}{dy}Q(x,y)$ is solely a function of $y$. Since $\frac{d}{dx}\left(N(x,y) - \frac{d}{dy}Q(x,y)\right) = N_x - M_y = 0$, it is not a function of $x$, and so there is function $\varPhi(x,y)$ such that $\frac{d}{dy}\varPhi = N(x,y)$ and $\frac{d}{dx}\varPhi = M(x,y)$.
\end{proof}

\begin{rmk}
    Some non-exact first order differential equations can be made exact by way of an integrating factor.
\end{rmk}

\end{document}
