\setchaptergraphic{}

\chapter{Topology}
\label{ch:topology}

\section{Topological spaces}

\begin{defn}
    A topological \emph{space} is a set $X$ of \emph{points} endowed with a \emph{topology} $\topo$. A topology for $X$ is a set of subsets of $X$ (the \emph{open sets} in the topology), such that $\emptyset \in \topo$ and $X \in \topo$ and which is closed under arbitrary unions, and finite intersections. That is, if $U_i \in \topo$ for $i = 1, \ldots, n$ then we must have
    \begin{align*}
        \bigintersect_{i=1}^{n}U_i \in \topo,
    \end{align*}
    and if $U_{a} \in \topo$ for all $a \in A$ then
    \begin{align*}
        \bigunion_{a\in A}U_a \in \topo.
    \end{align*}
\end{defn}

\begin{exmp}\proofbreak
    \begin{itemize}
        \item The \emph{trivial topology} (also \emph{indiscrete topology}) on $X$ is simply $\topo = \{\emptyset, X\}$.
        \item The \emph{discrete topology} on $X$ is $\topo = \mathcal{P}(X)$.
    \end{itemize}
\end{exmp}

\begin{defn}
    For a given set $X$ with topologies $\topo_1$ and $\topo_2$, we say that $\topo_1$ is \emph{coarser} than $\topo_2$ when $\topo_1 \subseteq \topo_2$, and \emph{finer} than $\topo_2$ when $\topo_2 \subseteq \topo_1$. When either relation holds, we say the topologies are comparable ---  of course, many topologies are not comparable.
\end{defn}

\begin{defn}
    Given a set $X$, a set $\mathcal{B}$ is said to be a topological \emph{basis} if it satisfies the following requirements:
    \begin{itemize}
        \item For every point $x \in X$, there exists at least one $B \in \mathcal{B}$ such that $x \in B$.
        \item If $x \in B_1 \in$ and $x \in B_2$, then $\mathcal{B}$ must contain $B_3$ such that $x \in B_3 \subseteq B_1 \intersect B_2$.
    \end{itemize}
    The topology \emph{generated} by $\mathcal{B}$ consists of all sets $U \subseteq X$ such that there exists $B \in \mathcal{B}$ for each $x \in U$, and $x \in B \subseteq U$.
\end{defn}

\begin{prop}\label{prop:basis-generates-topology}
    The set $\topo$ generated by $\mathcal{B}$ is a topology on $X$.
\end{prop}

\begin{proof}
    Note that $\emptyset \in \topo$ vacously, and it follows immediately from the definition that $X \in \topo$.

    Suppose that $U_a \in \topo$ for all $a$ in some index set $A$. Let $U = \union U_a$. For every $x \in U_a$, we know that there exists $B \in \mathcal{B}$ such that $x \in B \subseteq U_a$. Since $U_a \subseteq U$ by definition, we also have $x \in B \subseteq U$ for every $x \in U$, and so $U \in \topo$.

    For any $U_1, U_2 \in \topo$, let $U' = U_1 \union U_2$. Then for $x \in U'$, there must by definition be $B_1, B_2 \in \mathcal{B}$ such that $x \in B_1 \subseteq U_1$ and $x \in B_2 \subseteq U_2$. By the definition of $\mathcal{B}$, it follows that $\mathcal{B}$ must contain $B_3$ such that $x \in B_3 \subseteq B_1 \intersect B_2$. Therefore, $x \in B_3 \subseteq U_1 \intersect U_2$, and so $U_1 \intersect U_2 \in \topo$. It follows inductively that for any $U = \bigintersect_{i=1}^{n}U_i$, we must have $U \in \topo$.

    Therefore, $\topo$ contains $\emptyset$ and $X$, and is closed under finite intersections and arbitrary unions, and so is indeed a topology on $X$.
\end{proof}

\begin{exmp}
    $\mathcal{B} = \{X\}$ is a basis for the indiscrete topology on $X$, while $X$ itself (so the set of all single points in $X$) is a basis for the discrete topology.
\end{exmp}

\begin{lemma}\label{lemma:basis-is-union-collection}
    The topology $\topo$ generated by a basis $\mathcal{B}$ is equal to the set of all possible unions of basis elements.
\end{lemma}

\begin{proof}
    Suppose $U \in \topo$. By definition, for each $x \in U$ there is some $B_x \in \mathcal{B}$ such that $x \in B_x \subseteq U$, and so
    \begin{align*}
        U = \bigunion_{x\in U}B_x.
    \end{align*}

    Given an arbitrary union $U$ of basis elements, the union is an open set in the topology generated by $\mathcal{B}$ because basis elements are in the topology, and any topology is closed under arbitrary unions.
\end{proof}

\begin{rmk}
    The following lemma is useful, as it establishes a criterion to simultaneously prove a set is a basis, and that it generates a specific topology.
\end{rmk}

\begin{lemma}\label{lemma:basis-criterion}
    Let $X$ be a topological space, and let $\mathcal{C}$ be a collection of open sets in $X$. If, for every open set $U$ and point $x \in U$, there exists $C \in \mathcal{C}$ such that $x \in C \subseteq U$, then $\mathcal{C}$ is a basis for the topology on $X$.
\end{lemma}

\begin{proof}
    By definition, for every $x \in X$, there must exist an open set $U$ such that $x \in U$. Therefore, there exists $C \in \mathcal{C}$ such that $x \in C$, so $\mathcal{C}$ satisfies the first requirement of a basis.

    If $x \in C_1$ and $x \in C_2$, then we know that $U = C_1 \intersect C_2$ must be open. Then there exists $C_3$ such that $x \in C_3 \subseteq U = C_1 \intersect C_2$, so $\mathcal{C}$ satisfies the second requirement of a basis.

    We have shown that $\mathcal{C}$ is a basis, but it remains to show that it generates $\topo$ (as opposed to a different topology on $X$). By Lemma \ref{lemma:basis-is-union-collection}, it follows that the topology generated by $\mathcal{C}$ is contained in $\topo$. Furthermore, for any $U \in \topo$ and $x \in U$ we have (by definition of $\mathcal{C}$) that there exists $C_x \in \mathcal{C}$ such that $x \in C \subseteq U$. But then $U = \union_{x\in X}C_x$, and so $\topo$ is contained by the topology generated by $\mathcal{C}$.
\end{proof}

\begin{lemma}\label{lemma:subbasis-finer-topology}
    Let $\mathcal{B}$ and $\mathcal{B}'$ be bases on a set $X$, for topologies $\topo$ and $\topo'$ respectively. Then $\topo'$ is \emph{finer} than $\topo$ if and only if for any $x \in B \in \mathcal{B}$ there exists $B' \in \mathcal{B}'$ such that $x \in B' \subseteq B$.
\end{lemma}

\begin{proof}
    If $\topo'$ is finer, then $\topo'$ must contain $B$ by definition, and so there must be some $B' \in \mathcal{B}'$ such that $x \in B' \subseteq B$ by definition of a basis for $\topo'$.

    Suppose instead that the stated condition holds on $\topo'$, and fix any $U \in \topo$. For each $x \in U$, there must by definition exist $B_x \in \mathcal{B}$ such that $x \in B_x \subseteq U$. By the stated condition, there must then exist ${B'}_x$ such that $x \in {B'}_x \subseteq B_x \subseteq U$, and so
    \begin{align*}
        U = \bigunion_{x\in U}{B'}_x.
    \end{align*}
    Therefore, $U$ is the union of elements of $\topo'$, and so is contained in $\topo'$. Since this holds for any $U \in \topo$, it follows that $\topo \subseteq \topo'$, and so $\topo'$ is finer by definition.
\end{proof}

\begin{exmp}
    Let $\mathcal{B}$ consist of all intervals $(a, b) \subseteq \R$, then the topology generated by $\mathcal{B}$ is the \emph{standard} topology on $\R$. Let $\R_{\ell}$ denote the topology generated by all intervals $[a, b) \subseteq \R$ --- this is known as the \emph{lower limit topology}. By Lemma \ref{lemma:subbasis-finer-topology}, it follows that $\R_{\ell}$ is strictly finer than the standard topology.
\end{exmp}

\begin{defn}
    Given an ordered set $X$, the \emph{order topology} on $X$ has as basis elements all sets of the form $(a, b) = \{s \in X : a < x < b\}$, and if $X$ contains smallest or largests elements $x_0, x_1$, then it also contains the sets $[x_0, b)$ and $(a, x_1]$.
\end{defn}

\begin{defn}
    Given sets $X$ and $Y$ endowed with topologies, the \emph{product topology} on $X \times Y$ has basis consisting of $U \times V$ for all $U$ open in $X$ and $V$ open in $Y$.
\end{defn}

\begin{prop}
    The product topology is a topology on $X \times Y$.
\end{prop}

\begin{proof}
    Note that $X \times Y$ is itself in the chosen basis, so it trivially satisfies the first requirement of a basis. Furthermore, if $(x, y) \in U_1 \times V_1$ and also $(x, y) \in U_2 \times V_2$, then we have $U_3$ and $V_3$ such that $x \in U_3 \subseteq U_1 \intersect U_2$ and $y \in V_3 \subseteq V_1 \intersect V_2$, from which it follows that $(x, y) \in U_3 \times V_3 \subseteq (U_1 \intersect U_2) \times (V_1 \intersect V_2) = (U_1 \times V_1) \intersect (U_2 \times V_2)$. Therefore, we do indeed have a basis for a topology on $X \times Y$.
\end{proof}

\begin{thm}
    If $\mathcal{B}$ and $\mathcal{C}$ are bases for topologies on $X$ and $Y$ respectively, then
    \begin{align*}
        \mathcal{D} = \left\{B \times C : B \in \mathcal{B}, C \in \mathcal{C}\right\}
    \end{align*}
    is a basis for the corresponding product topology on $X \times Y$.
\end{thm}

\begin{proof}
    For any point $(x, y) \in W$, where $W = U \times V$ is open in the product topology, there must exist $B \in \mathcal{B}$ and $C \in \mathcal{C}$ such that $x \in B \subseteq U$ and $y \in C \subseteq V$. Therefore, $(x, y) \in B \times C \subseteq W$, and so the conclusion follows by Lemma \ref{lemma:basis-criterion}.
\end{proof}

\begin{thm}
    Let $\topo_X$ and $\topo_Y$ be topologies on spaces $X$ and $Y$. The set
    \begin{align*}
        S = \left\{\pi_1^{-1}(U) : U \in \topo_X\right\} \union \left\{\pi_2^{-1}(V) : V \in \topo_Y\right\}
    \end{align*}
    is a subbasis for the product topology on $X \times Y$, where $\pi_1: X \times Y \to X$ and $\pi_2: X \times Y \to Y$ are the first and second projections.
\end{thm}

\begin{proof}
    For any open set $W = U \times V$ in the product topology, we have $\pi_1^{-1}(U) = U \times Y \in S$ and $\pi_2^{-1}(V) = X \times V \in S$. Therefore $W = \pi_1^{-1}(U) \intersect \pi_2^{-1}(V)$ is a finite intersection of elements of $S$.
\end{proof}

\begin{defn}
    Let $\topo$ be a topology on $X$. Given a subset $Y$ of $X$, we can form the \emph{subspace topology}
    \begin{align*}
        \topo_Y = \left\{U \intersect Y : U \in \topo\right\}.
    \end{align*}
\end{defn}

\begin{prop}
    The subspace topology is a topology on $Y$.
\end{prop}

\begin{proof}
    Since $\emptyset \in \topo$ we have $\emptyset \in \topo_Y$ and since $X \in \topo$ we have $X \intersect Y = Y$ in $\topo_Y$. Furthermore,
    \begin{align*}
        \bigunion_{\alpha \in A}\left(U_{\alpha} \intersect Y\right) = \left(\bigunion_{\alpha \in A}U_{\alpha}\right) \intersect Y,
    \end{align*}
    so $\topo_Y$ is closed under arbitrary unions. Finally,
    \begin{align*}
        \bigintersect_{i=1}^{n}\left(U_i \intersect Y\right) = \left(\bigintersect_{i=1}^{n}U_i\right) \intersect Y,
    \end{align*}
    and so $\topo_Y$ is closed under finite intersections.
\end{proof}

\begin{lemma}\label{lemma:subspace-basis}
    Given a basis $\mathcal{B}$ for a topology $\topo$ on $X$, the set
    \begin{align*}
        \mathcal{B}_Y = \left\{B \intersect Y : B \in \mathcal{B}\right\}
    \end{align*}
    is a basis for the subspace topology $\topo_Y$.
\end{lemma}

\begin{proof}
    For any open set $V$ in the subspace topology, by definition there exists $U \in \topo$ such that $V = U \intersect Y$. Therefore, for any $x \in U$ there exists $B \in \mathcal{B}$ such that $x \in B \subseteq U$. Therefore, for any $x \in V$ there exists $B_Y = B \intersect Y$ in $\mathcal{B}_Y$ such that $x \in B_Y \subseteq V$. It follows by Lemma \ref{lemma:basis-criterion} that $\mathcal{B}_Y$ is a basis for the subspace topology.
\end{proof}

\begin{thm}
    Given a subspace $Y$ of $X$, if $Y$ is open in $X$ then all sets $U$ open in $Y$ are also open in $X$.
\end{thm}

\begin{proof}
    If $U$ is open in $Y$, then $U = V \intersect Y$ for some $V$ open in $X$. Therefore, $U$ is the intersection of two sets open in $X$, and so must also be open in $X$.
\end{proof}

\begin{lemma}
    Let $X$ be a topological space, with subsets $Y$ and $Z$ such that $Z \subseteq Y \subseteq X$. The subspace topologies of $Z$ as a subset of $X$ and as a subset of $Y$ are the same topology.
\end{lemma}

\begin{proof}
    A element of the basis of $Z$ as a subset of $X$ has the form $U \intersect Z$, and as a subset of $Y$ will have the form $(U \intersect Y) \intersect Z$. Since $Z$ is a subset of $Y$, it follows that $(U \intersect Y) \intersect Z = U \intersect Z$.
\end{proof}

\begin{thm}
    Let $A$ be a subspace of $X$ and $B$ a subspace of $Y$. The product topology of the subspace topologies on $A$ and $B$ is the same as the subspace topology of $A \times B$ of the product topology on $X \times Y$.
\end{thm}

\begin{proof}
    We know that the collection of sets $(U \times V) \intersect (A \times B)$ for all $U$ open in $X$ and $V$ open in $Y$ is a basis for the subspace topology of the product topology, by Lemma \ref{lemma:subspace-basis}.

    We also know that the product topology of the subspace topologies has a basis consisting of all sets of the form $(U \intersect A) \times (V \intersect B)$. But then, since
    \begin{align*}
        (U \intersect A) \times (V \intersect B) = (U \times V) \intersect (A \times B),
    \end{align*}
    it follows the two bases are equal, and therefore so are the topologies.
\end{proof}

\begin{defn}
    Let $X$ be an ordered set. A subset $Y$ is \emph{convex} in $X$ if $Y$ contains $(a, b) \subseteq X$ for all $a, b \in Y$.
\end{defn}

\begin{thm}
    Let $X$ be an ordered set with the order topology, and let $Y$ be a convex subset of $X$. The order topology and subspace topology on $Y$ are the same.
\end{thm}

\begin{proof}
    We will prove these topologies can both be generated by the same subbasis. Note that the set of all open rays $(a, +\infty)$ and $(-\infty, a)$ form a subbasis for the order topology, and the intersection of an open ray in $X$ with $Y$ is either an open ray in $Y$, all of $Y$, or empty. Conversely, any open ray in $Y$ is the intersection of $Y$ and an open ray in $X$, so these subbases are equal.
\end{proof}

\begin{defn}
    Letting $J$ be some index set, then a Cartesian product of sets $X_j$, denoted
    \begin{align*}
        \prod_{j \in }X_j,
    \end{align*}
    is the set of all functions $t: J \to \union_{j \in J}X_j$, such that $t(j) \in X_j$.
\end{defn}

\section{Closure and Limit Points}

\begin{defn}
    A set $A$ in the topological space $(X, \topo)$ is \emph{closed} when it's complement $X \setminus A$ is open in $\topo$.
\end{defn}

\begin{thm} Let $(X, \topo)$ be a topological space.
    \begin{itemize}
        \item Both $\emptyset$ and $X$ are closed.
        \item Arbitrary intersections of closed sets are closed.
        \item Finite unions of closed sets are closed.
    \end{itemize}
\end{thm}

\begin{proof}\proofbreak
    \begin{itemize}
        \item Both $\emptyset$ and $X$ are open, and are each other's complements, so both are closed.
        \item If $S_{\alpha}$ for $\alpha \in A$ are closed sets, then their complements are open, and so \begin{align*}
            X \setminus \bigintersection_{\alpha \in A}S_{\alpha} = \bigunion_{\alpha \in A}\left(X \setminus S_{\alpha}\right).
        \end{align*}
        Note that $X \setminus S_{\alpha}$ is open by definition, and so any arbitrary intersection of closed sets can be expressed as the complement of an arbitrary union of open sets.
        \item Similarly, finite unions of closed sets are complements of finite intersections of open sets.
    \end{itemize}
\end{proof}

\begin{lemma}\label{lemma:closed-subset-criterion}
    Let $(X, \topo)$ be a topological space, and consider the subspace topology on $Y \subseteq X$. Then $A \subseteq Y$ is closed in $\topo_Y$ if and only if $A = B \intersect Y$ for some $B$ that is closed in $\topo$.
\end{lemma}

\begin{proof}
    Suppose $A = B \intersect Y$ and $B$ is closed in $\topo$. Then $Y \setminus A$ equals $Y \setminus B = Y \intersect (X \setminus B)$. Since $X \setminus B$ is open in $\topo$ by definition, it follows that $Y \intersect (X \setminus B)$ is open in the subspace topology, and so $A$ is closed in $\topo_Y$.

    Suppose that $A$ is closed in $\topo_Y$, and so $Y \setminus A$ is open in $\topo_Y$. We then have $Y \setminus A = Y \intersect S$ for some $S$ open in $\topo$. If we let $B = X \setminus S$, it follows that
    \begin{align*}
        B \intersect Y = Y \intersect (X \setminus S) = Y \setminus S = Y \setminus (Y \intersect S).
    \end{align*}
    Of course, $Y \intersect S = Y \setminus A$ as we saw, and so $A = B \intersect Y$.
\end{proof}

\begin{lemma}
    Let $X$ be a topological space with subspace $Y$. If $A$ is closed in $Y$, and $Y$ is closed in $X$, then $A$ is closed in $X$.
\end{lemma}

\begin{proof}
    Since $A$ is closed in $Y$, then by Lemma \ref{lemma:closed-subset-criterion} $A = B \intersect Y$ where both $Y$ and $B$ are closed in $X$. Since $A$ is the intersection of closed sets, it it also closed in $X$.
\end{proof}

\begin{defn}
    In a space $X$, the \emph{interior} of $A \subseteq X$ is the union of all open sets contained in $A$, and the \emph{closure} of $A$ (denoted $\closure{A}$) is the intersection of all closed sets with contain $A$.
\end{defn}

\begin{thm}
    Consider a space $X$ with subspace $Y$, and some $A \subseteq Y$. Then the closure of $A$ in $Y$ equals $\closure{A} \intersect Y$, where $\closure{A}$ is taken with respect to $X$.
\end{thm}

\begin{proof}
    Suppose that $C$ is closed in $X$ and contains $A$. Then by Lemma \ref{lemma:closed-subset-criterion} $Y \intersect C$ is closed in $Y$, and contains $A$. Now instead suppose $C$ is closed in $Y$ and contains $A$. Then by the same lemma we have $C = Y \intersect D$ for some $D$ closed in $X$, and so $D$ is a closed set in $X$ which contains $A$.

    Let $C_{\alpha}$ be all closed sets in $Y$ which contain $A$, and then we have proven
    \begin{align*}
        \bigunion C_{\alpha} = \bigunion (Y \intersect D_{\alpha}) = Y \intersect \bigunion D_{\alpha} = Y \intersect \closure{A}.
    \end{align*}
\end{proof}

\begin{defn}
    Let $X$ be a space, and $x$ a point in that space. A \emph{neighborhood} of $x$ is an open set which contains $x$.
\end{defn}

\begin{thm}\label{thm:closure-neighborhood-characterization}
    Let $X$ be a space, and consider some $A \subseteq X$. A point $x \in X$ is in $\closure{A}$ if and only if $U \intersect A$ is non-empty for all neighborhoods $U$ of $x$.
\end{thm}

\begin{proof}
    Suppose there exists some neighborhood $U$ of $x$ such that $U \intersect A$ is empty. Then $X \setminus U$ is a closed set containing $A$, but not $x$, and so $x \not\in \closure{A}$.

    Conversely, if $x \not\in \closure{A}$, there must exist a closed set which contains $A$ but not $x$. Let $U$ be the open complement of that closed set. It follows that $A \intersect U$ is empty, and since $x \in U$ we know $U$ is a neighborhood of $x$.
\end{proof}

\begin{cor}
    If $X$ has a basis $\mathcal{B}$, then a point $x \in X$ is in $\closure{A}$ if and only if $B \intersect A$ is non-empty for all $B \in \mathcal{B}$ that contain $x$.
\end{cor}

\begin{proof}
    If $x \in \closure{A}$, then all basis elements containing $x$ have non-empty intersection with $A$ by Theorem \ref{thm:closure-neighborhood-characterization}, since such basis elements are neighborhoods of $x$.

    If every basis element containing $x$ has non-empty intersection with $A$, then so does every neighborhood of $x$, since for every neighborhood $U$ there is a basis element $B$ such that $x \in B \subseteq U$. Therefore, $x \in \closure{A}$ by Theorem \ref{thm:closure-neighborhood-characterization}.
\end{proof}

\begin{lemma}\label{lemma:closure-closed}
    A set is closed if and only if it equals its closure, and is open if and only if it equals its interior.
\end{lemma}

\begin{defn}
    Let $X$ be a topological space, and $A$ an subset of $X$. A point $p \in X$ is called a \emph{limit point} of $A$ if every neighborhood of $p$ contains some $q \in A$ such that $p \neq q$.
\end{defn}

\begin{thm}\label{thm:closure-limit-points}
    Consider a topological space $X$ with subset $A$, and let $A'$ denote the limit points of $A$. Then $\closure{A} = A \union A'$.
\end{thm}

\begin{proof}
    If $x \in A'$, then every neighborhood of $x$ must intersect $A$, and so by Theorem \ref{thm:closure-neighborhood-characterization} we know $x \in \closure{A}$ and so $A' \subseteq \closure{A}$. Since $A \subseteq \closure{A}$, it follows that $A \union A'$ is a subset of $\closure{A}$.

    Now suppose $x \in \closure{A}$. Since every neighborhood of $x$ must intersect $A$, if $x \not\in A$ then each neighborhood must contain at least one point in $A$ which is necessarily distinct from $x$ (since $x \not\in A$), and so $x$ would be a limit point of $A$. Therefore, $x \in A$ or $x \in A'$, and so $\closure{A} \subseteq A \union A'$.
\end{proof}

\begin{cor}\label{cor:closed-iff-self-closure}
    $A$ is closed if and only if $A' \subseteq A$.
\end{cor}

\begin{defn}
    Consider a topological space $X$. If for all $p, q \in X$ there exists neighborhoods $U$ of $p$ and $V$ of $q$ such that $p \not\in V$ and $q \not\in U$, then we say that $X$ is $T_1$ separable.
\end{defn}

\begin{lemma}
    A topological space $X$ is $T_1$ if and only if every finite subset is closed.
\end{lemma}

\begin{proof}
    Suppose $X$ is $T_1$, and consider a singleton set $\{p\}\subset X$. If $q \neq p$, then by $T_1$ there exists a neighborhood $U$ of $q$ that doesn't contain $p$. Therefore, $U^c$ is a closed set containing $\{p\}$ but not $q$, and so $\{p\}$ equals its own closure, and therefore is closed. Any finite $A \subset X$ is then a finite union of closed sets, and therefore is closed.

    Now instead suppose that every finite subset is closed. In particular, for any $p, q \in X$ we know $U^c = \{p\}$ is closed, and so $U$ is an open set containing $q$ but not $p$. Similarly, $V^c = \{q\}$ is closed, so $V$ is an open set containing $p$ but not $q$. Therefore $X$ is $T_1$.
\end{proof}

\begin{defn}
    Consider a topological space $X$. If, for any $p, q \in X$ there exist neighborhoods of $p$ and $q$ that are disjoint, then we say that $X$ is \emph{Hausdorff}.
\end{defn}

\begin{lemma}
    Every Hausdorff space is $T_1$.
\end{lemma}

\begin{exmp}
    Consider the finite complement topology on $\R$, where sets are open when they have finite complement. Any finite set is of course closed, so the space is $T_1$. However, this space is \emph{not} Hausdorff, since no open sets can possibly be disjoint in this space.
\end{exmp}

\begin{defn}
    Let $\{x_n\}$ be a sequence of points in a topological space. The sequence \emph{converges} to a point $x$ if for every neighborhood $U$ of $x$ there exists $N \in \N$ such that $x_n \in U$ for all $n \geq N$.
\end{defn}

\begin{lemma}
    If $X$ is a Hausdorff space, sequences in the space converge to at most one point.
\end{lemma}

\begin{proof}
    Suppose that $\{x_n\}$ is a sequence which converges to points $x$ and $y$. Since $X$ is Hausdorff, if $x \neq y$ then there exist neighborhoods $U$ and $V$ of $x$ and $y$ respectively such that $U$ and $V$ are disjoint. Since $\{x_n\}$ converges to $x$, the neighborhood $U$ must contain all but finitely many points of the sequence. Since $V$ is disjoint from $U$, it can only contain finitely many points of the sequence, and so the sequence cannot converge to $y$.
\end{proof}

\begin{lemma}
    Given a totally ordered set $X$, the order topology is Hausdorff.
\end{lemma}

\begin{proof}
    Consider distinct points $x$ and $y$, without loss of generality assume $x < y$. If there exists $z$ such that $x < z < y$, then $(-\infty, z)$ and $(z, +\infty)$ are neighborhoods of $x$ and $y$ respectively, and are disjoint.

    If no such $z$ exists, then $(-\infty, y)$ and $(x, +\infty)$ are disjoint neighborhoods.
\end{proof}

\begin{thm}\proofbreak
    \begin{itemize}
        \item The product of two Hausdorff spaces is Hausdorff.
        \item Any subspace of a Hausdorff space is Hausdorff.
    \end{itemize}
\end{thm}

\begin{proof}
    \proofbreak
    \begin{itemize}
        \item Consider two distinct points $(x, y)$ and $(p, q)$. Since $X$ and $Y$ are both Hausdorff, there exist disjoint neighborhoods $U, V \subseteq X$ of $x$ and $p$ respectively, and $W, Z \subseteq Y$ of $y$ and $q$ respectively. Then $U \times W$ and $V \times Z$ are open neighborhoods in $X \times Y$ which are necessarily disjoint, and contain $(x, y)$ and $(p, q)$ respectively.
        \item Suppose $Y \subseteq X$ is a subspace of $X$, and $p, q \in Y$. Since $X$ is Hausdorff, there exist disjoint neighborhoods $U, V \subseteq X$ such that $p \in U$ and $q \in V$. Then $p \in U \intersect Y$ and $q \in V \intersect Y$, and of course $(U \intersect Y)$ and $(V \intersect Y)$ are disjoint neighborhoods in $Y$.
    \end{itemize}
\end{proof}

\section{Continuity and Homeomorphisms}

\begin{defn}
    Let $X$ and $Y$ be topological spaces, and consider a function $f: X \to Y$. We say $f$ is \emph{continuous} when $f^{-1}(O) \subseteq X$ is open for every open set $O \subseteq Y$.
\end{defn}

\begin{lemma}
    If $\mathcal{B}$ is a basis for $Y$, and $f^{-1}(B) \subseteq X$ is open for every $B \in \mathcal{B}$, then $f$ is continuous.

    If $\mathcal{S}$ is a subbasis for $Y$, and $f^{-1}(S)$ is open for every $S \in \mathcal{S}$, then $f$ is continuous.
\end{lemma}

\begin{proof}
    Any open set $O \subseteq Y$ can be expressed as an arbitrary union of basis elements $B_{\alpha} \in \mathcal{B}$. Then,
    \begin{align*}
        f^{-1}(O) = f^{-1}\left(\bigunion B_{\alpha}\right) = \bigunion f^{-1}(B_{\alpha})
    \end{align*}
    is a union of open sets in $X$, and therefore must be open.

    Any basis element $B$ can be expressed as a finite intersection of subbasis elements $S_{i} \in S$, and so
    \begin{align*}
        f^{-1}(B) = f^{-1}\left(\bigintersection S_{i}\right) = \bigintersection f^{-1}(S_i).
    \end{align*}
    If $f^{-1}(S_i)$ is open, then $f^{-1}(B)$ is a finite intersection of open sets and therefore is open.
\end{proof}

\begin{thm}\label{thm:continuity-characterizations} Let $X$ and $Y$ be topological spaces, and $f: X \to Y$ a function between them. Then the following are equivalent:
    \begin{enumerate}[label=(\arabic*)]
        \item $f$ is continuous,
        \item For every $A \subseteq X$, $f(\closure{A})$ is a subset of the closure of $f(A)$.
        \item $f^{-1}(B) \subseteq X$ is closed for every closed set $B \subseteq Y$.
        \item Given any point $x \in X$ and neighborhood $V \subseteq Y$ of $f(x)$, there exists some neighborhood $U$ of $x$ such that $f(U) \subseteq V$.
    \end{enumerate}
\end{thm}

\begin{proof}
    First, we prove that $(1) \implies (2) \implies (3) \implies (1)$. Suppose that $f$ is continuous, and consider some $A \subseteq X$. For any $x \in \closure{A}$, let $V$ be any neighborhood of $f(x)$. Since $f$ is continuous, $f^{-1}(V)$ is a neighborhood of $x$, and so it must intersect $A$ at a point $y$. Then $V$ must intersect $f(y) \in f(A)$, and so $f(x) \in \closure{f(A)}$.

    Suppose instead that the second property holds, and let $B$ be a closed subset of $Y$, and let $A = f^{-1}(B)$. Then $f(A) = B$, and so $\closure{f(A)} = B$. It follows that $f(\closure{A}) \subseteq B$, and so $\closure{A} = A$, and so by Lemma \ref{lemma:closure-closed} it follows that $f^{-B}$ is closed.

    Suppose instead that the third property holds. Then for any open set $A \subseteq Y$, its complement $Y \setminus A$ is closed, and so $f^{-1}(Y \setminus A)$ is closed. But $f^{-1}(Y \setminus A) = f^{-1}(Y) \setminus f{-1}(A)$. Since $f^{-1}(Y) = X$, it follows that $f^{-1}(A)$ is the complement of the closed set $f^{-1}(Y \setminus A)$, and therefore must be open.

    Now we prove that $(1) \iff (4)$. If $f$ is continuous, then $U = f^{-1}(V)$ is an open neighborhood of $x$, and of course $f(U) = V$. If instead we assume that property four holds, let $O$ be an arbitrary open set in $Y$. For each $x \in V = f^{-1}(O)$, there exists a neighborhood $U_x$ of $x$ such that $f(U_x) \subseteq O$. Therefore,
    \begin{align*}
        f^{-1}(O) = \bigunion_{x \in V}U_x
    \end{align*}
    is a union of open sets, and therefore is open.
\end{proof}

\begin{defn}
    A function $f: X \to Y$ is a \emph{homeomorphism} when $f$ is continuous, and $f^{-1}$ both exists and is continuous.
\end{defn}

\begin{rmk}
    Equivalently, a homeomorphism is a bijection $f: X \to Y$ such that $f(U)$ is open if and only if $U$ is open. Therefore, a homeomorphism can also be viewed as a bijection between open sets.
\end{rmk}

\begin{exmp}
    The function $f(x) = 4x - 2$ from $\R \to \R$ is a homeomorphism under the standard topology.
\end{exmp}

\begin{exmp}
    The function $f(t) = (\cos(t), \sin(t))$ from $[0, 2\pi) \subseteq \R$ to the unit circle $S^1 \subseteq \R^2$ is not a homeomorphism, since $[0, t)$ is open in $[0, 2\pi)$, but $f([0, t))$ is not open in $\S^1$.
\end{exmp}

\begin{defn}
    If $f: X \to Y$ is a homeomorphism between $X$ and $f(X)$, we say that $f$ is a topological embedding of $X$ into $Y$.
\end{defn}

\begin{thm}
    If $f: X \to Y$ and $g: Y \to Z$ are continuous, then so is $g \circ f: X \to Z$.
\end{thm}

\begin{proof}
    Let $U$ be an open set in $Z$, then $V = g^{-1}(U)$ is open since $g$ is continuous, and so $f^{-1}(V)$ is open since $f$ is continuous. Therefore, $(g \circ f)^{-1}(U)$ is open in $X$.
\end{proof}

\section{Product Topology}

\begin{defn}
    Given an indexed family $\{X_{\alpha}\}_{\alpha\in J}$ of topological spaces, consider the basis consisting of all sets of the form $\prod_{\alpha\in J}U_{\alpha}$, where $U_{\alpha}$ is open in $X_{\alpha}$. We define the \emph{box topology} to be the topology on $\prod_{\alpha\in J}X_{\alpha}$ generated by this basis.
\end{defn}

\begin{defn}
    Given an indexed family $\{X_{\alpha}\}_{\alpha\in J}$ of topological spaces, consider the subbasis
    \begin{align*}
        \bigunion_{\beta\in J}\left\{\pi_{\beta}^{-1})(U_{\beta}) : U_{\beta} \in \tau_{\beta}\right\}.
    \end{align*}
    We define the \emph{product topology} to be the topology generated by this subbasis.
\end{defn}

\begin{prop}
    The box topology is finer than the product topology.
\end{prop}

\begin{proof}
    Any element of the basis of the product topology is a finite intersection of subbasis elements. Since $\pi_{\beta}^{-1}(U) \intersect \pi_{\beta}^{-1}(V) = \pi_{\beta}^{-1}(U \intersect V)$, we can consider an intersection over distinct indices. Then any basis element $\pi_{\beta_1}^{-1}(U_{\beta_1}) \intersect \cdots \intersect \pi_{\beta_n}^{-1}(U_{\beta_n})$ of the product topology is equal to $\prod_{\alpha\in J}U_{\alpha}$, where $U_{\alpha} = X_{\alpha}$ for all $\alpha\neq\beta_k$ for any $1 \leq k \leq n$. Therefore, the basis of the box topology contains the basis of the product topology.
\end{proof}

\begin{rmk}
    For finite products, any basis element of the box topology is a finite intersection of subbasis elements of the product topology, and so the box and product topologies are the same. However, they will generally differ for an infinite product.
\end{rmk}

\begin{thm}
    If $A_{\alpha} \subseteq X_{\alpha}$ is a subspace for all $\alpha \in J$, then $\prod A_{\alpha}$ is a subspace of $\prod X_{\alpha}$ when both products have the box topology, or both have the product topology.
\end{thm}

\begin{thm}
    If each $X_{\alpha}$ is a Hausdorff space, then $\prod X_{\alpha}$ is a Hausdorff space under both the box and product topologies.
\end{thm}

\begin{thm}
    Consider $\prod X_{\alpha}$ with the product topology. Let $f: A \to \prod X_{\alpha}$ be given component-wise by $f(p) = (f_{\alpha}(p))_{\alpha\in J}$, where $f_{\alpha}: A \to X_{\alpha}$. Then $f$ is continuous if and only if each $f_{\alpha}$ is continuous.
\end{thm}

\begin{proof}
    Suppose $f$ is continuous. Then since $\pi_{\alpha}$ is continuous, we know $f_{\alpha} = \pi_{\alpha} \circ f$ is continuous. Conversely, suppose each $f_{\alpha}$ is continuous. Then for any subbasis element $\pi_{\beta}^{-1}(U_{\beta})$, we know $f^{-1}(\pi_{\beta}^{-1}(U_{\beta})) = (\pi_{\beta}\circ f)^{-1}(U_{\beta}) = f_{\beta}^{-1}(U_{\beta})$, so the preimage of any subbasis element is open, and so $f$ is continuous.
\end{proof}

\begin{rmk}
    A component-wise function into the box topology can fail to be continuous even if each component is continuous. Consider $f: \R \to \R^{\omega}$, where each component is the identity function $\textrm{id}: \R \to \R$. Now consider the open set in the box topology on $\R^{\omega}$, given by $\prod U_{n}$ where $U_n = (-1/n, 1/n)$. But $f^{-1}(\prod U_n) = \{0\}$, which is not open, and so $f$ is not continuous.
\end{rmk}

\section{Metric Topology}

\begin{defn}
    A \emph{metric} on a set $X$ is a function $d: X \times X \to \R_{\geq 0}$ such that for all $p, q, r \in X$ we have
    \begin{itemize}
        \item $d(p, q) = 0 \iff p = q$,
        \item $d(p, q) = d(q, p)$,
        \item $d(p, r) \leq d(p, q) + d(q, r)$.
    \end{itemize}
\end{defn}

\begin{lemma}\label{lemma:inner-ball}
    Let $X$ be a set endowed with metric $d$. Let $B(x, r)$ be the ball centered at $x$ with radius $r$ --- that is,
    \begin{align*}
        B(x, r) = \left\{y \in X : d(x, y) < r\right\}.
    \end{align*}
    Then for any $y \in B(x, r)$, there exists $\delta \in \R^+$ such that $B(y, \delta) \subseteq B(x, r)$.
\end{lemma}

\begin{proof}
    Let $\delta = r - d(y, x)$. Then for any $z \in B(y, \delta)$, we have
    \begin{align*}
        d(z, x) \leq d(z, y) + d(y, x) < \delta + d(y, x) = r - d(y, x) + d(y, x) = r,
    \end{align*}
    and so $z \in B(x, r)$.
\end{proof}

\begin{lemma}
    Let $X$ be a set endowed with metric $d$. The set
    \begin{align*}
        \left\{B(x, r) : x \in X, r \in \R^{+}\right\}
    \end{align*}
    is a basis.
\end{lemma}

\begin{proof}
    Let $x$ be any point in $X$, then $B(x, 1)$ is a ball containing $x$.

    Suppose $B(u, s)$ and $B(v, t)$ are two balls which both contain a point $x$. By Lemma \ref{lemma:inner-ball}, let $B(x, \delta_1)$ and $B(x, \delta_2)$ be such that $B(x, \delta_1) \subseteq B(u, s)$ and $B(x, \delta_2) \subseteq B(v, t)$. Then letting $\varepsilon = \min(\delta_1,\delta_2)$, it follows that
    \begin{align*}
        B(x, \varepsilon) \subseteq B(u, s) \intersect B(v, t).
    \end{align*}
\end{proof}

\begin{defn}
    Given a set $X$ endowed with metric $d$, the corresponding \emph{metric topology} is generated by the basis of open balls.
\end{defn}

\begin{exmp}
    Let $X$ be any set, and consider the metric
    \begin{align*}
        d(x, y) = \begin{dcases}
            0, &x = y \\
            1, &x \neq y
        \end{dcases}.
    \end{align*}
    The corresponding metric topology will be the discrete topology.
\end{exmp}

\begin{lemma}
    A set $U$ is open in the metric topology if and only if for any $x \in U$ there exists $\delta > 0$ such that $B(x, \delta) \subseteq U$.
\end{lemma}

\begin{proof}
    If such $\delta$ exists, then $U$ is open by definition. Conversely, if $U$ is open then for any $x \in U$ there must exist $B(y, \varepsilon)$ such that $x \in B(y, \varepsilon) \subseteq U$. By Lemma \ref{lemma:inner-ball}, it follows that there exists $\delta$ such that $B(x, \delta) \subseteq B(y, \varepsilon)$.
\end{proof}

\begin{defn}
    A topology $\topo$ is \emph{metrizable} when there exists a metric $d$ such that the corresponding metric topology is the same as $\topo$.
\end{defn}

\begin{defn}
    Given a set $X$ with metric $d$, the metric
    \begin{align*}
        \bar{d}(x, y) = \min(d(x, y), 1)
    \end{align*}
    is the \emph{standard bounded metric}.
\end{defn}

\begin{rmk}
    The standard bounded metric induces the same metric topology as the original metric.
\end{rmk}

\begin{lemma}\label{lemma:finer-metric-topology}
    Let $d, d'$ be two metrics on a set $X$, and let $\topo, \topo'$ be the metric topologies they induce. $\topo'$ is finer than $\topo$ if and only if for all $B(x, \varepsilon) \in \topo$ there exists $B(x, \delta) \subseteq B(x, \varepsilon)$.
\end{lemma}

\begin{proof}
    Follows from Lemma \ref{lemma:subbasis-finer-topology} and Lemma \ref{lemma:inner-ball}.
\end{proof}

\begin{thm}
    The topologies induced by the $\ell_p$ (including $\ell_{\infty}$) metrics on $\R^n$ are all equivalent to the product topology.
\end{thm}

\begin{proof}
    Since
    \begin{align*}
        \norm{x}_{\infty} \leq \norm{x}_{p} \leq n^{1/p}\norm{x}_{\infty},
    \end{align*}
    it follows that
    \begin{align*}
        B_{\infty}(x, \delta) \subseteq B_{p}(x, \delta) \subseteq B_{\infty}(x, n^{1/p}\delta),
    \end{align*}
    and so by Lemma \ref{lemma:finer-metric-topology} that these topologies are all the same.

    Now we prove that $\ell_{\infty}$ induces the product topology. Let $B$ be any basis element in the product topology, so $B = (a_1, b_1) \times (a_2, b_2) \times \cdots \times (a_n, b_n)$. Let $x \in B$, and take $\delta_i$ such that $x + \delta_i < b_i$ and $x - \delta_i > a_i$. Let $\delta = \min(\delta_i)$, and then $B_{\infty}(x, \delta) \subseteq B$. Conversely, any basis element $B_{\infty}(x, \delta)$ in the metric topology is itself a basis element in the product topology. It follows by Lemma \ref{lemma:finer-metric-topology} that these topologies are the same.
\end{proof}

\begin{defn}
    A metric $d: X \times X \to \R$ is an \emph{ultrametric} if it satisfies the \emph{strong triangle inequality}
    \begin{align*}
        d(x, z) \leq \max\left(d(x, y), d(y, z)\right).
    \end{align*}
\end{defn}

\begin{exmp}
    Consider the field $\C((x))$ of formal Laurent series in the complex numbers, which are power series
    \begin{align*}
        f(x) &= \sum_{k=N}^{\infty}c_kx^k,
    \end{align*}
    where $c_k \in \C$ and $N > -\infty$. We define $\abs{f(x)} = 2^{-k}$, where $k$ is the smallest index such that $c_k$ is non-zero. Then $d(f, g) = \abs{f-g}$ is an ultrametric on $\C((x))$. We quickly verify this is an (ultra)metric. Let $f(x), g(x), h(x) \in \C((x))$ have coefficients $b_k, c_k$, $d_k$ respectively, then:
    \begin{itemize}
        \item $d(f, g) = 2^{-k} \geq 0$, which $d(f, g) = 0$ if and only if $k = \infty$ if and only if $f = g$, so positive-definiteness is satisfied.
        \item Since $b_k-c_k = 0 \iff c_k-b_k = 0$, it follows immediately that $d(f,g) = d(g, f)$.
        \item Suppose $b_{k}$, $c_{j}$, and $d_{\ell}$ are the non-zero coefficients of smallest index. If $k = \ell$, then $d(f,h) \leq 2^{-k}$ and $d(f, g), d(g, h) \geq 2^{-k}$, so $\max(d(f,g),d(g,h)) \geq d(f,h)$ and the strong triangle inequality holds. If $k \neq \ell$, then either $k \neq j$ and $j \neq \ell$, so $d(f,g) \geq 2^{-\min(k,j)} \geq 2^{-k}$ and $d(g,h) \geq 2^{-\min(j,\ell)} \geq 2^{-\ell}$. Therefore, $\max(d(f,g),d(g,h)) \geq 2^{-\min(k,\ell)} \geq d(f,h)$, so the strong triangle inequality holds in all cases.
    \end{itemize}
\end{exmp}

\begin{prop}
    Let $(M, d)$ be an ultrametric space.
    \begin{itemize}
        \item If $d(x,y) \neq d(y,z)$, then $d(x,z) = \max(d(x,y),d(y,z))$.This is known as the isosceles triangle principle.
        \item If $y \in B_{r}(x)$, then $B_r(y) = B_r(x)$. This says every point in an open ball is a center for the ball.
        \item If $B_r(x) \intersect B_s(y) \neq \emptyset$, then one ball is contained within the other.
        \item Every closed ball is open, and every open ball is closed.
    \end{itemize}
\end{prop}

\begin{proof}\proofbreak
    \begin{itemize}
        \item Without loss of generality, assume that $d(x, y) > d(y,z)$. Therefore, by the strong triangle inequality $d(x, z) \leq \max(d(x,y),d(y,z)) = d(x,y)$. Similarly, $d(x,y) \leq \max(d(x,z),d(z,y))$ but $d(x,y) > d(y,z)$ by assumption so $\max(d(x,z),d(z,y)) = d(x,z)$. Therefore, $d(x,y) \leq d(x,z)$ and so $d(x,z) = d(x,y) = \max(d(x,y),d(y,z))$.
        \item If $y \in B_r(x)$, then $d(x,y) < r$ by definition. Then for any $p \in M$, we have $d(y, p) \leq \max(d(y,x), d(x,p)) = \max(r, d(x, p))$ and similarly $d(x,p) \leq \max(r, d(y, p))$. But then $d(x, p) < r \implies d(x,p) = d(y,p)$ by the previous property, and similarly if $d(y,p) < r$ then $d(x,p) = d(y,p)$. Therefore $p \in B_r(x) \iff p \in B_r(y)$, so $B_r(x) = B_r(y)$.
        \item Without loss of generality, suppose that $r \leq s$. If $p \in B_r(x)$ and $p \in B_s(y)$, then $B_r(x) = B_r(p)$ and $B_s(p) = B_s(y)$. Since $r \leq s$ we know $B_r(p) \subseteq B_s(p)$, and so $B_r(x) \subseteq B_s(y)$.
        \item Let $B_r(x)$ be an open ball. Suppose $y$ is a limit point, so every $B_s(y)$ intersects $B_r(x)$. Since this holds in particular for $s < r$, by the previous property we know $B_s(y) \subseteq B_r(x)$ and so $y \in B_r(x)$. Therefore, $B_r(x)$ contains all its limit points, and so is closed.

        Let $C_r(x)$ be a closed ball, and consider any $y \in C_r(x)$ so $d(y, x) \leq r$. Considering $B_{r/2}(y)$, we see that $p \in B_{r/2}(y) \implies d(p, y) < r/2$, and so $d(p, x) \leq \max(d(p, y), d(y, x)) \leq \max(r/2, r) = r$. Therefore, $p \in C_r(x)$, and so $B_{r/2}(y) \subseteq C_r(x)$. It follows that all points in $C_r(x)$ are interiors points, and so $C_r(x)$ is open.
    \end{itemize}
\end{proof}

\section{Quotient Topology}

\begin{defn}
    Let $X$ and $Y$ be topological spaces, and $q: X \to Y$ be a surjective map. We say that $q$ is a \emph{quotient map} when $U \subseteq Y$ is open if and only if $q^{-1}(U) \subseteq X$ is open.
\end{defn}

\begin{defn}
    If $X$ is a topological space, $A$ is a set, and $q: X \to A$ a surjective map, then there is exactly one topology on $A$, called the \emph{quotient topology} induced by $q$, such that $q$ is a quotient map.
\end{defn}

\begin{rmk}
    Given a quotient map $q: X \to Y$, we can view $Y$ as the set of equivalence classes of $X$ under the relation that $a \sim b$ when $q(a) = q(b)$.
\end{rmk}

\begin{thm}
    Let $p: X \to Y$ be a quotient map, and let $g: X \to Z$ be a map that is constant on each set $p^{-1}(\{x\})$ for $x \in Y$. Then we can factor $g$ as $g = f \circ p$ for a unique map $f: Y \to Z$. Furthermore, $f$ is continuous if and only if $g$ is continuous, and similarly $f$ is a quotient map if and only if $g$ is.
\end{thm}

\begin{proof}
    If $f(p(x)) = g(x)$, then $f(y)$ must equal $g(x)$ for all $x \in p^{-1}(y)$. Since $g$ is constant on $p^{-1}(y)$, this gives us a well-defined function.

    Note that for any $U \subseteq Z$, we have $g^{-1}(U) = p^{-1}(f^{-1})(U)$, so $g^{-1}(U)$ is open if and only if $f^{-1}(U)$ is, and the conditions of continuity and quotient map for $f$ and $g$ become equivalent.
\end{proof}

\section{Connectedness}

\begin{defn}
    A \emph{separation} of a topological space $X$ is a pair of disjoint open subsets $U$ and $V$ such that $U \union V = X$. We say that $U$ and $V$ \emph{disconnect} $X$ if they form a separation.
\end{defn}

\begin{defn}
    A topological space is \emph{disconnected} if it has a separation, and \emph{connected} otherwise.
\end{defn}

\begin{prop}
    The connected components of a space form a partition of that space.
\end{prop}

\begin{proof}
    {\color{red}TODO}
\end{proof}

\begin{prop}
    Let $X$ be a connected space, $Y$ a space with the discrete topology, and $f: X \to Y$ a function. If $f$ is continuous, it must be constant.
\end{prop}

\begin{proof}
    Let $p \in f(X)$. Then $f^{-1}(\{p\})$ and $f^{-1}(f(X)\setminus \{p\})$ are preimages of open sets and therefore are disjoint open sets in $X$. Since $f^{-1}(\{p\})$ is non-empty, either $f$ is the constant function $f(x) = p$, or else $f^{-1}(f(X)\setminus \{p\})$ is also non-empty, in which case $f^{-1}(\{p\})$ and $f^{-1}(X\setminus\{p\})$ is a separation of $X$.
\end{proof}

\begin{prop}\label{prop:continuous-image-connectedness}
    Let $f: X \to Y$ be a continuous function between topological spaces. If $X$ is connected, then so is $f(X)$.
\end{prop}

\begin{proof}
    Let $U$, $V$ be a separation of $f(X)$. Then $f^{-1}(U)$ and $f^{-1}(V)$ are necessarily disjoint and non-empty since $f$ is a function, and also open since $f$ is continuous. Since $U \union V = f(X)$, we know $f^{-1}(U) \union f^{-1}(V) = X$, and so $X$ has a separation.
\end{proof}

\begin{thm}\label{thm:connected-tools}
    Consider an arbitrary topological space $X$.

    \begin{enumerate}[label=(\arabic*)]
        \item Consider disjoint open subsets $U$, $V$ of $X$. If $A \subseteq X$ is connected and $A \subseteq U \union V$, then either $A \subseteq U$ or $A \subset V$.
        \item If $A \subseteq X$ is connected, and $A \subseteq B \subseteq \bar{A}$, then $B$ is also connected.
    \end{enumerate}
\end{thm}

\begin{proof}\proofbreak
    \begin{enumerate}[label=(\arabic*)]
        \item Suppose $A$ is not entirely contained within either $U$ or $V$. Therefore $A \intersect U$ and $A \intersect V$ are open in the subspace topology, disjoint, non-empty, and $(A \intersect U) \union (A \intersect V) = A$, and so $A$ is not connected.
        \item Suppose $B = U \union V$ is a separation. Since $A \subseteq B$ is connected, so without loss of generality we have $A \subseteq U$ by (1). Therefore, $\bar{A} \subseteq \bar{U}$. Since $B \subseteq \bar{A}$, and we know $\bar{U}$ and $V$ are disjoint, it follows $V$ is empty.
    \end{enumerate}
\end{proof}

\begin{thm}\label{thm:connected-constructs}\proofbreak
    \begin{enumerate}[label=(\arabic*)]
        \item Let $X$ be a topological space, and $\{B_{\alpha}\}_{\alpha\in A}$ be a collection of subsets of $X$ that are connected and all share a point $p$ in common. Then their union is also connected.
        \item The Cartesian product of finitely many connected topological spaces is connected.
        \item Any quotient space of a connected space is connected.
    \end{enumerate}
\end{thm}

\begin{proof}\proofbreak
    \begin{enumerate}[label=(\arabic*)]
        \item Consider any disjoint open subsets $U$, $V$ such that $\bigunion_{\alpha\in A}B_{\alpha} \subseteq U \union V$. Suppose $p \in U$. By Theorem \ref{thm:connected-tools}, it follows that the union of $B_{\alpha}$ is entirely contained within either $U$, or within $V$.
        \item Let $X$, $Y$ be topological spaces, and suppose $U$, $V$ are disjoint open subsets of $X \times Y$. Consider some $(x_0, y_0) \in X \times Y$, and without loss of generality suppose $(x_0, y_0) \in U$. Since $(x_0, y_0) \in \{x_0\} \times Y$, and $\{x_0\} \times Y$ is isomorphic to $Y$ and therefore connected, it follows by Theorem \ref{thm:connected-tools} that $\{x_0\} \times Y$ is contained within $U$. For any $y \in Y$, it follows $X \times \{y\}$ is a connected space containing $(x_0, y)$, and so by the same argument $X \times \{y\} \subseteq U$. Therefore, the union $X \times Y$ is a subset of $U$, and so $V$ must be empty.
        \item A quotient space is the image of a continuous function, so if the domain is connected so is the quotient space by Proposition \ref{prop:continuous-image-connectedness}
    \end{enumerate}
\end{proof}

\begin{rmk}
    In fact, the Cartesian product $\prod_{\alpha\in A}X_{\alpha}$ of \emph{arbitrarily many} connected spaces is still connected under the product topology. Under the Axiom of Choice, the Cartesian product contains a fixed point $a$ with coordinates $a_{\alpha}$ for $\alpha \in A$. For any finite $K \subseteq A$, the set of points $x$ such that $\pi_{\alpha}(x) = a_{\alpha}$ for all $\alpha \not\in K$ is isomorphic to the finite product $\prod_{\alpha\in K}X_{\alpha}$, which is connected by Theorem \ref{thm:connected-constructs} (2). Since all such products contain $a$, their union is connected by Theorem \ref{thm:connected-constructs} (1). Finally, it can be shown the closure of this union is equal to $\prod X_{\alpha}$ under the product topology, and so $\prod X_{\alpha}$ is connected by Theorem \ref{thm:connected-tools} (2).
\end{rmk}

\begin{defn}
    A linearly ordered set $L$ with at least two distinct elements is a \emph{linear continuum} if it has the least upper bound property, and for any $x < y$ there exists $z$ such that $x < z < y$. A subset $S$ of a linear order is \emph{convex} if for any $a, b \in S$, we have $[a, b] \subseteq S$.
\end{defn}

\begin{lemma}\label{lemma:linear-continuum-hausdorff}
    A linear continuum $L$ with the order topology is Hausdorff.
\end{lemma}

\begin{proof}
    Consider points $x, y \in L$. Since $L$ is a linear continuum there exists $x < z < y$, and so $(-\infty, z)$ and $(z, +\infty)$ are disjoint open sets containing $x$ and $y$ respectively.
\end{proof}

\begin{rmk}
    The convex subsets are the intervals, rays, singletons, or the whole space.
\end{rmk}

\begin{prop}
    Let $L$ be a linear continuum with the order topology. In $L$, a set is connected if and only if it is convex.
\end{prop}

\begin{proof}
    Let $S \subseteq L$ be convex. If $S$ is a singleton set $\{p\}$, it is trivially connected. Otherwise, suppose for the sake of contradiction that $U$, $V$ separate $S$. Fix $a \in U$ and $b \in V$, without loss of generality we suppose $a < b$. Since $S$ is convex, we know $[a, b] \subseteq S$ so $[a, b] \intersect U$ and $[a, b] \intersect V$ is a separation of $[a, b]$. Let $c = \sup ([a, b] \intersect U)$, which exists by the least upper bound property as $[a, b] \intersect U$ is non-empty. Since $b$ is an upper bound on $[a, b]$, we know $a \leq c \leq b$ so $c \in [a, b]$ and we must have $c \in U$ or $c \in V$.

    Suppose $c \in V$. We know $[a, b] \intersect V$ is open in the subspace $[a, b]$, so there exists an interval $(d, c] \subseteq V \intersect [a, b]$. Since $U$ does not intersect $V$, and $c$ upper bounds $U \intersect [a, b]$, it follows $d$ also upper bounds $U \intersect [a, b]$ but $d < c$, which is a contradiction.

    Suppose $c \in U$, then since $[a, b] \intersect U$ is open, there exists $[c, e) \subseteq U \intersect [a, b]$. Since $L$ is a linear continuum, there exists $c < z < e$, so $c$ fails to upper bound $z \in U \intersect [a, b]$.

    We have shown that in either case, we reach a contradiction, so $S$ must be connected. Now conversely, suppose $S$ is convex and let us show it is convex. We again proceed by contradiction, and suppose it is not convex. Then there must exist $a < c < b$ such that $a, b \in S$ but $c \not\in S$. It follows that $S \intersect (-\infty, c)$ and $S \intersect (c, \infty)$ disconnect $S$.
\end{proof}

\begin{thm}
    Let $X$ be a connected topological space, $Y$ be linearly ordered set with the order topology, and $f: X \to Y$ be a continuous map. Consider any $p, q \in X$ such that $f(p) \leq f(q)$. For any $c \in Y$ such that $f(p) \leq c \leq f(q)$, there exists $x \in X$ such that $f(x) = c$.
\end{thm}

\begin{proof}
    Note that $A = f(X) \intersect (-\infty, c)$ and $B = f(x) \intersect (c, \infty)$ are open in the subspace topology of $f(X)$, are disjoint, and are non-empty as they contain $f(p)$ and $f(q)$ respectively. If we suppose, for the sake of contradiction, that $c \not\in f(X)$, it follows that $A \union B = f(X)$ and so $f(X)$ is disconnected. But by Proposition \ref{prop:continuous-image-connectedness} $f(X)$ is connected, and so we must have $c \in f(X)$.
\end{proof}

\begin{defn}
    Let $X$ be a topological space. For $p, q \in X$, a \emph{path} from $p$ to $q$ is a continuous function from $f: [a, b] \subset \R \to X$ such that $f(a) = p$ and $f(b) = q$.

    We say that $X$ is \emph{path-connected} if there is a path between any two points $p, q \in X$.
\end{defn}

\begin{thm}
    Any path connected topological space is also a connected space.
\end{thm}

\begin{proof}
    Let $X$ be a path-connected space, and suppose for the space of contradiction it has a separation $U$, $V$. Therefore, there exists $p\in U$ and $q \in V$. Since $X$ is path-connected, there exists a continuous path $f: [a, b] \to X$ from $p$ to $q$. Since $f$ is continuous and $[a, b]$ is connected, we know by Proposition \ref{prop:continuous-image-connectedness} that $f([a, b])$ is connected, and so by Theorem \ref{thm:connected-tools} that $f([a, b])$ is entirely contained within $U$ and $V$. But $f(a) = p \in U$ and $f(b) = q \in V$, so we have reached a contradiction.
\end{proof}

\section{Compactness}

\begin{defn}
    A \emph{cover} of $S \subseteq X$ is a collection of sets $\mathcal{A}$ such that $X = \bigunion \mathcal{A}$. It is an \emph{open} cover if all $U \in \mathcal{A}$ are open sets. A \emph{subcover} is a subset of the collection whose union is still all of $X$.
\end{defn}

\begin{defn}
    A topological space $X$ is \emph{compact} if every open cover of $X$ has a finite subcover.
\end{defn}

\begin{prop}
    If $X_{i}$ be a finite family of compact spaces, then $\bigunion_{i=1}^{n}X_i$ is compact.
\end{prop}

\begin{proof}
    Let $\mathcal{A}$ be any open cover of $\bigunion X_i$. Then $\{ U \intersect X_i : U \in \mathcal{A} \}$ is an open cover of $X_i$, and so has a finite subcover $\{U_1 \intersect X_i, \ldots, U_k \intersect X_i\}$. Let $\mathcal{A}_i$ be the set of $U_i$ corresponding to this finite subcover of $X_i$, and let $\mathcal{A}' = \bigunion_{i} \mathcal{A}_i$. Since $\mathcal{A}'$ is a finite union of finite sets, it is finite. And of course $X_i \subseteq \union_{U \in \mathcal{A}_i}U$, so $\bigunion X_i \subseteq \bigunion_{U \in \mathcal{A}'}U$, so $\mathcal{A}'$ is a finite subcover of $\mathcal{A}$, and so $\bigunion X_i$ is compact.
\end{proof}

\begin{prop}\label{prop:continuous-image-compact}
    Let $X$ be a compact space, and $f: X \to Y$ be a continuous function. Then $f(X)$ is compact.
\end{prop}

\begin{proof}
    Let $\mathcal{A}$ be any open cover of $f(X)$. Then $f^{-1}(U)$ is open for all $U \in \mathcal{A}$, and so $\{f^{-1}(U): U \in \mathcal{A}\}$ is an open cover of $X$. Since $X$ is compact, there is a finite subcover $\mathcal{B}$. Finally, $\mathcal{A}' = \{f(U) : U \in \mathcal{B}\}$ is a finite subcover of $\mathcal{A}$, so $f(X)$ is compact.
\end{proof}

\begin{prop}\label{prop:compactness}Let $X$ be a topological space.\proofbreak
    \begin{enumerate}
        \item If $X$ is compact, every closed subset of $X$ is compact.
        \item If $X$ is a metric space, every compact subset is bounded.
        \item If $X$ is compact, every quotient space of $X$ is compact.
    \end{enumerate}
\end{prop}

\begin{proof}\proofbreak
    \begin{enumerate}
        \item If $C \subseteq X$ is closed, and $\mathcal{A}$ is an open cover of $C$, then $\mathcal{A} \union (X\setminus C)$ is an open cover of $X$. Therefore it has a finite subcover, and removing $X \setminus C$ necessarily produces a finite subcover of $\mathcal{A}$.
        \item Let $\mathcal{A}$ be the open cover of all open balls of radius one, and let $\mathcal{A}'$ be a finite subcover. Let $M = \max_{B_1(p),B_2(q)\in \mathcal{A'}}d(x, y)$ be the greatest distance between the centers of any two balls in the subcover.

        Then for any $p, q \in X$, we know there are balls $B(c), B(d) \in \mathcal{A'}$ containing $p$ and $q$ respectively. Since $d(c, p), d(d, q) < 1$, it follows by the triangle inequality that
        \begin{align*}
            d(p, q) \leq d(p, c) + d(c, q) \leq d(p, c) + d(c, d) + d(d, q) < 1 + M + 1.
        \end{align*}
        Therefore $X$ is bounded.
        \item If $Y$ is a quotient space of $X$ under quotient map $q: X \to Y$, then $q(X) = Y$ so $Y$ is compact by Proposition \ref{prop:continuous-image-compact}.
    \end{enumerate}
\end{proof}

\begin{prop}\label{prop:hausdorff-compact-closed}
    Every compact subset $Y$ of a Hausdorff space $X$ is closed.
\end{prop}

\begin{proof}
    Let $x \in X \setminus Y$. If we can find a neighborhood of $x$ disjoint from $Y$, it follows that $x$ is not a limit point of $Y$, and so $Y$ is closed. Since $X$ is Hausdorff, we know that for every $y \in Y$ there exists disjoint open sets $U$ and $V$ such that $x \in U$ and $y \in V$. The set of all such $V$ are an open cover of $Y$, and so there is a finite subcover. The intersection of the corresponding $U$ is then open, disjoint from $Y$, and contains $x$.
\end{proof}

\begin{lemma}Tube lemma\label{lemma:tube}\proofbreak
    Let $X$ be any topological space, and let $Y$ be a compact space. For $x \in X$ and open $U \subseteq X \times Y$ containing $\{x\} \times Y$, there is an open neighborhood $V \subseteq X$ of $x$ such that $V \times Y \subseteq U$.
\end{lemma}

\begin{proof}
    We know the set $W \times Z$ for open $W \subseteq X$ and open $Z \subseteq Y$ is a basis for the product topology. Let $\mathcal{A}$ be the set of all open subsets $W \times Z \subseteq U$ such $x \in W$, and consider $\mathcal{B} = \{ Z : W \times Z \in \mathcal{A} \}$. We note $\mathcal{B}$ is an open cover of $Y$, and so has a finite subbasis. Let $\{W_i\}$ be the set of corresponding open subsets of $U$. Since $x \in W_i$, we know both that $\bigintersection_{i=1}^{n}W_i$ is an open neighborhood of $x$ such that
    \begin{align*}
        \left(\bigintersection_{i=1}^{n}W_i\right) \times Y \subseteq U.
    \end{align*}
\end{proof}

\begin{thm}
    If $X$ and $Y$ are compact spaces, then $X \times Y$ is compact.
\end{thm}

\begin{proof}
    Let $\mathcal{A}$ be an open cover of $X \times Y$. For every $x \in X$, we know $\{x\} \times Y$ is compact and so many be covered by finitely many elements $\{A_i\}$ of $\mathcal{A}$. Then by the tube lemma \ref{lemma:tube} we know $\bigunion\{A_i\}$ contains a tube $W_x \times Y$ containing $\{x\} \times Y$. Then the set of all such $W_x$ is an open cover of $X$, and so contains a finite subcover $\{W_i\}$. By the above, each $W_i \times Y$ is covered by a finite subset of $\mathcal{A}$ and so the union $\bigunion_{i}W_i \times Y$ can be covered by a finite subset of $\mathcal{A}$. But $\bigunion_{i}W_i \times Y = X \times Y$, and so $\mathcal{A}$ has a finite subcover.
\end{proof}

\begin{cor}\label{cor:finite-product-compact}
    A finite product of compact spaces is compact.
\end{cor}

\begin{lemma}\label{lemma:closed-bounded-interval-compactness}
    Let $L$ be a linear continuum, considered under the order topology. Then any closed and bounded interval in $L$ is compact.
\end{lemma}

\begin{proof}
    Consider $[a, b] \subseteq L$, and let $\mathcal{A}$ be an open cover of $[a, b]$. Let
    \begin{align}
        \mathcal{U} = \{x \in [a, b] : [a, x] \textrm{ has a finite subcover of }\mathcal{A}\}
    \end{align}
    Note that $a$ must be covered by some element of $\mathcal{A}$, so $a \in \mathcal{U}$. Therefore $\mathcal{U}$ is non-empty and so $c = \sup U$ exists. Furthermore, $a$ is contained in a basis element $(x, y)$ so there exists $a < z < y$, and so $c > a$. Since $a < c \leq b$, we know $c \in [a, b]$ so $c \in U$ for some $U \in \mathcal{A}$. Since $[a, b]$ is a subspace of a linear continuum, there exists $B = (x, y)$ such that $c \in B \subseteq U$, and so $c \in (x, c] \subseteq U$. It follows that $x \in \mathcal{U}$, so $[a, x]$ has a finite cover $\{A_i\}$. But then $\{A_i\}\union \{U\}$ is a finite cover of $[a, c]$, so $\sup \mathcal{U} \in \mathcal{U}$.

    We now consider the two possibilities: $c < b$ or $c = b$. If $c < b$ there exists $c \in (x, y)$ and $c < z < y$, but then $z \in \mathcal{B}$, contradicting the definition of $c$. Therefore, $c = b$, and so $[a, b]$ has a finite subcover of $\mathcal{A}$.
\end{proof}

\begin{rmk}
    The hypothesis of the above lemma can be relaxed from a linear continuum to a linear order with the least upper bound property, however this complicates the proof.
\end{rmk}

\begin{thm}Heine-Borel\label{thm:heine-borel-topology}\proofbreak
    For any $E \subseteq \R^{n}$, $E$ is compact if and only if it is closed and bounded.
\end{thm}

\begin{proof}
    Since $\R^n$ is a metric space, every compact set is bounded by Proposition \ref{prop:compactness}, and every compact set is closed by Proposition \ref{prop:hausdorff-compact-closed}.

    If $E$ is closed and bounded, there exists $m > 0$ such $d(p, 0) < m$ for all $p \in E$. Therefore, $E \subseteq [-m, m]^n$. By Lemma \ref{lemma:closed-bounded-interval-compactness} we know $[-m, m] \subseteq \R$ is compact, and so $[-m, m]^n$ is compact by Corollary \ref{cor:finite-product-compact}. Finally, $E$ is compact by Proposition \ref{prop:compactness} since it is a closed subset of the compact space $[-m, m]^n$.
\end{proof}

\begin{thm}{Extreme value theorem}\label{thm:extreme-value}\proofbreak
    Let $f: X \to Y$ be a continuous map, where $X$ is compact and $Y$ has the order topology. Then there exist $m, M \in X$ such that $f(m) \leq f(x) \leq f(M)$ for all $x \in X$.
\end{thm}

\begin{proof}
    Since $f$ is continuous and $X$ is compact, we know $f(X)$ is compact by Proposition \ref{prop:continuous-image-compact}. {\color{red}TODO}
\end{proof}

{\color{red}TODO: limit point compactness, sequential compactness, extreme value theorem, }

{\color{red}todo: one-point compactification}

\section{Countability and separability}

\begin{defn}
    For a topological space $X$ and $p \in X$, we say $\mathcal{B}$ is a neighborhood basis at $p$ if $\mathcal{B}$ is a set of neighborhoods of $p$ such that \emph{any} neighborhood of $p$ contains at least one $B \in \mathcal{B}$.
\end{defn}

\begin{defn}
    A topological space $X$ is \emph{first countable} when every $p \in X$ has a countable neighborhood basis.
\end{defn}

\begin{prop}
    Every metric space is first countable.
\end{prop}

\begin{proof}
    For each $p \in X$, the open balls $B_{1/n}(p)$ for $n \in \Z^+$ form a countable neighborhood basis of $p$.
\end{proof}

\begin{thm}
    Let $X$ be an arbitrary topological space. If $A \subseteq X$ is a subspace, and $\{x_n\} \subseteq A$ is a sequence converging to $x$ then $x \in \bar{A}$.

    If $f: X \to Y$ is a continuous function, then for every convergent sequence $x_n \to x$ it holds that $f(x_n) \to f(x)$.

    If $X$ is first countable, the converges hold. Given $x \in \bar{A}$, there exists a sequence in $A$ converging to $x$. If every convergent sequence $x_n \to x$ satisfies $f(x_n) \to f(x)$, then $f$ is continuous.
\end{thm}

\begin{proof}
    Suppose $x_n \to x$. By definition, for any open neighborhood $U$ of $x$ there exists $N$ such that $x_n \in U$ for all $n \geq N$. Therefore, every neighborhood of $x$ contains a point $x_N$ of $A$, and so $x \in \bar{A}$ by Theorem \ref{thm:closure-neighborhood-characterization}.

    Suppose $f$ is continuous, and $x_n \to x$. Then for any open neighborhood $U$ of $f(x)$, we know $f^{-1}(U)$ is an open neighborhood of $x$. Since $x_n \to x$, there must exist $N$ such that $x_n \in f^{-1}(U)$ for all $n \geq N$, and so $f(x_n) \in U$ for all $n \geq N$. It follows that $f(x_n) \to f(x)$.

    Now we prove the first converse, adding the assumption that $X$ is first countable. Let $\mathcal{B}$ be a countable neighborhood basis of $x \in \bar{A}$. First $\mathcal{B}$ is countable, it can be enumerated as $B_0, B_1, \ldots$. Let $C_0 = B_0$, and $C_k$ be an open neighborhood of $x$ contained within $C_{k-1} \intersect B_k$. Note that $C_{k+1} \subseteq C_k$, and so $x_m \in C_k$ for all $m \geq k$. Since $x \in \bar{A}$, we know that each $C_k$ must contain a point $x_k$ of $\bar{A}$. Then for any open neighborhood $U$ of $x$, by definition of a neighborhood basis we know $U$ contains some $B_N \in \mathcal{B}$. Since $C_N \subseteq B_N \subseteq U$, and $x_n \in C_N$ for all $n \geq N$, it follows that $x_n \in U$ for all $n \geq N$. Therefore $x_n \to x$.

    Suppose every convergent sequence $x_n \to x$ satisfies $f(x_n) \to f(x)$. Let $A$ be any subset of $X$, we will show $f(\bar{A}) \subseteq \bar{f(A)}$ which will imply $f$ is continuous by Theorem \ref{thm:continuity-characterizations}. For any $x \in \bar{A}$, we know by the first converse that there exists a convergent sequence $x_n \to x$, and by hypothesis we know that $f(x_n) \to f(x)$. But $f(x_n) \in f(A)$, and so by the first half of this theorem it follows that $f(x) \in \bar{f(A)}$.
\end{proof}

\begin{defn}
    A topological space $X$ is \emph{second countable} when there exists a countable basis for $X$.
\end{defn}

\begin{prop}
    Second countability implies first countability.
\end{prop}

\begin{exmp}
    $\R^n$ is second countable.

    $\R^{\omega}$ under the uniform metric $d(x, y) = \sup_{i}\min(1, |x_i-y_i|)$ is first countable but not second countable.
\end{exmp}

\begin{thm}
    Subspaces and countable products of first countable spaces (respectively second countable) are themselves first countable (respectively second countable).
\end{thm}

\begin{defn}
    A subset $A$ of a topological space $X$ is \emph{dense} in $X$ if $\bar{A} = X$. If there exists a countable dense subset, we say that $X$ is separable.
\end{defn}

\begin{thm}
    Let $X$ be second countable. Then $X$ is Lindel\"of (every open cover of $X$ admits a countable subcover) and separable.
\end{thm}

\begin{proof}
    Let $\{B_n\}$ be a countable basis for $X$, and let $\mathcal{A}$ be any open cover of $X$. For each $n$, let $A_n \in \mathcal{A}$ be a set containing $B_n$, if such a set exists. Then $\{A_n\}$ is clearly countable, and it covers $X$: for any $p \in X$, there exists $A \in \mathcal{A}$ such that $p \in A$ since $A$ is an open cover. Since $\{B_n\}$ is a basis and $A$ is open, there exists $B_n$ such that $p \in B_n \subseteq A$, and so $p \in A \in \{A_n\}$.

    For each basis element $B_n$, let $x_n \in B_n$ be an arbitrary point, and let $D = \{x_n\}$. Clearly $D$ is countable, and it is also dense in $X$: if $p \in X$, then every basis element containing $p$ must contain some $x_n$, and so $p \in \bar{D}$.
\end{proof}

\begin{defn}
    Let $X$ be a topological space such that every singleton is closed.

    $X$ is said to be regular if for every $p \in X$ and closed $B \subseteq X$ where $p \not\in B$, there exists disjoint open $U, V$ containing $p$ and $B$ respectively.

    $X$ is said to be normal if for every pair of disjoint closed sets $A, B \subseteq X$, there exists disjoint open sets $U, V$ containing $A$ and $B$ respectively.
\end{defn}

\begin{thm}
    Let $X$ be a topological space. If $X$ is normal, then it is regular. If $X$ is regular, then it is Hausdorff.
\end{thm}

\begin{thm}\label{thm:topo-regular-normal-characterization}
    $X$ is regular if and only if for all $p \in X$ and neighborhood $U$ of $p$, there exists a neighborhood $V$ of $p$ such that $\bar{V} \subseteq U$.

    $X$ is normal if and only if for all closed $A$ and open $U$ containing $A$, there exists open $V$ such that $A \subseteq V$ and $\bar{V} \subseteq U$.
\end{thm}

\begin{proof}
    Suppose $X$ is regular, and consider some $p \in X$ and neighborhood $U$ of $p$. Let $B = X \setminus U$, which is closed and does not contain $p$. Since $X$ is regular, there exists disjoint open $V, W$ such that $p \in V$ and $B \subseteq W$. Since $V \subseteq (X \setminus W)$ and $X \setminus W$ is closed, we also know that $\bar{V} \subseteq (X \setminus W)$. Furthermore, $B \subseteq W$ implies $X \setminus W \subseteq X \setminus B = U$, and so we conclude $p \in V$ and $\bar{V} \subseteq U$.

    Conversely, suppose $B \subseteq X$ is closed and does not contain $p \in X$. Then $X\setminus B$ is open and does contain $p$, so by hypothesis there exists $V$ such that $p \in V$ and $\bar{V} \subseteq X \setminus B$. Then $X \setminus \bar{V}$ is open, contains $B$, and is disjoint from $V$.

    The proof for when $X$ is normal proceeds by effectively the same argument.
\end{proof}

\begin{prop}
    Subspaces and products of Hausdorff (respectively regular) spaces are themselves Hausdorff (respectively regular), but this is not true in general for normal spaces.
\end{prop}

\begin{itemize}
    \item $T_0$ (Kolmogorov)
    \item $T_1$ (Frechet),
    \item $T_2$ (Hausdorff),
    \item $T_3$ (regular),
    \item $T_{3\frac{1}{2}}$ (Tychonoff/completely regular),
    \item $T_4$ (normal)
\end{itemize}

\begin{lemma}\label{lemma:urysohn-proto}
    Let $X$ be a topological space, such that for each $q \in [0, 1] \intersect \Q$ there exists an open set $U_q \subseteq X$ and $\bar{U_q} \leq U_p$ when $q < p$. Then there exists a continuous function $f: X \to [0, 1]$ such that
    \begin{itemize}
        \item $f(x) \leq q$ if $x \in U_q$,
        \item $f(x) = 1$ if $x \in U_1$.
    \end{itemize}
\end{lemma}

\begin{proof}
    Note that $\mathcal{S} = \{[0, a) : a \in [0, 1]\} \union \{(b, 1] : b \in [0, 1]\}$ is a subbasis for the standard topology on $[0, 1]$, because any element of the subbasis is already a basis element, and all other basis elements are of the form $(a, b)$ which is simply the intersection of $[0, a)$ with $(b, 1]$.

    Consider $f: X \to [0, 1]$ given by $f(x) = 1$ if $x \not\in U_1$ and $f(x) = \inf \{q : x \in U_q\}$ if $x \in U_1$. Then if $x \in U_q$ we know $x\in U_1$, and so $f(x) = \inf \{p : x \in U_p\} \leq q$.

    Note that if $f(x) < b$ then by construction there exists $p \in [0, 1] \intersect \Q$ such that $f(x) \leq p < b$ and $x \in U_p$. Therefore,
    \begin{align*}
        f^{-1}([0, b)) = \bigunion_{p < b}U_p,
    \end{align*}
    which is open since it is a union of open sets.

    If $f(x) > a$, then necessarily $x \not\in U_q$ for all $f(x) > q$. Since $\bar{U_p} \subseteq U_q$ for any $p < q$, it follows $x \not\in \bar{U_p}$. Conversely, if $x \not\in \bar{U_p}$, then $x \not\in U_q$ for all $q < p$, and so $f(x) \geq p$. Therefore,
    \begin{align*}
        f^{-1}((a, 1]) = \bigunion_{p > a}(X \setminus \bar{U_p}),
    \end{align*}
    since we just proved $x \in X \setminus \bar{U_p}$ implies $f(x) \geq p > a$, and that $f(x) > a$ implies $x \in X \setminus \bar{U_p}$ for any $p < f(x)$, so in particular for $p = (a + f(x))/2$ which satisfies $a < p < f(x)$. Since this is a union of open sets, it is also open.

    Therefore $f^{-1}(U)$ is open for all $U \in \mathcal{S}$, and so $f(x)$ is continuous.
\end{proof}

\begin{thm}{Urysohn's lemma}\label{lemma:urysohn}\proofbreak
    Let $X$ be a normal space, and let $A$ and $B$ be disjoint closed subsets of $X$. There exists a continuous map $f: X \to [0, 1] \subseteq \R$ such that $A \subseteq f^{-1}(\{0\})$ and $B \subseteq f^{-1}(\{1\})$.
\end{thm}

\begin{proof}
    Since $X$ is normal and $A$ and $B$ are closed and disjoint, by definition there exists an open set $U_1$ containing $A$ and disjoint from $B$ (so $B \subseteq X \setminus U_1$). Then by Theorem \ref{thm:topo-regular-normal-characterization}, there exists $U_0$ such that $A \subseteq U_0$ and $\bar{U_0} \subseteq U_1$.

    Now we will proceed by induction on $q_k \in [0, 1] \intersect \Q$, where $q_k = 0$, $q_1 = 1$, and the remainder are in no particular order. Suppose that for distinct $a, b \in [0, 1] \intersect \Q$ (where $a < b$ without loss of generality), we have $U_a, U_b$ such that $\bar{U_a} \subseteq U_b$. Then for any distinct $c \in (0, 1) \intersect \Q$ we have either $0 < c < a$, $a < c < b$, or $b < c < 1$. Let $U_p, U_q$ be $U_0, U_a$, or $U_a,U_b$, or $U_b,U_1$ respectively. Suppose that $\bar{U_p} \subseteq U_q$. By Theorem \ref{thm:topo-regular-normal-characterization}, there exists $U_c$ such that $\bar{U_p} \subseteq U_c$, and $\bar{U_c} \subseteq U_q$.

    It follows by induction that there exists $\{U_q\}$ such that $A \subseteq U_0$, $B\subseteq X\setminus U_1$, and $\bar{U_p} \subseteq U_q$ if $p < q$. By Lemma \ref{lemma:urysohn-proto}, there exists a continuous map $f: X \to [0, 1]$ such that $f(x) \leq q$ if $x \in U_q$ and $f(x) = 1$ if $x \not\in X_1$. In particular, this implies that $f(X \setminus U_1) = 1$ so $B \subseteq f^{-1}(\{1\})$, and also that $f(A) \subseteq f(U_0) = \{0\}$, so $A \subseteq f^{-1}(\{0\})$.
\end{proof}

\begin{defn}
    A topological space $X$ is said to be \emph{completely regular}, \emph{Tychonoff},or satisfies the $T_{3\frac{1}{2}}$ separation axiom, when singletons are closed in $X$, and given any $x \in X$ and closed $A \subseteq X$ not containing $x$ there exists a function $f: X \to [0, 1]$ such that $f(x) = 1$ and $f(A) = \{0\}$.
\end{defn}

\begin{prop}
    If a topological space $X$ is $T_4$ (normal), then it is $T_{3\frac{1}{2}}$ (completely regular), and if it is $T_{3\frac{1}{2}}$ then it is $T_3$ (regular).
\end{prop}

\begin{proof}
    If $X$ is normal, then it is completely regular by Urysohn's lemma \ref{lemma:urysohn} since $\{x\}$ is closed.

    If $X$ is completely regular, then by definition for any $x \in X$ and closed $A \subseteq X$ not containing $x$, we know there exists $f: X \to [0, 1]$ such that $x \in f^{-1}(\{1\})$ and $A \subseteq f^{-1}(\{0\})$. Then $f^{-1}((1/2, 1])$ and $f^{-1}([0, 1/2)]$ are disjoint open sets containing $x$ and $A$ respectively, and so $X$ is regular.
\end{proof}

\begin{rmk}
    Regular spaces are not necessarily completely regular --- the Tychonoff corkscrew is a regular Hausdorff space containing two points such that every continuous real-valued map is constant on those points.
\end{rmk}

\begin{thm}{Tietze extension theorem}\label{thm:tietze-extension}\proofbreak
    Let $X$ be a normal topological space, and $A \subseteq X$ a closed subset. Then
    \begin{enumerate}[label=(\alph*)]
        \item Given continuous $f: A \to [a, b] \subseteq \R$, there exists a continuous function $\hat{f}: X \to [a, b]$ such that $\hat{f}|_{A} = f$.
        \item Given continuous $f: A \to \R$, there exists a continuous function $\hat{f}: X \to \R$ such that $\hat{f}|_{A} = f$.
    \end{enumerate}
\end{thm}

\begin{proof}\proofbreak
    \begin{enumerate}[label=(\alph*)]
        \item For any $n \in \Z^{+}$, by Urysohn's lemma \ref{lemma:urysohn} there exists $f_n: X \to [a, b]$ such that {\color{red}TODO: complete}
        \item Since $(-1, 1)$ is homeomorphic to $\R$, it suffices to show that any continuous map $f: A \to (-1, 1)$ can be extended to a continuous map $f: X \to (-1, 1)$. Since such $f$ is also a function $f: A \to [-1, 1]$, by part (a) there exists $g: X \to [-1, 1]$ such that $g|_A = f$. Let $B = g^{-1}(\{-1, 1\})$, so $A$ and $B$ must be disjoint closed subsets of $X$. By Urysohn's lemma \ref{lemma:urysohn}, there exists a continuous function $h: X \to [0, 1]$ such that $h(A) = \{1\}$ and $h(B) = \{0\}$. It follows that $\hat{f}(x) = g(x)h(x)$ is a continuous map such that for $x \in A$ we have $\hat{f}(x) = g(x)1 = f(x)$, and for $x \not\in A$ either $\abs{g(x)} < 1$ and $h(x) \leq 1$ so $\hat{f(x)} \in (-1, 1)$, or else $\abs{g(x)} = 1$ and $h(x) = 0$ so $\hat{f(x)} = 0 \in (-1, 1)$. Therefore $\hat{f}: X \to (-1, 1)$ is a continuous function extending $f$.
    \end{enumerate}
\end{proof}

\begin{thm}
    Let $X$ be a $T_1$ topological space. The following are equivalent:
    \begin{enumerate}[label=(\arabic*)]
        \item $X$ is normal,
        \item Given disjoint closed $A, B \subseteq X$, there exists a continuous map $f: X \to [0, 1]$ such that $A \subseteq f^{-1}(\{0\})$ and $B \subseteq f^{-1}(\{1\})$.
        \item Given closed $A \subseteq X$, any continuous map $f: A \to \R$ can be extended to a continuous map $\hat{f}: X \to \R$.
    \end{enumerate}
\end{thm}

\begin{proof}
    We know (2) follows from (1) by Urysohn's lemma \ref{lemma:urysohn}, and (3) follows from (1) by the Tietze extension theorem \ref{thm:tietze-extension}.

    Suppose (2) holds, so given $A, B \subseteq X$ disjoint open sets and the continuous map $f$ given by (2), we have $U = f^{-1}([0, 1/2))$ and $V = f^{-1}((1/2, 1])$ are disjoint open sets containing $A$ and $B$ respectively.

    ({\color{red}TODO}: (3) $\implies$ (1)).
\end{proof}

\begin{exmp}
    Peano curve, partition of unity, {\color{red}TODO}
\end{exmp}

\begin{defn}
    An imbedding of $X$ into $Y$ is a homeomorphism from $X$ to a subset of $Y$.
\end{defn}

\begin{thm}{Urysohn's metrization theorem}\label{thm:urysohn-metrization}\proofbreak
    Every second countable regular space is metrizable.
\end{thm}

\begin{proof}
    Let $\{B_n\}$ be a countable basis for $X$. For every pair $n, m$ such that $\bar{B_n} \subseteq B_m$, by Urysohn's lemma \ref{lemma:urysohn} there exists a continuous map $g_{n,m}: X \to [0, 1]$ such that $g_{n,m}(\bar{B_{n}}) = \{1\}$ and $g_{n,m}(X\setminus B_m) = \{0\}$. Then for any point $p \in X$ and neighborhood $U$ of $p$, there must exist a basis element $B_m$ such that $p \in B_m \subseteq U$. Since $X$ is regular, by Theorem \ref{thm:topo-regular-normal-characterization} there exists $B_n$ such that $p \in B_n$ and $\bar{B_n} \subseteq B_m$. Therefore, $g_{n,m}$ was constructed, so $g_{n,m}(p) = 1$ and $g_{n,m}$ vanishes outside of $U$. Note that $\abs{\{g_{n,m}\}}$ is at most $\abs{\N^2}$, so this is a countable collection. Therefore, we can re-index it as a countable collection $\{h_n\}$, and then define the countable collection $\{f_n\}$, where $f_n = h_n/n$, so that $f_n < 1/n$ since $g_{n,m} \leq 1$.

    We now show we can imbed $X$ into the metric space $(\R^{\omega}, \bar{\rho})$, where $\bar{\rho}(x, y) = \sup_{i}\max(\abs{x_i-y_i}, 1)$. Then since a subspace of a metric space is a metric space, it will follow $X$ is homeomorphic to a metric space, and so must be metrizable.

    Define $F(x) = (f_{n}(x))$. Suppose $x \neq y$, then since singletons are closed we know there exists an open neighborhood $U$ containing $x$ but not $y$. Therefore, there exists $f_n$ which is positive at $x$ but zero at $y$, so $F(x) \neq F(y)$. {\color{red}TODO: complete}
\end{proof}

\section{Fundamental Groups}

\begin{defn}
    Let $f, f': X \to Y$ be continuous. A \emph{homotopy} between $f$ and $f'$ is a continuous map $F: X \times [0, 1] \to Y$ such that $F(x, 0) = f(x)$ and $F(x, 1)= f'(x)$.

    When there exists a homotopy between $f$ and $f'$, we write $f \simeq f'$ and say $f$ is homotopic to $f'$. If $f'$ is a constant function, we say $f$ is \emph{nulhomotopic}.
\end{defn}

\begin{defn}
    Two paths $f, f': [0, 1] \to X$ with the same endpoints $x_0 = f(0) = f'(0)$ and $x_1 = f(1) = f'(1)$ are \emph{path homotopic} if there is a continuous function $F: [0, 1] \times [0, 1] \to X$ such that
    \begin{align*}
        F(s, 0) &= f(s), \\
        F(s, 1) &= f'(s), \\
        F(0, t) &= x_0, \\
        F(1, t) &= x_1.
    \end{align*}
    The first two conditions ensure $F$ is a homotopy between $f$ and $f'$, and the second two conditions ensure any ``intermediate'' path $g_{t}(s) = F(s, t)$ still have fixed endpoints $x_0$ and $x_1$.

    If $f$ and $f'$ are path homotopic, we write $f \simeq_{h} f'$.
\end{defn}

\begin{prop}
    Homotopic and path homotopic are both equivalence relations.
\end{prop}

\begin{proof}
    Taking $F(x, s) = f(x)$ is clearly a (path) homotopy from $f$ to itself, so the relations are reflexive. If $F$ is a homotopy from $f$ to $g$, then $G(x, s) = F(x, 1-s)$ is a (path) homotopy from $g$ to $f$, so the relations are symmetric. Finally, given (path) homotopies $F$ and $G$ from $f: X \to Y$ to $g: X \to Y$, and from $g: X \to Y$ to $h: X \to Y$ respectively, consider $T: X\times [0, 1] \to Y$ given by
    \begin{align*}
        T(x, s) = \begin{dcases}
            F(x, 2s), &0 \leq s \leq \frac{1}{2} \\
            G(x, 2s-1), &\frac{1}{2} < s \leq 1
        \end{dcases}.
    \end{align*}
    Since $F(x, 1) = g(x) = G(x, 0)$, $T(x, s)$ is continuous at $s = 1/2$ where $F$ and $G$ are ``glued'' together. Therefore $T$ is continuous and satisfies $T(x, 0) = F(x, 0) = f(x)$ and $T(x, 1) = G(x, 1) = h(x)$. Furthermore, if $F$ and $G$ are \emph{path} homotopies, then $T(0, t)$ is equal to either $F(0, 2t)$ or $G(0, 2t-1)$, so in either case $T(0, t) = x_0$ and similarly $T(1, t) = x_1$. Therefore the relations are transitive.
\end{proof}

\begin{defn}
    For paths $f, g: [0, 1] \to X$ such that $f(1) = g(0)$, we define $f * g$ to be the path from $f(0)$ to $g(1)$ such that
    \begin{align*}
        (f * g)(s) &= \begin{dcases}
            f(2s), &s \in [0, 1/2] \\
            g(2s-1), &s \in [1/2, 1]
        \end{dcases}.
    \end{align*}
\end{defn}

\begin{prop}
    $*$ induces a well-defined operation on path homotopy equivalence classes.
\end{prop}

\begin{proof}
    Suppose $f \simeq_p f'$ and $g \simeq_p g'$, such that $f(1) = g(0)$, where $F, G: [0, 1]^2 \to X$ are path homotopies from $f$ to $f'$ and from $g$ to $g'$ respectively. Then $H: [0, 1]^2 \to X$ given by
    \begin{align*}
        H(s, t) &= \begin{dcases}
            F(2s, t), &s \in [0, 1/2] \\
            G(2s-1, t), &s \in [0, 1'2]
        \end{dcases}
    \end{align*}
    is a path homotopy from $f * g$ to $f' * g'$, since it is continuous, satisfies $H(s, 0) = f * g$, $H(s, 1) = f' * g'$, and $H(0, t) = F(0, t) = x_0$, $H(1, t) = G(1, t) = x_1$.
\end{proof}

\begin{prop}
    The path product $*$ on the set of all paths in a space $X$ is a groupoid called the \emph{fundamental groupoid}.
\end{prop}

\begin{proof}
    Associativity holds because given $f, g, h: [0, 1] \to X$ such that $f(1) = g(0)$ and $g(1) = h(0)$, we have
    \begin{align*}
        (f * g) * h &= \begin{dcases}
            f(2(2s)), &s \in [0, 1/4] \\
            g(2(2s-1)), &s \in [1/4, 1/2] \\
            h(2s-1), &s \in [1/2, 1]
        \end{dcases}, \\
        f * (g * h) &= \begin{dcases}
            f(2s), &s \in [0, 1/2] \\
            g(2(2s-1)) &s \in [1/2, 3/4] \\
            h(2(2s-1)-1), &s \in [3/4, 1]
        \end{dcases}.
    \end{align*}
    Define $\ell_{a,b}(x) = (x-a)/(b-a)$, so $\ell_{a,b}(a) = 0$, $\ell_{a,b}(b) = 1$, and $\ell_{a,b}(x)$ is linear between $a$ and $b$. Then we can express
    \begin{align*}
        (f * g) * h &= \begin{dcases}
            f(\ell_{0,1/4}(s)), &s \in [0, 1/4] \\
            g(\ell_{1/4,1/2}(s)), &s \in [1/4, 1/2] \\
            h(\ell_{1/2,1}(s)), &s \in [1/2, 1]
        \end{dcases}, \\
        f * (g * h) &= \begin{dcases}
            f(\ell_{0,1/2}(s)), &s \in [0, 1/2] \\
            g(\ell_{1/2,3/4}(s)), &s \in [1/2, 3/4] \\
            h(\ell_{3/4,1}(s)), &s \in [3/4, 1]
        \end{dcases},
    \end{align*}
    and we can express one possible homotopy from $(f * g) * h$ to $f * (g * h)$ as
    \begin{align*}
        a &= 1/4 + 1/4t, \\
        b &= 1/2 + 1/4t, \\
        T(s, t) &= \begin{dcases}
            f(\ell_{0,a}(s)), &s \in [0, a] \\
            g(\ell_{a,b}(s)), &s \in [a, b] \\
            h(\ell_{b,1}(s)), &s \in [b, 1]
        \end{dcases}.
    \end{align*}
    Then $T(s, 0) = (f * g) * h$, $T(s, 1) = f * (g * h)$, $T(0, t) = f(0)$, and $T(1, t) = h(1)$, so we indeed see that $[(f * g) * h] = [f * (g * h)]$.

    For $x \in X$, define $e_x: [0, 1] \to X$. Then $e_{f(0)} * f = f$ and $f * e_{f(1)} = f$ for all paths $f$.

    For any $f: [0, 1] \to X$, define $f^{-1} = f(1-s)$. Then $f * f^{-1}$ is path homotopic to $e_{f(0)}$ and $f^{-1} * f$ is path homotopic to $e_{f(1)}$, via the homotopy
    \begin{align*}
        F(s, t) &= \begin{dcases}
            f(2(1-t)s), &s \in [0, 1/2] \\
            f(2(1-t)(1-s)), &s \in [1/2, 1]
        \end{dcases}
    \end{align*}
    Note that at $s = 1/2$ we have $f(2(1-t)s) = f(1-t)$ and $f(2(1-t)(1-s)) = f(1-t)$. Furthermore, $F(s, 0) = f * f^{-1}$, $F(s, 1) = f(0)$, and $F(0, t) = f(0) e_{f(0)}(t) = F(1, t)$. Since $(f^{-1})^{-1} = f$, a symmetric argument proves $f^{-1} * f$ is path homotopic to $e_{f(1)}$.
\end{proof}

\begin{defn}
    Given a specified \emph{base point} $x_0$, the subgroup of the fundamental groupoid consisting of paths starting and ending at $x_0$ is called the \emph{fundamental group}.
\end{defn}

\begin{prop}
    Within a path connected component, all choices of base point $x_0$ result in the same fundamental group.
\end{prop}

\section{Covering Maps}

\begin{defn}
    Let $p: E \to B$ be a continuous surjection function, and $U \subseteq B$ an open subset. We say that $U$ is \emph{evenly covered} by $p$ if the preimage $p^{-1}(U)$ is a disjoint union of $U_{\alpha} \subseteq E$ such that the restriction of $p$ to any \emph{single} $U_{\alpha}$ is a homeomorphism between $U_{\alpha}$ and $U$.

    If there exists an open neighborhood $U_b$ for each $b \in B$ such that $p$ evenly covers $U_b$, then $p$ is a \emph{covering map} for $B$ and $E$ is a \emph{covering space} for $B$.
\end{defn}

\begin{exmp}
    Let $X$ be any topological space, and let $E = X \times \N$. Then $p: E \to X$ given by $(x, n) \mapsto x$ is a covering map.

    Let $\varepsilon: \R \to S^1$ be the exponential map $\varepsilon(r) = \exp(2\pi i r)$.

    For any $k \in \Z^+$, the map $p: S^1 \to S^1$ be given by $z \to z^{k}$ (where we consider $S^1 \subseteq \C$) is a covering map.
\end{exmp}

\begin{rmk}
    Typically we want $E$ to be connected and locally path-connected (so $E$ is path connected), in order to avoid covering spaces like $X \times \N$ which are homeomorphic to a disjoint union of $B$, and so a covering map carries effectively no information.
\end{rmk}

\begin{thm}
    Let $p: E \to B$ be a covering map, and $B_{0} \subseteq B$. Then the restriction of $p$ to the preimage of $B_{0}$ is a covering map of $B_{0}$.
\end{thm}

\begin{proof}
    Let $q: E_{0} \to B_{0}$ be this restricted map. We know $q$ is continuous since $p$ is, and $q$ is surjective by construction. For any $b \in B_{0} \subseteq B$, there exists a neighborhood $U$ containing $b$ and evenly covered by $p$ via $\{U_{\alpha}\}$. Take $V = U \intersect B_0$, which contains $b$ and is evenly covered by $q$ via $\{U_{\alpha} \intersect E_{0}\}$.
\end{proof}

\begin{thm}\label{thm:covering-map-product}
    If $p: E \to B$ and $p': E' \to B'$ are covering maps, then so is $p \times p': E \times E' \to B \times B'$.
\end{thm}

\begin{proof}
    Let $(b, b') \in B \times B'$. We know there are neighborhoods $U$, $U'$ of $b$ and $b'$ which are evenly covered by $p$ and $p'$ via $\{U_{\alpha}\}$ and $\{V_{\alpha}\}$ respectively. Let $W_{\alpha,\beta} = U_{\alpha} \times V_{\beta}$. Then the preimage of $U \times U'$ is the disjoint union of $W_{\alpha,\beta}$, and each $W_{\alpha,\beta}$ is homeomorphic to $U \times U'$ via the restriction of $p \times p'$.
\end{proof}

\begin{exmp}
    Consider $B = \R^2 \setminus \{0\}$. We know the exponential map from $\R$ to $S^1$ is a covering map, and the identity map $\R^+ \to \R^+$ is a covering map. By THeorem \ref{thm:covering-map-product}, there is a covering map from $\R \times \R^+$ to $S^1 \times \R^+$. Since $S^+ \times \R^+$ is homeomorphic to $\R^2 \times \{0\}$ via $\varphi(x, \lambda) = \lambda x$, it follows that we have a covering map $p: \R \times \R^+ \to \R^2 \times \{0\}$ given by $p(a, b) = b\exp(2\pi i a)$.
\end{exmp}

\begin{defn}
    Let $p: E \to B$ be a covering map, and $\varphi: Y \to B$ a continuous map. A \emph{lift} $\tilde{\varphi}: Y \to E$ of $\varphi$ is a continuous function such that $p \circ \tilde{\varphi} = \varphi$.
\end{defn}

\begin{thm}{Unique lifting}\label{thm:unique-lifting}\proofbreak
    Let $\varphi: Y \to X$ be continuous, and $p: E \to X$ a covering map. Suppose $\tilde{\varphi}_{1}$ and $\tilde{\varphi}_{2}$ are two lifts of $\varphi$. If $\tilde{\varphi}_{1}(x) = \tilde{\varphi}_{2}(x)$ for any $x$, then the lifts are the same.
\end{thm}

\begin{thm}
    Let $p: E \to X$ be a covering map, and $\varphi_{0},\varphi_{1}$ be continuous maps $Y \to X$ with a homotopy $H: Y \times I \to X$ between them, then there is a unique homotopy $\tilde{H}: Y \times I \to E$ such that $\tilde{H}_{0} = \tilde{\varphi}_{0}$.

    If $H$ is a path-homotopy, then so is $\tilde{H}$.
\end{thm}

\begin{thm}
    {\color{red}TODO: add group work 3 theeorem}
\end{thm}

\section{Homotopy Equivalences}

\begin{defn}
    Let $f: X \to Y$ and $g: Y \to X$ be continuous maps such that $g \circ f$ is homotopic to $\textrm{id}_{X}$ and $f \circ g$ is homotopic to $\textrm{id}_{Y}$. Then $f$ and $g$ are \emph{homotopy equivalences}, $g$ ($f$) is the homotopy inverse of $f$ ($g$), and $X$ and $Y$ have the same \emph{homotopy type}.
\end{defn}

\begin{prop}
    Having the same homotopy type is an equivalence relation.
\end{prop}

\begin{defn}
    A subspace $A \subseteq x$ is a \emph{deformation retract} of $X$ if there exists a homotopy $H: X \times I \to X$ such that $H(\cdot, 0) = \textrm{id}_{X}$, $H(x, 1) \in A$, and $H(a, t) = a$ for all $a \in A$. That is, the identity map on $X$ can be continuously deformed into a map from $X$ to $A$, such that all points in $A$ are fixed throughout the deformation.

    We call $r(x) = H(x, 1)$ a \emph{retraction} of $X$ onto $A$.
\end{defn}

\begin{rmk}
    Let $\iota: A \to X$ be the inclusion map. Then $r \circ \iota = \textrm{id}_{A}$, and $H$ is a homotopy between $\textrm{id}_{X}$ and $\iota \circ r$. Therefore, $A$ and $X$ have the same homotopy type.
\end{rmk}

\begin{exmp}
    Let $X = \R^{n+1}\setminus \{0\}$, let $A = S^n$, and $H(x, t) = t\frac{x}{\norm{x}} + (1-t)x$.
\end{exmp}

\begin{lemma}
    Let $h, k: (X, x_0) \to (Y, y_0)$ be pointed continuous maps which are homotopic and $x_0 \mapsto y_0$ is fixed throughout the homotopy. Then $h_{*}, k_{*}: \pi_1(X, x_0) \to \pi_1(Y, y_0)$ are equal.
\end{lemma}

\begin{proof}
    Let $H: (X, x_0) \times I \to (Y, y_0)$ be the homotopy from $h$ to $k$. Let $f$ is any loop at $x_0$, then $h_*(f) = H(\cdot, 0) \circ f$ and $k_* = H(\cdot, 1)$. {\color{red}TODO}
\end{proof}

\begin{thm}
    Let $A$ be a deformation retract of $X$. The inclusion map $\iota: A \to X$ induces an isomorphism $\iota_{*}: \pi_1(A, x_0) \to \pi_1(X, x_0)$ on the fundamental groups.
\end{thm}

\begin{proof}
    ({\color{red}TODO: from homework}) We know $\iota_*$ is injective and a homomorphism of groups, so it remains only to prove it is surjective. Let $r: X \to A$ be the retraction, which we know satisfies $(\iota \circ r)_{*} = (\textrm{id}_{X})_{*} = \textrm{id}_{\pi_1(X,x_0)}$. Since $(\iota \circ r)_{*} = \iota_* \circ r_*$, it follows that $\iota_*$ must be surjective.
\end{proof}

\begin{thm}
    If $f: (X, x_0) \to (Y, y_0)$ is a pointed continuous map which is a homotopy equivalence, then $f_{*}: \pi_1(X, x_0) \to \pi_1(Y, y_0)$ is an isomorphism.
\end{thm}

\begin{proof}
    Let $g: (Y, y_0) \to (X, x_1)$ be a homotopy inverse of $f$. Consider
    % https://q.uiver.app/#q=WzAsNCxbMiwwLCJcXGxlZnQoWSx5XzBcXHJpZ2h0KSJdLFs2LDAsIihZLHlfMSkiXSxbMCwwLCJcXGxlZnQoWCx4XzBcXHJpZ2h0KSJdLFs0LDAsIlxcbGVmdChYLHhfMVxccmlnaHQpIl0sWzIsMCwiZiJdLFswLDMsImciXSxbMywxLCJmIl1d
    \tikzexternaldisable
    \[\begin{tikzcd}
        {\left(X,x_0\right)} && {\left(Y,y_0\right)} && {\left(X,x_1\right)} && {(Y,y_1)}
        \arrow["f", from=1-1, to=1-3]
        \arrow["g", from=1-3, to=1-5]
        \arrow["f", from=1-5, to=1-7]
    \end{tikzcd}\]
    \tikzexternalenable
\end{proof}
