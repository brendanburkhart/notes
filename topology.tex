\setchaptergraphic{}

\chapter{Topology}
\label{ch:topology}

\section{Topological spaces}

\begin{defn}
    A topological \emph{space} is a set $X$ of \emph{points} endowed with a \emph{topology} $\topo$. A topology for $X$ is a set of subsets of $X$ (the \emph{open sets} in the topology), such that $\emptyset \in \topo$ and $X \in \topo$ and which is closed under arbitrary unions, and finite intersections. That is, if $U_i \in \topo$ for $i = 1, \ldots, n$ then we must have
    \begin{align*}
        \bigintersect_{i=1}^{n}U_i \in \topo,
    \end{align*}
    and if $U_{a} \in \topo$ for all $a \in A$ then
    \begin{align*}
        \bigunion_{a\in A}U_a \in \topo.
    \end{align*}
\end{defn}

\begin{exmp}\proofbreak
    \begin{itemize}
        \item The \emph{trivial topology} (also \emph{indiscrete topology}) on $X$ is simply $\topo = \{\emptyset, X\}$.
        \item The \emph{discrete topology} on $X$ is $\topo = \mathcal{P}(X)$. 
    \end{itemize}
\end{exmp}

\begin{defn}
    For a given set $X$ with topologies $\topo_1$ and $\topo_2$, we say that $\topo_1$ is \emph{coarser} than $\topo_2$ when $\topo_1 \subseteq \topo_2$, and \emph{finer} than $\topo_2$ when $\topo_2 \subseteq \topo_2$. When either relation holds, we say the topologies are comparable ---  of course, many topologies are not comparable.
\end{defn}

\begin{defn}
    Given a set $X$, a set $\mathcal{B}$ is said to be a topological \emph{basis} if it satisfies the following requirements:
    \begin{itemize}
        \item For every point $x \in X$, there exists at least one $B \in \mathcal{B}$ such that $x \in B$.
        \item If $x \in B_1 \in$ and $x \in B_2$, then $\mathcal{B}$ must contain $B_3$ such that $x \in B_3 \subseteq B_1 \intersect B_2$.
    \end{itemize}
    The topology \emph{generated} by $\mathcal{B}$ consists of all sets $U \subseteq X$ such that there exists $B \in \mathcal{B}$ for each $x \in U$, and $x \in B \subseteq U$.
\end{defn}

\begin{prop}\label{prop:basis-generates-topology}
    The set $\topo$ generated by $\mathcal{B}$ is a topology on $X$.
\end{prop}

\begin{proof}
    Note that $\emptyset \in \topo$ vacously, and it follows immediately from the definition that $X \in \topo$.

    Suppose that $U_a \in \topo$ for all $a$ in some index set $A$. Let $U = \union U_a$. For every $x \in U_a$, we know that there exists $B \in \mathcal{B}$ such that $x \in B \subseteq U_a$. Since $U_a \subseteq U$ by definition, we also have $x \in B \subseteq U$ for every $x \in U$, and so $U \in \topo$.

    For any $U_1, U_2 \in \topo$, let $U' = U_1 \union U_2$. Then for $x \in U'$, there must by definition be $B_1, B_2 \in \mathcal{B}$ such that $x \in B_1 \subseteq U_1$ and $x \in B_2 \subseteq U_2$. By the definition of $\mathcal{B}$, it follows that $\mathcal{B}$ must contain $B_3$ such that $x \in B_3 \subseteq B_1 \intersect B_2$. Therefore, $x \in B_3 \subseteq U_1 \intersect U_2$, and so $U_1 \intersect U_2 \in \topo$. It follows inductively that for any $U = \bigintersect_{i=1}^{n}U_i$, we must have $U \in \topo$.

    Therefore, $\topo$ contains $\emptyset$ and $X$, and is closed under finite intersections and arbitrary unions, and so is indeed a topology on $X$.
\end{proof}

\begin{exmp}
    $\mathcal{B} = \{X\}$ is a basis for the indiscrete topology on $X$, while $X$ itself (so the set of all single points in $X$) is a basis for the discrete topology.
\end{exmp}

\begin{lemma}\label{lemma:basis-is-union-collection}
    The topology $\topo$ generated by a basis $\mathcal{B}$ is equal to the set of all possible unions of basis elements.
\end{lemma}

\begin{proof}
    Suppose $U \in \topo$. By definition, for each $x \in U$ there is some $B_x \in \mathcal{B}$ such that $x \in B_x \subseteq U$, and so
    \begin{align*}
        U = \bigunion_{x\in U}B_x.
    \end{align*}

    Given an arbitrary union $U$ of basis elements, the union is an open set in the topology generated by $\mathcal{B}$ because basis elements are in the topology, and any topology is closed under arbitrary unions.
\end{proof}

\begin{rmk}
    The following lemma is useful, as it establishes a criterion to simultaneously prove a set is a basis, and that it generates a specific topology.
\end{rmk}

\begin{lemma}\label{lemma:basis-criterion}
    Let $X$ be a topological space, and let $\mathcal{C}$ be a collection of open sets in $X$. If, for every open set $U$ and point $x \in U$, there exists $C \in \mathcal{C}$ such that $x \in C \subseteq U$, then $\mathcal{C}$ is a basis for the topology on $X$.
\end{lemma}

\begin{proof}
    By definition, for every $x \in X$, there must exist an open set $U$ such that $x \in U$. Therefore, there exists $C \in \mathcal{C}$ such that $x \in C$, so $\mathcal{C}$ satisfies the first requirement of a basis.

    If $x \in C_1$ and $x \in C_2$, then we know that $U = C_1 \intersect C_2$ must be open. Then there exists $C_3$ such that $x \in C_3 \subseteq U = C_1 \intersect C_2$, so $\mathcal{C}$ satisfies the second requirement of a basis.

    We have shown that $\mathcal{C}$ is a basis, but it remains to show that it generates $\topo$ (as opposed to a different topology on $X$). By Lemma \ref{lemma:basis-is-union-collection}, it follows that the topology generated by $\mathcal{C}$ is contained in $\topo$. Furthermore, for any $U \in \topo$ and $x \in U$ we have (by definition of $\mathcal{C}$) that there exists $C_x \in \mathcal{C}$ such that $x \in C \subseteq U$. But then $U = \union_{x\in X}C_x$, and so $\topo$ is contained by the topology generated by $\mathcal{C}$.
\end{proof}

\begin{lemma}\label{lemma:subbasis-finer-topology}
    Let $\mathcal{B}$ and $\mathcal{B}'$ be bases on a set $X$, for topologies $\topo$ and $\topo'$ respectively. Then $\topo'$ is \emph{finer} than $\topo$ if and only if for any $x \in B \in \mathcal{B}$ there exists $B' \in \mathcal{B}'$ such that $x \in B' \subseteq B$.
\end{lemma}

\begin{proof}
    If $\topo'$ is finer, then $\topo'$ must contain $B$ by definition, and so there must be some $B' \in \mathcal{B}'$ such that $x \in B' \subseteq B$ by definition of a basis for $\topo'$.

    Suppose instead that the stated condition holds on $\topo'$, and fix any $U \in \topo$. For each $x \in U$, there must by definition exist $B_x \in \mathcal{B}$ such that $x \in B_x \subseteq U$. By the stated condition, there must then exist ${B'}_x$ such that $x \in {B'}_x \subseteq B_x \subseteq U$, and so
    \begin{align*}
        U = \bigunion_{x\in U}{B'}_x.
    \end{align*}
    Therefore, $U$ is the union of elements of $\topo'$, and so is contained in $\topo'$. Since this holds for any $U \in \topo$, it follows that $\topo \subseteq \topo'$, and so $\topo'$ is finer by definition.
\end{proof}

\begin{exmp}
    Let $\mathcal{B}$ consist of all intervals $(a, b) \subseteq \R$, then the topology generated by $\mathcal{B}$ is the \emph{standard} topology on $\R$. Let $\R_{\ell}$ denote the topology generated by all intervals $[a, b) \subseteq \R$ --- this is known as the \emph{lower limit topology}. By Lemma \ref{lemma:subbasis-finer-topology}, it follows that $\R_{\ell}$ is strictly finer than the standard topology.
\end{exmp}

\begin{defn}
    Given an ordered set $X$, the \emph{order topology} on $X$ has as basis elements all sets of the form $(a, b) = \{s \in X : a < x < b\}$, and if $X$ contains smallest or largests elements $x_0, x_1$, then it also contains the sets $[x_0, b)$ and $(a, x_1]$.
\end{defn}

\begin{defn}
    Given sets $X$ and $Y$ endowed with topologies, the \emph{product topology} on $X \times Y$ has basis consisting of $U \times V$ for all $U$ open in $X$ and $V$ open in $Y$.
\end{defn}

\begin{prop}
    The product topology is a topology on $X \times Y$.
\end{prop}

\begin{proof}
    Note that $X \times Y$ is itself in the chosen basis, so it trivially satisfies the first requirement of a basis. Furthermore, if $(x, y) \in U_1 \times V_1$ and also $(x, y) \in U_2 \times V_2$, then we have $U_3$ and $V_3$ such that $x \in U_3 \subseteq U_1 \intersect U_2$ and $y \in V_3 \subseteq V_1 \intersect V_2$, from which it follows that $(x, y) \in U_3 \times V_3 \subseteq (U_1 \intersect U_2) \times (V_1 \intersect V_2) = (U_1 \times V_1) \intersect (U_2 \times V_2)$. Therefore, we do indeed have a basis for a topology on $X \times Y$.
\end{proof}

\begin{thm}
    If $\mathcal{B}$ and $\mathcal{C}$ are bases for topologies on $X$ and $Y$ respectively, then
    \begin{align*}
        \mathcal{D} = \left\{B \times C : B \in \mathcal{B}, C \in \mathcal{C}\right\}
    \end{align*}
    is a basis for the corresponding product topology on $X \times Y$.
\end{thm}

\begin{proof}
    For any point $(x, y) \in W$, where $W = U \times V$ is open in the product topology, there must exist $B \in \mathcal{B}$ and $C \in \mathcal{C}$ such that $x \in B \subseteq U$ and $y \in C \subseteq V$. Therefore, $(x, y) \in B \times C \subseteq W$, and so the conclusion follows by Lemma \ref{lemma:basis-criterion}.
\end{proof}

\begin{thm}
    Let $\topo_X$ and $\topo_Y$ be topologies on spaces $X$ and $Y$. The set
    \begin{align*}
        S = \left\{\pi_1^{-1}(U) : U \in \topo_X\right\} \union \left\{\pi_2^{-1}(V) : V \in \topo_Y\right\}
    \end{align*}
    is a subbasis for the product topology on $X \times Y$, where $\pi_1: X \times Y \to X$ and $\pi_2: X \times Y \to Y$ are the first and second projections. 
\end{thm}

\begin{proof}
    For any open set $W = U \times V$ in the product topology, we have $\pi_1^{-1}(U) = U \times Y \in S$ and $\pi_2^{-1}(V) = X \times V \in S$. Therefore $W = \pi_1^{-1}(U) \intersect \pi_2^{-1}(V)$ is a finite intersection of elements of $S$.
\end{proof}

\begin{defn}
    Let $\topo$ be a topology on $X$. Given a subset $Y$ of $X$, we can form the \emph{subspace topology}
    \begin{align*}
        \topo_Y = \left\{U \intersect Y : U \in \topo\right\}.
    \end{align*}
\end{defn}

\begin{prop}
    The subspace topology is a topology on $Y$.
\end{prop}

\begin{proof}
    Since $\emptyset \in \topo$ we have $\emptyset \in \topo_Y$ and since $X \in \topo$ we have $X \intersect Y = Y$ in $\topo_Y$. Furthermore,
    \begin{align*}
        \bigunion_{\alpha \in A}\left(U_{\alpha} \intersect Y\right) = \left(\bigunion_{\alpha \in A}U_{\alpha}\right) \intersect Y,
    \end{align*}
    so $\topo_Y$ is closed under arbitrary unions. Finally,
    \begin{align*}
        \bigintersect_{i=1}^{n}\left(U_i \intersect Y\right) = \left(\bigintersect_{i=1}^{n}U_i\right) \intersect Y,
    \end{align*}
    and so $\topo_Y$ is closed under finite intersections.
\end{proof}

\begin{lemma}\label{lemma:subspace-basis}
    Given a basis $\mathcal{B}$ for a topology $\topo$ on $X$, the set
    \begin{align*}
        \mathcal{B}_Y = \left\{B \intersect Y : B \in \mathcal{B}\right\}
    \end{align*}
    is a basis for the subspace topology $\topo_Y$.
\end{lemma}

\begin{proof}
    For any open set $V$ in the subspace topology, by definition there exists $U \in \topo$ such that $V = U \intersect Y$. Therefore, for any $x \in U$ there exists $B \in \mathcal{B}$ such that $x \in B \subseteq U$. Therefore, for any $x \in V$ there exists $B_Y = B \intersect Y$ in $\mathcal{B}_Y$ such that $x \in B_Y \subseteq V$. It follows by Lemma \ref{lemma:basis-criterion} that $\mathcal{B}_Y$ is a basis for the subspace topology.
\end{proof}

\begin{thm}
    Given a subspace $Y$ of $X$, if $Y$ is open in $X$ then all sets $U$ open in $Y$ are also open in $X$.
\end{thm}

\begin{proof}
    If $U$ is open in $Y$, then $U = V \intersect Y$ for some $V$ open in $X$. Therefore, $U$ is the intersection of two sets open in $X$, and so must also be open in $X$.
\end{proof}

\begin{lemma}
    Let $X$ be a topological space, with subsets $Y$ and $Z$ such that $Z \subseteq Y \subseteq X$. The subspace topologies of $Z$ as a subset of $X$ and as a subset of $Y$ are the same topology.
\end{lemma}

\begin{proof}
    A element of the basis of $Z$ as a subset of $X$ has the form $U \intersect Z$, and as a subset of $Y$ will have the form $(U \intersect Y) \intersect Z$. Since $Z$ is a subset of $Y$, it follows that $(U \intersect Y) \intersect Z = U \intersect Z$.
\end{proof}

\begin{thm}
    Let $A$ be a subspace of $X$ and $B$ a subspace of $Y$. The product topology of the subspace topologies on $A$ and $B$ is the same as the subspace topology of $A \times B$ of the product topology on $X \times Y$.
\end{thm}

\begin{proof}
    We know that the collection of sets $(U \times V) \intersect (A \times B)$ for all $U$ open in $X$ and $V$ open in $Y$ is a basis for the subspace topology of the product topology, by Lemma \ref{lemma:subspace-basis}.

    We also know that the product topology of the subspace topologies has a basis consisting of all sets of the form $(U \intersect A) \times (V \intersect B)$. But then, since
    \begin{align*}
        (U \intersect A) \times (V \intersect B) = (U \times V) \intersect (A \times B),
    \end{align*}
    it follows the two bases are equal, and therefore so are the topologies.
\end{proof}

\begin{defn}
    Let $X$ be an ordered set. A subset $Y$ is \emph{convex} in $X$ if $Y$ contains $(a, b) \subseteq X$ for all $a, b \in Y$.
\end{defn}

\begin{thm}
    Let $X$ be an ordered set with the order topology, and let $Y$ be a convex subset of $X$. The order topology and subspace topology on $Y$ are the same.
\end{thm}

\begin{proof}
    We will prove these topologies can both be generated by the same subbasis. Note that the set of all open rays $(a, +\infty)$ and $(-\infty, a)$ form a subbasis for the order topology, and the intersection of an open ray in $X$ with $Y$ is either an open ray in $Y$, all of $Y$, or empty. Conversely, any open ray in $Y$ is the intersection of $Y$ and an open ray in $X$, so these subbases are equal.
\end{proof}

\begin{defn}
    Letting $J$ be some index set, then a Cartesian product of sets $X_j$, denoted
    \begin{align*}
        \prod_{j \in }X_j,
    \end{align*}
    is the set of all functions $t: J \to \union_{j \in J}X_j$, such that $t(j) \in X_j$.
\end{defn}

\section{Closure and Limit Points}

\begin{defn}
    A set $A$ in the topological space $(X, \topo)$ is \emph{closed} when it's complement $X \setminus A$ is open in $\topo$.
\end{defn}

\begin{thm} Let $(X, \topo)$ be a topological space.
    \begin{itemize}
        \item Both $\emptyset$ and $X$ are closed.
        \item Arbitrary intersections of closed sets are closed.
        \item Finite unions of closed sets are closed.
    \end{itemize}
\end{thm}

\begin{proof}\proofbreak
    \begin{itemize}
        \item Both $\emptyset$ and $X$ are open, and are each other's complements, so both are closed.
        \item If $S_{\alpha}$ for $\alpha \in A$ are closed sets, then their complements are open, and so \begin{align*}
            X \setminus \bigintersection_{\alpha \in A}S_{\alpha} = \bigunion_{\alpha \in A}\left(X \setminus S_{\alpha}\right).
        \end{align*}
        Note that $X \setminus S_{\alpha}$ is open by definition, and so any arbitrary intersection of closed sets can be expressed as the complement of an arbitrary union of open sets.
        \item Similarly, finite unions of closed sets are complements of finite intersections of open sets.
    \end{itemize}
\end{proof}

\begin{lemma}\label{lemma:closed-subset-criterion}
    Let $(X, \topo)$ be a topological space, and consider the subspace topology on $Y \subseteq X$. Then $A \subseteq Y$ is closed in $\topo_Y$ if and only if $A = B \intersect Y$ for some $B$ that is closed in $\topo$.
\end{lemma}

\begin{proof}
    Suppose $A = B \intersect Y$ and $B$ is closed in $\topo$. Then $Y \setminus A$ equals $Y \setminus B = Y \intersect (X \setminus B)$. Since $X \setminus B$ is open in $\topo$ by definition, it follows that $Y \intersect (X \setminus B)$ is open in the subspace topology, and so $A$ is closed in $\topo_Y$.

    Suppose that $A$ is closed in $\topo_Y$, and so $Y \setminus A$ is open in $\topo_Y$. We then have $Y \setminus A = Y \intersect S$ for some $S$ open in $\topo$. If we let $B = X \setminus S$, it follows that
    \begin{align*}
        B \intersect Y = Y \intersect (X \setminus S) = Y \setminus S = Y \setminus (Y \intersect S).
    \end{align*}
    Of course, $Y \intersect S = Y \setminus A$ as we saw, and so $A = B \intersect Y$.
\end{proof}

\begin{lemma}
    Let $X$ be a topological space with subspace $Y$. If $A$ is closed in $Y$, and $Y$ is closed in $X$, then $A$ is closed in $X$.
\end{lemma}

\begin{proof}
    Since $A$ is closed in $Y$, then by Lemma \ref{lemma:closed-subset-criterion} $A = B \intersect Y$ where both $Y$ and $B$ are closed in $X$. Since $A$ is the intersection of closed sets, it it also closed in $X$.
\end{proof}

\begin{defn}
    In a space $X$, the \emph{interior} of $A \subseteq X$ is the union of all open sets contained in $A$, and the \emph{closure} of $A$ (denoted $\closure{A}$) is the intersection of all closed sets with contain $A$.
\end{defn}

\begin{thm}
    Consider a space $X$ with subspace $Y$, and some $A \subseteq Y$. Then the closure of $A$ in $Y$ equals $\closure{A} \intersect Y$, where $\closure{A}$ is taken with respect to $X$.
\end{thm}

\begin{proof}
    Suppose that $C$ is closed in $X$ and contains $A$. Then by Lemma \ref{lemma:closed-subset-criterion} $Y \intersect C$ is closed in $Y$, and contains $A$. Now instead suppose $C$ is closed in $Y$ and contains $A$. Then by the same lemma we have $C = Y \intersect D$ for some $D$ closed in $X$, and so $D$ is a closed set in $X$ which contains $A$.

    Let $C_{\alpha}$ be all closed sets in $Y$ which contain $A$, and then we have proven
    \begin{align*}
        \bigunion C_{\alpha} = \bigunion (Y \intersect D_{\alpha}) = Y \intersect \bigunion D_{\alpha} = Y \intersect \closure{A}.
    \end{align*}
\end{proof}

\begin{defn}
    Let $X$ be a space, and $x$ a point in that space. A \emph{neighborhood} of $x$ is an open set which contains $x$.
\end{defn}

\begin{thm}\label{thm:closure-neighborhood-characterization}
    Let $X$ be a space, and consider some $A \subseteq X$. A point $x \in A$ is in $\closure{A}$ if and only if $U \intersect A$ is non-empty for all neighborhoods $U$ of $x$.
\end{thm}

\begin{proof}
    Suppose there exists some neighborhood $U$ of $x$ such that $U \intersect A$ is empty. Then $X \setminus U$ is a closed set containing $A$, but not $x$, and so $x \not\in \closure{A}$.

    Conversely, if $x \not\in \closure{A}$, there must exist a closed set which contains $A$ but not $x$. Let $U$ be the open complement of that closed set. It follows that $A \intersect U$ is empty, and since $x \in U$ we know $U$ is a neighborhood of $x$.
\end{proof}

\begin{cor}
    If $X$ has a basis $\mathcal{B}$, then a point $x \in A$ is in $\closure{A}$ if and only if $B \intersect A$ is non-empty for all $B \in \mathcal{B}$ that contain $x$.
\end{cor}

\begin{proof}
    If $x \in \closure{A}$, then all basis elements containing $x$ have non-empty intersection with $A$ by Theorem \ref{thm:closure-neighborhood-characterization}, since such basis elements are neighborhoods of $x$.

    If every basis element containing $x$ has non-empty intersection with $A$, then so does every neighborhood of $x$, since for every neighborhood $U$ there is a basis element $B$ such that $x \in B \subseteq U$. Therefore, $x \in \closure{A}$ by Theorem \ref{thm:closure-neighborhood-characterization}.
\end{proof}

\begin{lemma}\label{lemma:closure-closed}
    A set is closed if and only if it equals its closure, and is open if and only if it equals its interior.
\end{lemma}

\begin{defn}
    Let $X$ be a topological space, and $A$ an subset of $X$. A point $p \in X$ is called a \emph{limit point} of $A$ if every neighborhood of $p$ contains some $q \in A$ such that $p \neq q$.
\end{defn}

\begin{thm}\label{thm:closure-limit-points}
    Consider a topological space $X$ with subset $A$, and let $A'$ denote the limit points of $A$. Then $\closure{A} = A \union A'$.
\end{thm}

\begin{proof}
    If $x \in A'$, then every neighborhood of $x$ must intersect $A$, and so by Theorem \ref{thm:closure-neighborhood-characterization} we know $x \in \closure{A}$ and so $A' \subseteq \closure{A}$. Since $A \subseteq \closure{A}$, it follows that $A \union A'$ is a subset of $\closure{A}$.

    Now suppose $x \in \closure{A}$. Since every neighborhood of $x$ must intersect $A$, if $x \not\in A$ then each neighborhood must contain at least one point in $A$ which is necessarily distinct from $x$ (since $x \not\in A$), and so $x$ would be a limit point of $A$. Therefore, $x \in A$ or $x \in A'$, and so $\closure{A} \subseteq A \union A'$.
\end{proof}

\begin{cor}\label{cor:closed-iff-self-closure}
    $A$ is closed if and only if $A' \subseteq A$.
\end{cor}

\begin{defn}
    Consider a topological space $X$. If, for any $p, q \in X$ there exist neighborhoods of $p$ and $q$ that are disjoint, then we say that $X$ is \emph{Hausdorff}.
\end{defn}

\begin{lemma}
    Let $X$ be a Hausdorff space with subset $A$. If $A$ is finite, then it is closed.
\end{lemma}

\begin{proof}
    Consider the closure of a singleton set $\{x\}$. Given any point $y \neq x$, we know that $y$ and $x$ have disjoint neighborhoods $U$ and $V$, which implies that $U$ contains $y$ but not $x$. Then by Theorem \ref{thm:closure-neighborhood-characterization}, the closure of $\{x\}$ cannot contain $y$, and so the closure of $\{x\}$ is simply itself. By Corolllary \ref{cor:closed-iff-self-closure}, it follows that $\{x\}$ is closed.

    Any finite $A$ is therefore a finite union of closed sets, and so is closed.
\end{proof}

\begin{exmp}
    Consider the finite complement topology on $\R$, where sets are open when they have finite complement. Any finite set is of course closed. However, this space is \emph{not} Hausdorff, since no open sets can possibly be disjoint in this space.
\end{exmp}

\begin{defn}
    Let $\{x_n\}$ be a sequence of points in a topological space. The sequence \emph{converges} to a point $x$ if for every neighborhood $U$ of $x$ there exists $N \in \N$ such that $x_n \in U$ for all $n \geq N$.
\end{defn}

\begin{lemma}
    If $X$ is a Hausdorff space, sequences in the space converge to at most one point.
\end{lemma}

\begin{proof}
    Suppose that $\{x_n\}$ is a sequence which converges to points $x$ and $y$. Since $X$ is Hausdorff, if $x \neq y$ then there exist neighborhoods $U$ and $V$ of $x$ and $y$ respectively such that $U$ and $V$ are disjoint. Since $\{x_n\}$ converges to $x$, the neighborhood $U$ must contain all but finitely many points of the sequence. Since $V$ is disjoint from $U$, it can only contain finitely many points of the sequence, and so the sequence cannot converge to $y$.
\end{proof}

\begin{lemma}
    Given a totally ordered set $X$, the order topology is Hausdorff.
\end{lemma}

\begin{proof}
    Consider distinct points $x$ and $y$, without loss of generality assume $x < y$. If there exists $z$ such that $x < z < y$, then $(-\infty, z)$ and $(z, +\infty)$ are neighborhoods of $x$ and $y$ respectively, and are disjoint.

    If no such $z$ exists, then $(-\infty, y)$ and $(x, +\infty)$ are disjoint neighborhoods.
\end{proof}

\begin{thm}\proofbreak
    \begin{itemize}
        \item The product of two Hausdorff spaces is Hausdorff.
        \item Any subspace of a Hausdorff space is Hausdorff.
    \end{itemize}
\end{thm}

\begin{proof}
    \proofbreak
    \begin{itemize}
        \item Consider two distinct points $(x, y)$ and $(p, q)$. Since $X$ and $Y$ are both Hausdorff, there exist disjoint neighborhoods $U, V \subseteq X$ of $x$ and $p$ respectively, and $W, Z \subseteq Y$ of $y$ and $q$ respectively. Then $U \times W$ and $V \times Z$ are open neighborhoods in $X \times Y$ which are necessarily disjoint, and contain $(x, y)$ and $(p, q)$ respectively.
        \item Suppose $Y \subseteq X$ is a subspace of $X$, and $p, q \in Y$. Since $X$ is Hausdorff, there exist disjoint neighborhoods $U, V \subseteq X$ such that $p \in U$ and $q \in V$. Then $p \in U \intersect Y$ and $q \in V \intersect Y$, and of course $(U \intersect Y)$ and $(V \intersect Y)$ are disjoint neighborhoods in $Y$.
    \end{itemize}
\end{proof}

\section{Continuity and Homeomorphisms}

\begin{defn}
    Let $X$ and $Y$ be topological spaces, and consider a function $f: X \to Y$. We say $f$ is \emph{continuous} when $f^{-1}(O) \subseteq X$ is open for every open set $O \subseteq Y$.
\end{defn}

\begin{lemma}
    If $\mathcal{B}$ is a basis for $Y$, and $f^{-1}(B) \subseteq X$ is open for every $B \in \mathcal{B}$, then $f$ is continuous.

    If $\mathcal{S}$ is a subbasis for $Y$, and $f^{-1}(S)$ is open for every $S \in \mathcal{S}$, then $f$ is continuous.
\end{lemma}

\begin{proof}
    Any open set $O \subseteq Y$ can be expressed as an arbitrary union of basis elements $B_{\alpha} \in \mathcal{B}$. Then,
    \begin{align*}
        f^{-1}(O) = f^{-1}\left(\bigunion B_{\alpha}\right) = \bigunion f^{-1}(B_{\alpha})
    \end{align*}
    is a union of open sets in $X$, and therefore must be open.

    Any basis element $B$ can be expressed as a finite intersection of subbasis elements $S_{i} \in S$, and so
    \begin{align*}
        f^{-1}(B) = f^{-1}\left(\bigintersection S_{i}\right) = \bigintersection f^{-1}(S_i).
    \end{align*}
    If $f^{-1}(S_i)$ is open, then $f^{-1}(B)$ is a finite intersection of open sets and therefore is open.
\end{proof}

\begin{thm} Let $X$ and $Y$ be topological spaces, and $f: X \to Y$ a function between them. Then the following are equivalent:
    \begin{enumerate}[label=(\arabic*)]
        \item $f$ is continuous,
        \item For every $A \subseteq X$, $f(\closure{A})$ is a subset of the closure of $f(A)$.
        \item $f^{-1}(B) \subseteq X$ is closed for every closed set $B \subseteq Y$.
        \item Given any point $x \in X$ and neighborhood $V \subseteq Y$ of $f(x)$, there exists some neighborhood $U$ of $x$ such that $f(U) \subseteq V$.
    \end{enumerate}
\end{thm}

\begin{proof}
    First, we prove that $(1) \implies (2) \implies (3) \implies (1)$. Suppose that $f$ is continuous, and consider some $A \subseteq X$. For any $x \in \closure{A}$, let $V$ be any neighborhood of $f(x)$. Since $f$ is continuous, $f^{-1}(V)$ is a neighborhood of $x$, and so it must intersect $A$ at a point $y$. Then $V$ must intersect $f(y) \in f(A)$, and so $f(x) \in \closure{f(A)}$.

    Suppose instead that the second property holds, and let $B$ be a closed subset of $Y$, and let $A = f^{-1}(B)$. Then $f(A) = B$, and so $\closure{f(A)} = B$. It follows that $f(\closure{A}) \subseteq B$, and so $\closure{A} = A$, and so by Lemma \ref{lemma:closure-closed} it follows that $f^{-B}$ is closed.

    Suppose instead that the third property holds. Then for any open set $A \subseteq Y$, its complement $Y \setminus A$ is closed, and so $f^{-1}(Y \setminus A)$ is closed. But $f^{-1}(Y \setminus A) = f^{-1}(Y) \setminus f{-1}(A)$. Since $f^{-1}(Y) = X$, it follows that $f^{-1}(A)$ is the complement of the closed set $f^{-1}(Y \setminus A)$, and therefore must be open.

    Now we prove that $(1) \iff (4)$. If $f$ is continuous, then $U = f^{-1}(V)$ is an open neighborhood of $x$, and of course $f(U) = V$. If instead we assume that property four holds, let $O$ be an arbitrary open set in $Y$. For each $x \in V = f^{-1}(O)$, there exists a neighborhood $U_x$ of $x$ such that $f(U_x) \subseteq O$. Therefore,
    \begin{align*}
        f^{-1}(O) = \bigunion_{x \in V}U_x
    \end{align*}
    is a union of open sets, and therefore is open.
\end{proof}

\begin{defn}
    A function $f: X \to Y$ is a \emph{homeomorphism} when $f$ is continuous, and $f^{-1}$ both exists and is continuous.
\end{defn}

\begin{rmk}
    Equivalently, a homeomorphism is a bijection $f: X \to Y$ such that $f(U)$ is open if and only if $U$ is open. Therefore, a homeomorphism can also be viewed as a bijection between open sets.
\end{rmk}

\begin{exmp}
    The function $f(x) = 4x - 2$ from $\R \to \R$ is a homeomorphism under the standard topology.
\end{exmp}

\begin{exmp}
    The function $f(t) = (\cos(t), \sin(t))$ from $[0, 2\pi) \subseteq \R$ to the unit circle $S^1 \subseteq \R^2$ is not a homeomorphism, since $[0, t)$ is open in $[0, 2\pi)$, but $f([0, t))$ is not open in $\S^1$.
\end{exmp}

\begin{defn}
    If $f: X \to Y$ is a homeomorphism between $X$ and $f(X)$, we say that $f$ is a topological embedding of $X$ into $Y$.
\end{defn}

\begin{thm}
    If $f: X \to Y$ and $g: Y \to Z$ are continuous, then so is $g \circ f: X \to Z$.
\end{thm}

\begin{proof}
    Let $U$ be an open set in $Z$, then $V = g^{-1}(U)$ is open since $g$ is continuous, and so $f^{-1}(V)$ is open since $f$ is continuous. Therefore, $(g \circ f)^{-1}(U)$ is open in $X$.
\end{proof}

\section{Metric Topology}

\begin{defn}
    A \emph{metric} on a set $X$ is a function $d: X \times X \to \R_{\geq 0}$ such that for all $p, q, r \in X$ we have
    \begin{itemize}
        \item $d(p, q) = 0 \iff p = q$,
        \item $d(p, q) = d(q, p)$,
        \item $d(p, r) \leq d(p, q) + d(q, r)$.
    \end{itemize}
\end{defn}

\begin{lemma}\label{lemma:inner-ball}
    Let $X$ be a set endowed with metric $d$. Let $B(x, r)$ be the ball centered at $x$ with radius $r$ --- that is,
    \begin{align*}
        B(x, r) = \left\{y \in X : d(x, y) < r\right\}.
    \end{align*}
    Then for any $y \in B(x, r)$, there exists $\delta \in \R^+$ such that $B(y, \delta) \subseteq B(x, r)$.
\end{lemma}

\begin{proof}
    Let $\delta = r - d(y, x)$. Then for any $z \in B(y, \delta)$, we have
    \begin{align*}
        d(z, x) \leq d(z, y) + d(y, x) < \delta + d(y, x) = r - d(y, x) + d(y, x) = r,
    \end{align*}
    and so $z \in B(x, r)$.
\end{proof}

\begin{lemma}
    Let $X$ be a set endowed with metric $d$. The set
    \begin{align*}
        \left\{B(x, r) : x \in X, r \in \R^{+}\right\}
    \end{align*}
    is a basis.
\end{lemma}

\begin{proof}
    Let $x$ be any point in $X$, then $B(x, 1)$ is a ball containing $x$.

    Suppose $B(u, s)$ and $B(v, t)$ are two balls which both contain a point $x$. By Lemma \ref{lemma:inner-ball}, let $B(x, \delta_1)$ and $B(x, \delta_2)$ be such that $B(x, \delta_1) \subseteq B(u, s)$ and $B(x, \delta_2) \subseteq B(v, t)$. Then letting $\varepsilon = \min(\delta_1,\delta_2)$, it follows that
    \begin{align*}
        B(x, \varepsilon) \subseteq B(u, s) \intersect B(v, t).
    \end{align*}
\end{proof}

\begin{defn}
    Given a set $X$ endowed with metric $d$, the corresponding \emph{metric topology} is generated by the basis of open balls.
\end{defn}

\begin{exmp}
    Let $X$ be any set, and consider the metric
    \begin{align*}
        d(x, y) = \begin{dcases}
            0, &x = y \\
            1, &x \neq y
        \end{dcases}.
    \end{align*}
    The corresponding metric topology will be the discrete topology.
\end{exmp}

\begin{lemma}
    A set $U$ is open in the metric topology if and only if for any $x \in U$ there exists $\delta > 0$ such that $B(x, \delta) \subseteq U$.
\end{lemma}

\begin{proof}
    If such $\delta$ exists, then $U$ is open by definition. Conversely, if $U$ is open then for any $x \in U$ there must exist $B(y, \varepsilon)$ such that $x \in B(y, \varepsilon) \subseteq U$. By Lemma \ref{lemma:inner-ball}, it follows that there exists $\delta$ such that $B(x, \delta) \subseteq B(y, \varepsilon)$.
\end{proof}

\begin{defn}
    A topology $\topo$ is \emph{metrizable} when there exists a metric $d$ such that the corresponding metric topology is the same as $\topo$.
\end{defn}

\begin{defn}
    Given a set $X$ with metric $d$, the metric
    \begin{align*}
        \bar{d}(x, y) = \min(d(x, y), 1)
    \end{align*}
    is the \emph{standard bounded metric}.
\end{defn}

\begin{rmk}
    The standard bounded metric induces the same metric topology as the original metric.
\end{rmk}

\begin{lemma}\label{lemma:finer-metric-topology}
    Let $d, d'$ be two metrics on a set $X$, and let $\topo, \topo'$ be the metric topologies they induce. $\topo'$ is finer than $\topo$ if and only if for all $B(x, \varepsilon) \in \topo$ there exists $B(x, \delta) \subseteq B(x, \varepsilon)$.
\end{lemma}

\begin{proof}
    Follows from Lemma \ref{lemma:subbasis-finer-topology} and Lemma \ref{lemma:inner-ball}.
\end{proof}

\begin{thm}
    The topologies induced by the $\ell_p$ (including $\ell_{\infty}$) metrics on $\R^n$ are all equivalent to the product topology.
\end{thm}

\begin{proof}
    Since
    \begin{align*}
        \norm{x}_{\infty} \leq \norm{x}_{p} \leq n^{1/p}\norm{x}_{\infty},
    \end{align*}
    it follows that
    \begin{align*}
        B_{\infty}(x, \delta) \subseteq B_{p}(x, \delta) \subseteq B_{\infty}(x, n^{1/p}\delta),
    \end{align*}
    and so by Lemma \ref{lemma:finer-metric-topology} that these topologies are all the same.

    Now we prove that $\ell_{\infty}$ induces the product topology. Let $B$ be any basis element in the product topology, so $B = (a_1, b_1) \times (a_2, b_2) \times \cdots \times (a_n, b_n)$. Let $x \in B$, and take $\delta_i$ such that $x + \delta_i < b_i$ and $x - \delta_i > a_i$. Let $\delta = \min(\delta_i)$, and then $B_{\infty}(x, \delta) \subseteq B$. Conversely, any basis element $B_{\infty}(x, \delta)$ in the metric topology is itself a basis element in the product topology. It follows by Lemma \ref{lemma:finer-metric-topology} that these topologies are the same.
\end{proof}

\section{Quotient Topology}


